\chapter{Кластеризация}
\section{Введение}
Кластер (в информацонных технологиях)~--- группа серверов (программных или аппаратных), объединённых логически, 
способных обрабатывать идентичные запросы и использующихся как единый ресурс. Для PostgreSQL это означает, что несколько серверов 
баз данных ведут себя как одна база данных. В большинстве случаев, кластеры серверов функционируют на раздельных компьютерах. 
Это позволяет повышать производительность за счёт распределения нагрузки на аппаратные ресурсы и обеспечивает отказоустойчивость 
на аппаратном уровне.

Для создания кластера PostgreSQL существует несколько решений:
\begin{itemize}
\item \textbf{Greenplum Database}\footnote{http://www.greenplum.com/index.php?page=greenplum-database}
\item \textbf{GridSQL for EnterpriseDB Advanced Server}\footnote{http://www.enterprisedb.com/products/gridsql.do}
\item \textbf{PL/Proxy}\footnote{http://plproxy.projects.postgresql.org/doc/tutorial.html}
\item \textbf{HadoopDB}\footnote{http://db.cs.yale.edu/hadoopdb/hadoopdb.html}
\end{itemize}