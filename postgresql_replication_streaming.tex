\section{Streaming Replication (Потоковая репликация)}
\subsection{Введение}
Потоковая репликация (Streaming Replication, SR) дает возможность непрерывно отправлять и применять 
wall xlog записи на резервные сервера для создания точной копии текущего. Данная функциональность 
появилась у PostgreSQL начиная с 9 версии (репликация из коробки!). Этот тип репликации простой, надежный и, вероятней всего,  
будет использоваться в качестве стандартной репликации в большинстве высоконагруженых приложений, что используют PostgreSQL. 

Отличительными особенностями решения являются:
\begin{itemize}
\item репликация всего кластера PostgreSQL
\item асинхронный механизм репликации
\item простота установки
\item мастер база данных может обслуживать огромное количество слейвов из-за минимальной нагрузки
\end{itemize}

К недостаткам можно отнести:
\begin{itemize}
\item невозможность реплицировать только определенную базу данных из всех на кластере 
\item асинхронный механизм~--- слейв отстает от мастера (но в отличие от других методов репликации, 
это отставание очень короткое, и может составлять всего лишь одну транзакцию, в зависимости от скорости сети, 
нагружености БД и настроек <<Hot Standby>>~---<<Горячий резерв>>)
\end{itemize}

\subsection{Установка}
Для начала нам потребуется PostgreSQL не ниже 9 версии. В момент написания этой главы была доступна 9.0.1 версия. 
Все работы, как пологается, будут проводится на ОС Linux. 

\subsection{Настройка}
Для начала обозначим мастер сервер как masterdb и слейв как slavedb.
