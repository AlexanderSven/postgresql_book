\section{Streaming Replication (Потоковая репликация)}
\subsection{Введение}
Потоковая репликация (Streaming Replication, SR) дает возможность непрерывно отправлять и применять 
wall xlog записи на резервные сервера для создания точной копии текущего. Данная функциональность 
появилась у PostgreSQL начиная с 9 версии (репликация из коробки!). Этот тип репликации простой, надежный и, вероятней всего,  
будет использоваться в качестве стандартной репликации в большинстве высоконагруженых приложений, что используют PostgreSQL. 

Отличительными особенностями решения являются:
\begin{itemize}
\item репликация всего кластера PostgreSQL
\item асинхронный механизм репликации
\item простота установки
\item мастер база данных может обслуживать огромное количество слейвов из-за минимальной нагрузки
\end{itemize}

К недостаткам можно отнести:
\begin{itemize}
\item невозможность реплицировать только определенную базу данных из всех на кластере 
\item асинхронный механизм~--- слейв отстает от мастера (но в отличие от других методов репликации, 
это отставание очень короткое, и может составлять всего лишь одну транзакцию, в зависимости от скорости сети, 
нагружености БД и настроек <<Hot Standby>>~---<<Горячий резерв>>)
\end{itemize}

\subsection{Установка}
Для начала нам потребуется PostgreSQL не ниже 9 версии. В момент написания этой главы была доступна 9.0.1 версия. 
Все работы, как пологается, будут проводится на ОС Linux. 

\subsection{Настройка}
Для начала обозначим мастер сервер как masterdb(192.168.0.10) и слейв как slavedb(192.168.0.20).

\subsubsection{Предварительная настройка}
Для начала позволим определенному пользователю без пароля ходить по ssh. Пусть это будет postgres юзер. 
Если же нет, то создаем набором команд:
\begin{lstlisting}[label=lst:streaming1,caption=Создаем пользователя userssh]
$sudo groupadd userssh
$sudo useradd -m -g userssh -d /home/userssh -s /bin/bash \
-c "user ssh allow" userssh
\end{lstlisting}

Дальше выполняем команды от имени пользователя (в данном случае postgres):
\begin{lstlisting}[label=lst:streaming2,caption=Логинимся под пользователем postgres]
su postgres
\end{lstlisting}

Генерим RSA-ключ для обеспечения аутентификации в условиях отсутствия возможности использовать пароль:
\begin{lstlisting}[label=lst:streaming3,caption=Генерим RSA-ключ]
postgres@localhost ~ $ ssh-keygen -t rsa -P ""
Generating public/private rsa key pair.
Enter file in which to save the key (/var/lib/postgresql/.ssh/id_rsa): 
Created directory '/var/lib/postgresql/.ssh'.
Your identification has been saved in /var/lib/postgresql/.ssh/id_rsa.
Your public key has been saved in /var/lib/postgresql/.ssh/id_rsa.pub.
The key fingerprint is:
16:08:27:97:21:39:b5:7b:86:e1:46:97:bf:12:3d:76 postgres@localhost
\end{lstlisting}

И добавляем его в список авторизованных ключей:
\begin{lstlisting}[label=lst:streaming4,caption=Добавляем его в список авторизованных ключей]
cat $HOME/.ssh/id_rsa.pub >> $HOME/.ssh/authorized_keys
\end{lstlisting}

Этого должно быть более чем достаточно, проверить работоспособность соединения можно просто написав:
\begin{lstlisting}[label=lst:streaming5,caption=Пробуем зайти на ssh без пароля]
ssh localhost
\end{lstlisting}

Не забываем предварительно инициализировать sshd:
\begin{lstlisting}[label=lst:streaming6,caption=Запуск sshd]
/etc/init.d/sshd start
\end{lstlisting}

После успешно проделаной операции скопируйте <<\$HOME/.ssh>> на slavedb. 
Теперь мы должны иметь возможность без пароля заходить с мастера на слейв и со слейва на мастер через ssh.

Также отредактируем pg\_hba.conf на мастере и слейве, разрешив им друг к другу доступ без пароля(trust):
\begin{lstlisting}[label=lst:streaming7,caption=Мастер pg\_hba.conf]
host  all  all  192.168.0.20/32  trust
\end{lstlisting}
\begin{lstlisting}[label=lst:streaming8,caption=Слейв pg\_hba.conf]
host  all  all  192.168.0.10/32  trust
\end{lstlisting}

Не забываем после этого перегрузить postgresql на обоих серверах.

\subsubsection{Настройка мастера}
Для начала настроим masterdb. Установим параметры в postgresql.conf для репликации:
\begin{lstlisting}[label=lst:streaming9,caption=Настройка мастера]
# To enable read-only queries on a standby server, wal_level must be set to
# "hot_standby". But you can choose "archive" if you never connect to the
# server in standby mode.
wal_level = hot_standby

# Set the maximum number of concurrent connections from the standby servers.
max_wal_senders = 5

# To prevent the primary server from removing the WAL segments required for
# the standby server before shipping them, set the minimum number of segments
# retained in the pg_xlog directory. At least wal_keep_segments should be
# larger than the number of segments generated between the beginning of
# online-backup and the startup of streaming replication. If you enable WAL
# archiving to an archive directory accessible from the standby, this may
# not be necessary.
wal_keep_segments = 32

# Enable WAL archiving on the primary to an archive directory accessible from
# the standby. If wal_keep_segments is a high enough number to retain the WAL
# segments required for the standby server, this may not be necessary.
archive_mode    = on
archive_command = 'cp %p /path_to/archive/%f'
\end{lstlisting}

Давайте по порядку:
\begin{itemize}
\item <<wal\_level = hot\_standby>>~--- сервер начнет писать в WAL логи так же как и при режиме <<archive>>, 
добавляя информацию, необходимую для востановления транзакции (можно также поставить <<archive>>, 
но тогда сервер не может быть слейвом при необходимости).
\item <<max\_wal\_senders = 5>>~--- максимальное количество слейвов.
\item <<wal\_keep\_segments = 32>>~--- минимальное количество файлов c WAL сегментами в pg\_xlog директории.
\item <<archive\_mode = on>>~--- позволяем сохранять WAL сегменты в указаное переменной <<archive\_command>> хранилище. 
В данном случае в директорию <</path\_to/archive/>>.
\end{itemize}

После изменения параметров перегружаем postgresql сервер. Теперь перейдем к slavedb.

\subsubsection{Настройка слейва}
Для начала нам потребуется создать на slavedb точную копию masterdb. Перенесем данные с помощью <<Онлайн бекапа>>.

Для начала зайдем на masterdb сервер. Выполним в консоли:
\begin{lstlisting}[label=lst:streaming10,caption=Выполняем на мастере]
psql -c "SELECT pg_start_backup('label', true)" 
\end{lstlisting}

Теперь нам нужно перенести данные с мастера на слейв. Выполняем на мастере:
\begin{lstlisting}[label=lst:streaming11,caption=Выполняем на мастере]
rsync -C -a --delete -e ssh --exclude postgresql.conf --exclude postmaster.pid \
--exclude postmaster.opts --exclude pg_log --exclude pg_xlog \
--exclude recovery.conf master_db_datadir/ slavedb_host:slave_db_datadir/ 
\end{lstlisting}
где
\begin{itemize}
\item <<master\_db\_datadir>>~--- директория с postgresql данными на masterdb
\item <<slave\_db\_datadir>>~--- директория с postgresql данными на slavedb
\item <<slavedb\_host>>~--- хост slavedb(в нашем случае - 192.168.1.20)
\end{itemize}

После копирования данных с мастера на слейв, остановим онлайн бекап. Выполняем на мастере:
\begin{lstlisting}[label=lst:streaming12,caption=Выполняем на мастере]
psql -c "SELECT pg_stop_backup()"
\end{lstlisting}

Устанавливаем такие же данные в конфиге postgresql.conf, что и у мастера (чтобы при падении мастера слейв мог его заменить). 
Так же установим допольнительный параметр:
\begin{lstlisting}[label=lst:streaming13,caption=Конфиг слейва]
hot_standby = on
\end{lstlisting}
Внимание! Если на мастере поставили <<wal\_level = archive>>, тогда параметр оставляем по умолчанию (hot\_standby = off).

Далее на slavedb в директории с данными PostgreSQL создадим файл recovery.conf с таким содержимым:
\begin{lstlisting}[label=lst:streaming14,caption=Конфиг recovery.conf]
# Specifies whether to start the server as a standby. In streaming replication,
# this parameter must to be set to on.
standby_mode          = 'on'

# Specifies a connection string which is used for the standby server to connect
# with the primary.
primary_conninfo      = 'host=192.168.0.10 port=5432 user=postgres'

# Specifies a trigger file whose presence should cause streaming replication to
# end (i.e., failover).
trigger_file = '/path_to/trigger'

# Specifies a command to load archive segments from the WAL archive. If
# wal_keep_segments is a high enough number to retain the WAL segments
# required for the standby server, this may not be necessary. But
# a large workload can cause segments to be recycled before the standby
# is fully synchronized, requiring you to start again from a new base backup.
restore_command = 'scp masterdb_host:/path_to/archive/%f "%p"'
\end{lstlisting}

где
\begin{itemize}
\item <<standby\_mode='on'>>~--- указываем серверу работать в режиме слейв
\item <<primary\_conninfo>>~--- настройки соединения слейва с мастером
\item <<trigger\_file>>~--- указываем триггер-файл, при наличии которого будет остановлена репликация.
\item <<restore\_command>>~--- команда, которой будет востанавливатся WAL логи. В нашем случае через 
scp копируем с masterdb (masterdb\_host - хост masterdb).
\end{itemize}

Теперь мы можем запустить PostgreSQL на slavedb.

\subsubsection{Тестирование репликации}
Теперь мы можем посмотреть отставание слейвов от мастера с помощью таких команд:
\begin{lstlisting}[label=lst:streaming15,caption=Тестирование репликации]
$ psql -c "SELECT pg_current_xlog_location()" -h192.168.0.10 (masterdb)
 pg_current_xlog_location 
--------------------------
 0/2000000
(1 row)

$ psql -c "select pg_last_xlog_receive_location()" -h192.168.0.20 (slavedb)
 pg_last_xlog_receive_location 
-------------------------------
 0/2000000
(1 row)

$ psql -c "select pg_last_xlog_replay_location()" -h192.168.0.20 (slavedb)
 pg_last_xlog_replay_location 
------------------------------
 0/2000000
(1 row)
\end{lstlisting}

Еще проверить работу репликации можно с помощью утилиты ps:
\begin{lstlisting}[label=lst:streaming16,caption=Тестирование репликации]
[masterdb] $ ps -ef | grep sender
postgres  6879  6831  0 10:31 ?        00:00:00 postgres: wal sender process postgres 127.0.0.1(44663) streaming 0/2000000

[slavedb] $ ps -ef | grep receiver
postgres  6878  6872  1 10:31 ?        00:00:01 postgres: wal receiver process   streaming 0/2000000
\end{lstlisting}

\subsection{Общие задачи}
\subsubsection{Переключение на слейв при падении мастера}
Достаточно создать триггер-файл (trigger\_file) на слейве, который становится мастером.

\subsubsection{Остановка репликации на слейве}
Создать триггер-файл (trigger\_file) на слейве.

\subsubsection{Перезапуск репликации после сбоя}
Повторяем операции из раздела <<Настройка слейва>>. Хочется заметить, 
что мастер при этом не нуждается в остановке при выполнении данной задачи.

\subsubsection{Перезапуск репликации после сбоя слейва}
Перезагрузить PostgreSQL на слейве после устранения сбоя.

\subsubsection{Повторно синхронизировать репликации на слейве}
Это может потребоватся, например, после длительного отключения от мастера. 
Для этого останавливаем PostgreSQL на слейве и повторяем операции из раздела <<Настройка слейва>>.