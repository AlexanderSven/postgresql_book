\section{Введение}
Скорость работы, вообще говоря, не является основной причиной использования реляционных СУБД. Более того, первые реляционные базы работали медленнее своих предшественников.
Выбор этой технологии был вызван скорее:

\begin{itemize}
  \item возможностью возложить поддержку целостности данных на СУБД;
  \item независимостью логической структуры данных от физической.
\end{itemize}

Эти особенности позволяют сильно упростить написание приложений, но требуют для своей реализации дополнительных ресурсов.

Таким образом, прежде чем искать ответ на вопрос <<как заставить РСУБД работать быстрее в моей задаче?>>, следует ответить на вопрос <<нет ли более подходящего средства для решения моей задачи, чем РСУБД?>> Иногда использование другого средства потребует меньше усилий, чем настройка производительности.

Данная глава посвящена возможностям повышения производительности PostgreSQL. Глава не претендует на исчерпывающее изложение вопроса, наиболее полным и точным руководством по использованию PostgreSQL является, конечно, официальная документация и официальный FAQ. Также существует англоязычный список рассылки postgresql-performance, посвящённый именно этим вопросам. Глава состоит из двух разделов, первый из которых ориентирован скорее на администратора, второй~--- на разработчика приложений. Рекомендуется прочесть оба раздела: отнесение многих вопросов к какому-то одному из них весьма условно.


\subsection{Не используйте настройки по умолчанию}

По умолчанию PostgreSQL сконфигурирован таким образом, чтобы он мог быть запущен практически на любом компьютере и не слишком мешал при этом работе других приложений. Это особенно касается используемой памяти. Настройки по умолчанию подходят только для следующего использования: с ними вы сможете проверить, работает ли установка PostgreSQL, создать тестовую базу уровня записной книжки и потренироваться писать к ней запросы. Если вы собираетесь разрабатывать (а тем более запускать в работу) реальные приложения, то настройки придётся радикально изменить. В дистрибутиве PostgreSQL, к сожалению, не поставляется файлов с <<рекомендуемыми>> настройками. Вообще говоря, такие файлы создать весьма сложно, т.к. оптимальные настройки конкретной установки PostgreSQL будут определяться:

\begin{itemize}
  \item конфигурацией компьютера;
  \item объёмом и типом данных, хранящихся в базе;
  \item отношением числа запросов на чтение и на запись;
  \item тем, запущены ли другие требовательные к ресурсам процессы (например, веб-сервер).
\end{itemize}


\subsection{Используйте актуальную версию сервера}
Если у вас стоит устаревшая версия PostgreSQL, то наибольшего ускорения работы вы сможете добиться, обновив её до текущей. Укажем лишь наиболее значительные из связанных с производительностью изменений.

\begin{itemize}
  \item В версии 7.1 появился журнал транзакций, до того данные в таблицу сбрасывались каждый раз при успешном завершении транзакции;
  \item В версии 7.2 появились:
  \begin{itemize}
    \item новая версия команды VACUUM, не требующая блокировки;
    \item команда ANALYZE, строящая гистограмму распределения данных в столбцах, что позволяет выбирать более быстрые планы выполнения запросов;
    \item подсистема сбора статистики;
  \end{itemize}
  \item В версии 7.4 была ускорена работа многих сложных запросов (включая печально известные подзапросы IN/NOT IN);
  \item В версии 8.0 были внедрены метки восстановления, улучшение управления буфером, CHECKPOINT и VACUUM улучшены;
  \item В версии 8.1 был улучшен одновременный доступ к разделяемой памяти, автоматическое использование индексов для MIN() и MAX(), pg\_autovacuum внедрен в сервер (автоматизирован), повышение производительности для секционированных таблиц;
  \item В версии 8.2 была улучшена скорость множества SQL запросов, усовершенствован сам язык запросов;
  \item В версии 8.3 внедрен полнотекстовый поиск, поддержка SQL/XML стандарта, параметры конфигурации сервера могут быть установлены на основе отдельных функций;
  \item В версии 8.4 были внедрены общие табличные выражения, рекурсивные запросы, параллельное восстановление, улучшена производительность для EXISTS/NOT EXISTS запросов;
  \item В версии 9.0 <<асинхронная репликация из коробки>>, VACUUM/VACUUM FULL стали быстрее, расширены хранимые процедуры;
  \item В версии 9.1 <<синхронная репликация из коробки>>, нелогируемые таблицы (очень быстрые на запись, но при падении БД данные могут пропасть), новые типы индексов, наследование таблиц в запросах теперь может вернуться многозначительно отсортированные результаты, позволяющие оптимизации MIN/MAX;
  \item В версии 9.2 <<каскадная репликация из коробки>>, сканирование по индексу, JSON тип данных, типы данных на диапазоны, сортировка в памяти улучшена на 25\%, ускорена команда COPY;
  \item В версии 9.3 materialized view, доступные на запись внешние таблицы, переход с использования SysV shared memory на POSIX shared memory и mmap, cокращено время распространения реплик, а также значительно ускорена передача управления от запасного сервера к первичному, увеличена производительность и улучшена система блокировок для внешних ключей;
  \item В версии 9.4 появился новый тип поля JSONB (бинарный JSON с поддержкой индексов), логическое декодирование для репликации, GIN индекс в 2 раза меньше по размеру и в 3 раза быстрее, неблокируюшие обновление materialized view, поддержка Linux Huge Pages;
  \item В версии 9.5 добавленна поддержка UPSERT, Row Level Security, CUBE, ROLLUP, GROUPING SETS функции, TABLESAMPLE, BRIN индекс, ускорена скорость работы индексов для текстовых и цифровых полей;
\end{itemize}

Следует также отметить, что большая часть изложенного в статье материала относится к версии сервера не ниже 9.0.


\subsection{Стоит ли доверять тестам производительности}

Перед тем, как заниматься настройкой сервера, вполне естественно ознакомиться с опубликованными данными по производительности, в том числе в сравнении с другими СУБД. К сожалению, многие тесты служат не столько для облегчения вашего выбора, сколько для продвижения конкретных продуктов в качестве <<самых быстрых>>. При изучении опубликованных тестов в первую очередь обратите внимание, соответствует ли величина и тип нагрузки, объём данных и сложность запросов в тесте тому, что вы собираетесь делать с базой? Пусть, например, обычное использование вашего приложения подразумевает несколько одновременно работающих запросов на обновление к таблице в миллионы записей. В этом случае СУБД, которая в несколько раз
быстрее всех остальных ищет запись в таблице в тысячу записей, может оказаться не лучшим выбором. Ну и наконец, вещи, которые должны сразу насторожить:

\begin{itemize}
  \item Тестирование устаревшей версии СУБД;
  \item Использование настроек по умолчанию (или отсутствие информации о настройках);
  \item Тестирование в однопользовательском режиме (если, конечно, вы не предполагаете использовать СУБД именно так);
  \item Использование расширенных возможностей одной СУБД при игнорировании расширенных возможностей другой;
  \item Использование заведомо медленно работающих запросов (см. <<\ref{sec:pg-optimize-sql}~\nameref{sec:pg-optimize-sql}>>);
\end{itemize}

