\section{Примеры настроек}

\subsection{Среднестатистическая настройка для максимальной производительности}

Возможно для конкретного случая лучше подойдут другие настройки. Внимательно изучите данное руководство и настройте
PostgreSQL опираясь на эту информацию.

RAM~--- размер памяти;
\begin{itemize}
\item shared\_buffers = 1/8 RAM или больше (но не более 1/4);
\item work\_mem в 1/20 RAM;
\item maintenance\_work\_mem в 1/4 RAM;
\item max\_fsm\_relations в планируемое кол--во таблиц в базах * 1.5;
\item max\_fsm\_pages в max\_fsm\_relations * 2000;
\item fsync = true;
\item wal\_sync\_method = fdatasync;
\item commit\_delay = от 10 до 100 ;
\item commit\_siblings = от 5 до 10;
\item effective\_cache\_size = 0.9 от значения cached, которое показывает free;
\item random\_page\_cost = 2 для быстрых cpu, 4 для медленных;
\item cpu\_tuple\_cost = 0.001 для быстрых cpu, 0.01 для медленных;
\item cpu\_index\_tuple\_cost = 0.0005 для быстрых cpu, 0.005 для медленных;
\item autovacuum = on;
\item autovacuum\_vacuum\_threshold = 1800;
\item autovacuum\_analyze\_threshold = 900;
\end{itemize}

\subsection{Среднестатистическая настройка для оконного приложения (1С), 2 ГБ памяти}

\begin{itemize}
\item maintenance\_work\_mem = 128MB
\item effective\_cache\_size = 512MB
\item work\_mem = 640kB
\item wal\_buffers = 1536kB
\item shared\_buffers = 128MB
\item max\_connections = 500
\end{itemize}

\subsection{Среднестатистическая настройка для Web приложения, 2 ГБ памяти}

\begin{itemize}
\item maintenance\_work\_mem = 128MB;
\item checkpoint\_completion\_target = 0.7
\item effective\_cache\_size = 1536MB
\item work\_mem = 4MB
\item wal\_buffers = 4MB
\item checkpoint\_segments = 8
\item shared\_buffers = 512MB
\item max\_connections = 500
\end{itemize}

\subsection{Среднестатистическая настройка для Web приложения, 8 ГБ памяти}

\begin{itemize}
\item maintenance\_work\_mem = 512MB
\item checkpoint\_completion\_target = 0.7
\item effective\_cache\_size = 6GB
\item work\_mem = 16MB
\item wal\_buffers = 4MB
\item checkpoint\_segments = 8
\item shared\_buffers = 2GB
\item max\_connections = 500
\end{itemize}

\section{Автоматическое создание оптимальных настроек: pgtune}

Для оптимизации настроек для PostgreSQL Gregory Smith создал утилиту pgtune\footnote{http://pgtune.projects.postgresql.org/}
в расчёте на обеспечение максимальной производительности для заданной аппаратной конфигурации.
Утилита проста в использовании и во многих Linux системах может идти в составе пакетов.
Если же нет, можно просто скачать архив и распаковать.
Для начала:
\begin{lstlisting}[label=lst:p_settings1,caption=Pgtune]
pgtune -i $PGDATA/postgresql.conf \
-o $PGDATA/postgresql.conf.pgtune
\end{lstlisting}
опцией
\lstinline!-i, --input-config!
указываем текущий файл postgresql.conf,
а
\lstinline!-o, --output-config!
указываем имя файла для нового postgresql.conf.

Есть также дополнительные опции для настройки конфига.
\begin{itemize}
\item
\lstinline!-M, --memory!
Используйте этот параметр, чтобы определить общий объем системной памяти.
Если не указано, pgtune будет пытаться использовать текущий объем системной памяти.

\item
\lstinline!-T, --type!
Указывает тип базы данных. Опции: DW,  OLTP,  Web, Mixed, Desktop.

\item
\lstinline!-c, --connections!
Указывает максимальное количество соединений. Если он не указан, то будет браться в зависимости от типа базы данных.

\end{itemize}

Хочется сразу добавить, что pgtune не панацея для оптимизации настройки PostgreSQL. Многие настройки зависят не только от
аппаратной конфигурации, но и от размера базы данных, числа соединений и сложности запросов, так что оптимально настроить базу данных
возможно только учитывая все эти параметры.