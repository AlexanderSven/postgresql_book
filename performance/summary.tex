\section{Оптимизация БД и приложения}

Для быстрой работы каждого запроса в вашей базе в основном требуется следующее:

\begin{enumerate}
  \item Отсутствие в базе мусора, мешающего добраться до актуальных данных. Можно сформулировать две подзадачи:
  \begin{enumerate}
    \item Грамотное проектирование базы. Освещение этого вопроса выходит далеко за рамки этой книги;
    \item Сборка мусора, возникающего при работе СУБД;
  \end{enumerate}
  \item Наличие быстрых путей доступа к данным~--- индексов;
  \item Возможность использования оптимизатором этих быстрых путей;
  \item Обход известных проблем;
\end{enumerate}


\subsection{Поддержание базы в порядке}

В данном разделе описаны действия, которые должны периодически выполняться для каждой базы. От разработчика требуется только настроить их автоматическое выполнение (при помощи cron) и опытным путём подобрать оптимальную частоту.


\subsubsection{Команда ANALYZE}

Служит для обновления информации о распределении данных в таблице. Эта информация используется оптимизатором для выбора наиболее быстрого плана выполнения запроса.

Обычно команда используется в связке с \lstinline!VACUUM ANALYZE!. Если в базе есть таблицы, данные в которых не изменяются и не удаляются, а лишь добавляются, то для таких таблиц можно использовать отдельную команду ANALYZE. Также стоит использовать эту команду для отдельной таблицы после добавления в неё большого количества записей.


\subsubsection{Команда REINDEX}

Команда \lstinline!REINDEX! используется для перестройки существующих индексов. Использовать её имеет смысл в случае:

\begin{itemize}
  \item порчи индекса;
  \item постоянного увеличения его размера;
\end{itemize}

Второй случай требует пояснений. Индекс, как и таблица, содержит блоки со старыми версиями записей. PostgreSQL не всегда может заново использовать эти блоки, и поэтому файл с индексом постепенно увеличивается в размерах. Если данные в таблице часто меняются, то расти он может весьма быстро.

Если вы заметили подобное поведение какого-то индекса, то стоит настроить для него периодическое выполнение команды REINDEX. Учтите: команда \lstinline!REINDEX!, как и \lstinline!VACUUM FULL!, полностью блокирует таблицу, поэтому выполнять её надо тогда, когда загрузка сервера минимальна.


\subsection{Использование индексов}

Опыт показывает, что наиболее значительные проблемы с производительностью вызываются отсутствием нужных индексов. Поэтому столкнувшись с медленным запросом, в первую очередь проверьте, существуют ли индексы, которые он может использовать. Если нет~--- постройте их. Излишек индексов, впрочем, тоже чреват проблемами:

\begin{itemize}
  \item Команды, изменяющие данные в таблице, должны изменить также и индексы. Очевидно, чем больше индексов построено для таблицы, тем медленнее это будет происходить;
  \item Оптимизатор перебирает возможные пути выполнения запросов. Если построено много ненужных индексов, то этот перебор будет идти дольше;
\end{itemize}

Единственное, что можно сказать с большой степенью определённости~--- поля, являющиеся внешними ключами, и поля, по которым объединяются таблицы, индексировать надо обязательно.


\subsubsection{Команда EXPLAIN [ANALYZE]}

Команда \lstinline!EXPLAIN [запрос]! показывает, каким образом PostgreSQL собирается выполнять ваш запрос. Команда \lstinline!EXPLAIN ANALYZE [запрос]! выполняет запрос (и поэтому EXPLAIN ANALYZE DELETE \dots~--- не слишком хорошая идея) и показывает как изначальный план, так и реальный процесс его выполнения.

Чтение вывода этих команд~--- искусство, которое приходит с опытом. Для начала обращайте внимание на следующее:

\begin{itemize}
  \item Использование полного просмотра таблицы (seq scan);
  \item Использование наиболее примитивного способа объединения таблиц (nested loop);
  \item Для \lstinline!EXPLAIN ANALYZE!: нет ли больших отличий в предполагаемом количестве записей и реально выбранном? Если оптимизатор использует устаревшую статистику, то он может выбирать не самый быстрый план выполнения запроса;
\end{itemize}

Следует отметить, что полный просмотр таблицы далеко не всегда медленнее просмотра по индексу. Если, например, в таблице--справочнике несколько сотен записей, умещающихся в одном-двух блоках на диске, то использование индекса приведёт лишь к тому, что придётся читать ещё и пару лишних блоков индекса. Если в запросе придётся выбрать 80\% записей из большой таблицы, то полный просмотр опять же получится быстрее.

При тестировании запросов с использованием \lstinline!EXPLAIN ANALYZE! можно воспользоваться настройками, запрещающими оптимизатору использовать определённые планы выполнения. Например,

\begin{lstlisting}[language=SQL,label=lst:summary-explain1,caption=enable\_seqscan]
SET enable_seqscan=false;
\end{lstlisting}

запретит использование полного просмотра таблицы, и вы сможете выяснить, прав ли был оптимизатор, отказываясь от использования индекса. Ни в коем случае не следует прописывать подобные команды в postgresql.conf! Это может ускорить выполнение нескольких запросов, но сильно замедлит все остальные!


\subsubsection{Использование собранной статистики}

Результаты работы сборщика статистики доступны через специальные системные представления. Наиболее интересны для наших целей следующие:

\begin{itemize}
  \item \lstinline!pg_stat_user_tables! содержит~--- для каждой пользовательской таблицы в текущей базе данных~--- общее количество полных просмотров и просмотров с использованием индексов, общие количества записей, которые были возвращены в результате обоих типов просмотра, а также общие количества вставленных, изменённых и удалённых записей;
  \item \lstinline!pg_stat_user_indexes! содержит~--- для каждого пользовательского индекса в текущей базе данных~--- общее количество просмотров, использовавших этот индекс, количество прочитанных записей, количество успешно прочитанных записей в таблице (может быть меньше предыдущего значения, если в индексе есть записи, указывающие на устаревшие записи в таблице);
  \item \lstinline!pg_statio_user_tables! содержит~--- для каждой пользовательской таблицы в текущей базе данных~--- общее количество блоков, прочитанных из таблицы, количество блоков, оказавшихся при этом в буфере (см. пункт 2.1.1), а также аналогичную статистику для всех индексов по таблице и, возможно, по связанной с ней таблицей TOAST;
\end{itemize}

Из этих представлений можно узнать, в частности:

\begin{itemize}
 \item Для каких таблиц стоит создать новые индексы (индикатором служит большое количество полных просмотров и большое количество прочитанных блоков);
 \item Какие индексы вообще не используются в запросах. Их имеет смысл удалить, если, конечно, речь не идёт об индексах, обеспечивающих выполнение ограничений PRIMARY KEY и UNIQUE;
 \item Достаточен ли объём буфера сервера;
\end{itemize}

Также возможен <<дедуктивный>> подход, при котором сначала создаётся большое количество индексов, а затем неиспользуемые индексы удаляются.

\subsection{Перенос логики на сторону сервера}

Этот пункт очевиден для опытных пользователей PostrgeSQL и предназначен для тех, кто использует или переносит на PostgreSQL приложения, написанные изначально для более примитивных СУБД.

Реализация части логики на стороне сервера через хранимые процедуры, триггеры, правила\footnote{RULE~--- реализованное в PostgreSQL расширение стандарта SQL, позволяющее, в частности, создавать обновляемые представления} часто позволяет ускорить работу приложения. Действительно, если несколько запросов объединены в процедуру, то не требуется

\begin{itemize}
  \item пересылка промежуточных запросов на сервер;
  \item получение промежуточных результатов на клиент и их обработка;
\end{itemize}

Кроме того, хранимые процедуры упрощают процесс разработки и поддержки: изменения надо вносить только на стороне сервера, а не менять запросы во всех приложениях.

\subsection{Оптимизация конкретных запросов}
\label{sec:pg-optimize-sql}

В этом разделе описываются запросы, для которых по разным причинам нельзя заставить оптимизатор использовать индексы, и которые будут всегда вызывать полный просмотр таблицы. Таким образом, если вам требуется использовать эти запросы в требовательном к быстродействию приложении, то придётся их изменить.

\subsubsection{SELECT count(*) FROM <огромная таблица>}

Функция \lstinline!count()! работает очень просто: сначала выбираются все записи, удовлетворяющие условию, а потом к полученному набору записей применяется агрегатная функция~--- считается количество выбранных строк. Информация о видимости записи для текущей транзакции (а конкурентным транзакциям может быть видимо разное количество записей в таблице!) не хранится в индексе, поэтому, даже если использовать для выполнения запроса индекс первичного ключа таблицы, всё равно потребуется чтение записей собственно из файла таблицы.

\textbf{Проблема} Запрос вида

\begin{lstlisting}[language=SQL,label=lst:sql_performance1,caption=SQL]
SELECT count(*) FROM foo;
\end{lstlisting}

осуществляет полный просмотр таблицы foo, что весьма долго для таблиц с большим количеством записей.

\textbf{Решение} Простого решения проблемы, к сожалению, нет. Возможны следующие подходы:

\begin{enumerate}
  \item Если точное число записей не важно, а важен порядок\footnote{<<на нашем форуме более 10000 зарегистрированных пользователей, оставивших более 50000 сообщений!>>}, то можно использовать информацию о количестве записей в таблице, собранную при выполнении команды ANALYZE:
\begin{lstlisting}[language=SQL,label=lst:sql_performance2,caption=SQL]
SELECT reltuples FROM pg_class WHERE relname = 'foo';
\end{lstlisting}
  \item Если подобные выборки выполняются часто, а изменения в таблице достаточно редки, то можно завести вспомогательную таблицу, хранящую число записей в основной. На основную же таблицу повесить триггер, который будет уменьшать это число в случае удаления записи и увеличивать в случае вставки. Таким образом, для получения количества записей потребуется лишь выбрать одну запись из вспомогательной таблицы;
  \item Вариант предыдущего подхода, но данные во вспомогательной таблице обновляются через определённые промежутки времени (cron);
\end{enumerate}


\subsubsection{Медленный DISTINCT}

Текущая реализация \lstinline!DISTINCT! для больших таблиц очень медленна. Но возможно использовать \lstinline!GROUP BY! взамен \lstinline!DISTINCT!. \lstinline!GROUP BY! может использовать агрегирующий хэш, что значительно быстрее, чем \lstinline!DISTINCT! (актуально до версии 8.4 и ниже).

\begin{lstlisting}[language=SQL,label=lst:sql_performance3,caption=DISTINCT]
postgres=# select count(*) from (select distinct i from g) a;
 count
-------
 19125
(1 row)

Time: 580,553 ms


postgres=# select count(*) from (select distinct i from g) a;
 count
-------
 19125
(1 row)

Time: 36,281 ms
\end{lstlisting}

\begin{lstlisting}[language=SQL,label=lst:sql_performance4,caption=GROUP BY]
postgres=# select count(*) from (select i from g group by i) a;
 count
-------
 19125
(1 row)

Time: 26,562 ms


postgres=# select count(*) from (select i from g group by i) a;
 count
-------
 19125
(1 row)

Time: 25,270 ms

\end{lstlisting}


\subsection{Утилиты для оптимизации запросов}

\subsubsection{pgFouine}

\href{http://pgfouine.projects.pgfoundry.org/}{pgFouine}~--- это анализатор log-файлов для PostgreSQL, используемый для генерации детальных отчетов из log-файлов PostgreSQL. pgFouine поможет определить, какие запросы следует оптимизировать в первую очередь. pgFouine написан на языке программирования PHP с использованием объектно-ориентированных технологий и легко расширяется для поддержки специализированных отчетов, является свободным программным обеспечением и распространяется на условиях GNU General Public License. Утилита спроектирована таким образом, чтобы обработка очень больших log-файлов не требовала много ресурсов.

Для работы с pgFouine сначала нужно сконфигурировать PostgreSQL для создания нужного формата log-файлов:

\begin{itemize}
  \item Чтобы включить протоколирование в syslog
  \begin{lstlisting}[label=lst:sql_performance5,caption=pgFouine]
  log_destination = 'syslog'
  redirect_stderr = off
  silent_mode = on
  \end{lstlisting}
  \item Для записи запросов, длящихся дольше n миллисекунд:
  \begin{lstlisting}[label=lst:sql_performance6,caption=pgFouine]
  log_min_duration_statement = n
  log_duration = off
  log_statement = 'none'
  \end{lstlisting}
\end{itemize}

Для записи каждого обработанного запроса установите \lstinline!log_min_duration_statement! на 0. Чтобы отключить запись запросов, установите этот параметр на -1.

pgFouine~--- простой в использовании инструмент командной строки. Следующая команда создаёт HTML-отчёт со стандартными параметрами:

\begin{lstlisting}[label=lst:sql_performance7,caption=pgFouine]
pgfouine.php -file your/log/file.log > your-report.html
\end{lstlisting}

С помощью этой строки можно отобразить текстовый отчёт с 10 запросами на каждый экран на стандартном выводе:

\begin{lstlisting}[label=lst:sql_performance8,caption=pgFouine]
pgfouine.php -file your/log/file.log -top 10 -format text
\end{lstlisting}

Более подробно о возможностях, а также много полезных примеров, можно найти на официальном сайта проекта \href{http://pgfouine.projects.pgfoundry.org/}{pgfouine.projects.pgfoundry.org}.


\subsubsection{pgBadger}


\href{http://dalibo.github.io/pgbadger/}{pgBadger}~--- аналогичная утилита, что и pgFouine, но написанная на Perl. Еще одно большое преимущество проекта в том, что он более активно сейчас разрабатывается (на момент написания этого текста последний релиз pgFouine был в 24.02.2010, а последняя версия pgBadger~--- 22.02.2016). Установка pgBadger проста:

\begin{lstlisting}[language=Bash,label=lst:sql_performance9,caption=Установка pgBadger]
$ tar xzf pgbadger-2.x.tar.gz
$ cd pgbadger-2.x/
$ perl Makefile.PL
$ make && sudo make install
\end{lstlisting}

Как и в случае с pgFouine нужно настроить PostgreSQL логи:

\begin{lstlisting}[label=lst:sql_performance10,caption=Настройка логов PostgreSQL]
logging_collector = on
log_min_messages = debug1
log_min_error_statement = debug1
log_min_duration_statement = 0
log_line_prefix = '%t [%p]: [%l-1] user=%u,db=%d '
log_checkpoints = on
log_connections = on
log_disconnections = on
log_lock_waits = on
log_temp_files = 0
\end{lstlisting}

Парсим логи PostgreSQL через pgBadger:

\begin{lstlisting}[language=Bash,label=lst:sql_performance11,caption=Запуск pgBadger]
$ ./pgbadger ~/pgsql/master/pg_log/postgresql-2012-08-30_132*
[========================>] Parsed 10485768 bytes of 10485768 (100.00%)
[========================>] Parsed 10485828 bytes of 10485828 (100.00%)
[========================>] Parsed 10485851 bytes of 10485851 (100.00%)
[========================>] Parsed 10485848 bytes of 10485848 (100.00%)
[========================>] Parsed 10485839 bytes of 10485839 (100.00%)
[========================>] Parsed 982536 bytes of 982536 (100.00%)
\end{lstlisting}

В результате получится HTML файлы, которые содержат статистику по запросам к PostgreSQL. Более подробно о возможностях можно найти на официальном сайта проекта \href{http://dalibo.github.io/pgbadger/}{http://dalibo.github.io/pgbadger/}.

\subsubsection{pg\_stat\_statements}

Pg\_stat\_statements~--- расширение для сбора статистики выполнения запросов в рамках всего сервера. Преимущество данного расширения в том, что ему не требуется собирать и парсить логи PostgreSQL, как это делает pgFouine и pgBadger. Для начала установим и настроим его:

\begin{lstlisting}[label=lst:sql_performance12,caption=Настройка pg\_stat\_statements в postgresql.conf]
shared_preload_libraries = 'pg_stat_statements'
custom_variable_classes = 'pg_stat_statements' # данная настройка нужна для PostgreSQL 9.1 и ниже

pg_stat_statements.max = 10000
pg_stat_statements.track = all
\end{lstlisting}

После внесения этих параметров PostgreSQL потребуется перегрузить. Параметры конфигурации pg\_stat\_statements:

\begin{enumerate}
  \item \lstinline!pg_stat_statements.max (integer)!>>~--- максимальное количество sql запросов, которые будет хранится расширением (удаляются записи с наименьшим количеством вызовов);
  \item \lstinline!pg_stat_statements.track (enum)!>>~--- какие SQL запросы требуется записывать. Возможные параметры: top (только запросы от приложения/клиента), all (все запросы, например в функциях) и none (отключить сбор статистики);
  \item \lstinline!pg_stat_statements.save (boolean)!>>~--- следует ли сохранять собранную статистику после остановки PostgreSQL. По умолчанию включено;
\end{enumerate}

Далее активируем расширение:

\begin{lstlisting}[language=SQL,label=lst:sql_performance13,caption=Активация pg\_stat\_statements]
# CREATE EXTENSION pg_stat_statements;
\end{lstlisting}

Пример собранной статистики:

\begin{lstlisting}[language=SQL,label=lst:sql_performance14,caption=pg\_stat\_statements статистика]
# SELECT query, calls, total_time, rows, 100.0 * shared_blks_hit /
               nullif(shared_blks_hit + shared_blks_read, 0) AS hit_percent
          FROM pg_stat_statements ORDER BY total_time DESC LIMIT 10;
-[ RECORD 1 ]----------------------------------------------------------------------------
query       | SELECT query, calls, total_time, rows, ? * shared_blks_hit /
            |                nullif(shared_blks_hit + shared_blks_read, ?) AS hit_percent
            |           FROM pg_stat_statements ORDER BY total_time DESC LIMIT ?;
calls       | 3
total_time  | 0.994
rows        | 7
hit_percent | 100.0000000000000000
-[ RECORD 2 ]----------------------------------------------------------------------------
query       | insert into x (i) select generate_series(?,?);
calls       | 2
total_time  | 0.591
rows        | 110
hit_percent | 100.0000000000000000
-[ RECORD 3 ]----------------------------------------------------------------------------
query       | select * from x where i = ?;
calls       | 2
total_time  | 0.157
rows        | 6
hit_percent | 100.0000000000000000
-[ RECORD 4 ]----------------------------------------------------------------------------
query       | SELECT pg_stat_statements_reset();
calls       | 1
total_time  | 0.102
rows        | 1
hit_percent |
\end{lstlisting}

Для сброса статистики есть команда \lstinline!pg_stat_statements_reset!:

\begin{lstlisting}[language=SQL,label=lst:sql_performance15,caption=Сброс статистика]
# SELECT pg_stat_statements_reset();
-[ RECORD 1 ]------------+-
pg_stat_statements_reset |

# SELECT query, calls, total_time, rows, 100.0 * shared_blks_hit /
               nullif(shared_blks_hit + shared_blks_read, 0) AS hit_percent
          FROM pg_stat_statements ORDER BY total_time DESC LIMIT 10;
-[ RECORD 1 ]-----------------------------------
query       | SELECT pg_stat_statements_reset();
calls       | 1
total_time  | 0.175
rows        | 1
hit_percent |
\end{lstlisting}

Хочется сразу отметить, что расширение только с версии PostgreSQL 9.2 contrib нормализирует SQL запросы. В версиях 9.1 и ниже SQL запросы сохраняются как есть, а значит <<select * from table where id = 3>> и <<select * from table where id = 21>> буду разными записями, что почти бесполезно для сбора полезной статистики.