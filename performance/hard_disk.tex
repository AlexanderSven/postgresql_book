\section{Диски и файловые системы}

Очевидно, что от качественной дисковой подсистемы в сервере БД зависит немалая часть производительности. Вопросы выбора и тонкой настройки <<железа>>, впрочем, не являются темой данной главы, ограничимся уровнем файловой системы.

Единого мнения насчёт наиболее подходящей для PostgreSQL файловой системы нет, поэтому рекомендуется использовать ту, которая лучше всего поддерживается вашей операционной системой. При этом учтите, что современные журналирующие файловые системы не намного медленнее нежурналирующих, а выигрыш~--- быстрое восстановление после сбоев~--- от их использования велик.

Вы легко можете получить выигрыш в производительности без побочных эффектов, если примонтируете файловую систему, содержащую базу данных, с параметром noatime\footnote{при этом не будет отслеживаться время последнего доступа к файлу}.

\subsection{Перенос журнала транзакций на отдельный диск}
При доступе к диску изрядное время занимает не только собственно чтение данных, но и перемещение магнитной головки.

Если в вашем сервере есть несколько физических дисков (несколько логических разделов на одном диске здесь, очевидно, не помогут: головка всё равно будет одна), то вы можете разнести файлы базы данных и журнал транзакций по разным дискам. Данные в сегменты журнала пишутся последовательно, более того, записи в журнале транзакций сразу сбрасываются на диск, поэтому в случае нахождения его на отдельном диске магнитная головка не будет лишний раз двигаться, что позволит ускорить запись.

Порядок действий:

\begin{itemize}
  \item Остановите сервер (!);
  \item Перенесите каталоги \lstinline!pg_clog! и \lstinline!pg_xlog!, находящийся в каталоге с базами данных, на другой диск;
  \item Создайте на старом месте символическую ссылку;
  \item Запустите сервер.
\end{itemize}

Примерно таким же образом можно перенести и часть файлов, содержащих таблицы и индексы, на другой диск, но здесь потребуется больше кропотливой ручной работы, а при внесении изменений в схему базы процедуру, возможно, придётся повторить.


\subsection{CLUSTER}
\label{sec:hard-drive-cluster}

\lstinline!CLUSTER table [ USING index ]!~--- команда для упорядочивания записей таблицы на диске согласно индексу, что иногда за счет уменьшения доступа к диску ускоряет выполнение запроса. Возможно создать только один физический порядок в таблице, поэтому и таблица может иметь только один кластерный индекс. При таком условии нужно тщательно выбирать, какой индекс будет использоваться для кластерного индекса.

Кластеризация по индексу позволяет сократить время поиска по диску: во время поиска по индексу выборка данных может быть значительно быстрее, так как последовательность данных в таком же порядке, как и индекс. Из минусов можно отметить то, что команда \lstinline!CLUSTER! требует <<ACCESS EXCLUSIVE>> блокировку, что предотвращает любые другие операции с данными (чтения и записи) пока кластеризация не завершит выполнение. Также кластеризация индекса в PostgreSQL не утверждает четкий порядок следования, поэтому требуется повторно выполнять \lstinline!CLUSTER! для поддержания таблицы в порядке.