\section{Slony-I}
\subsection{Введение}
Slony это система репликации реального времени, позволяющая организовать синхронизацию нескольких серверов 
PostgreSQL по сети. Slony использует триггеры Postgre для привязки к событиям INSERT/ DELETE/UPDATE и 
хранимые процедуры для выполнения действий.

Система Slony с точки зрения администратора состоит из двух главных компонент, репликационного демона slony и 
административной консоли slonik. Администрирование системы сводится к общению со slonik-ом, демон slon только 
следит за собственно процессом репликации. А админ следит за тем, чтобы slon висел там, где ему положено. 

\subsubsection{О slonik-e}
Все команды slonik принимает на свой stdin. До начала выполнения скрипт slonik-a проверяется на соответствие синтаксису, 
если обнаруживаются ошибки, скрипт не выполняется, так что можно не волноваться если slonik сообщает о syntax error, 
ничего страшного не произошло. И он ещё ничего не сделал. Скорее всего. 

\subsection{Установка}
Установка на Ubuntu производится простой командой:
\begin{verbatim}
sudo aptitude install slony1-bin
\end{verbatim}

\subsection{Настройка}
\label{sec:slonyI}
Рассмотрим теперь установку на гипотетическую базу данных customers 
(названия узлов, кластеров и таблиц являются вымышленными).

Наши данные
\begin{itemize}
\item БД: customers
\item master\_host: customers\_master.com
\item slave\_host\_1: customers\_slave.com
\item cluster name (нужно придумать): customers\_rep
\end{itemize}

\subsubsection{Подготовка master-сервера}
Для начала нам нужно создать пользователя Postgres, под которым будет действовать Slony. 
По умолчанию, и отдавая должное системе, этого пользователя обычно называют slony.
\begin{verbatim}
pgsql@customers_master$ createuser -a -d slony
pgsql@customers_master$ psql -d template1 -c "alter \
user slony with password 'slony_user_password';"
\end{verbatim}

Также на каждом из узлов лучше завести системного пользователя slony, чтобы запускать от его имени 
репликационного демона slon. В дальнейшем подразумевается, что он (и пользователь и slon) есть на 
каждом из узлов кластера. 

\subsubsection{Подготовка одного slave-сервера}
Здесь я рассматриваю, что серверы кластера соединены посредством сети Internet (как в моём случае), необходимо 
чтобы с каждого из ведомых серверов можно было установить соединение с PostgreSQL на мастер-хосте, и наоборот. 
То есть, команда:
\begin{verbatim}
anyuser@customers_slave$ psql -d customers \
-h customers_master.com -U slony
\end{verbatim}

должна подключать нас к мастер-серверу (после ввода пароля, желательно). Если что-то не так, возможно требуется 
поковыряться в настройках firewall-a, или файле pg\_hba.conf, который лежит в \$PGDATA.

Теперь устанавливаем на slave-хост сервер PostgreSQL. Следующего обычно не требуется, сразу после установки Postgres 
<<up and ready>>, но в случае каких-то ошибок можно начать <<с чистого листа>>, выполнив следующие команды 
(предварительно сохранив конфигурационные файлы и остановив postmaster):
\begin{verbatim}
pgsql@customers_slave$ rm -rf $PGDATA
pgsql@customers_slave$ mkdir $PGDATA
pgsql@customers_slave$ initdb -E UTF8 -D $PGDATA
pgsql@customers_slave$ createuser -a -d slony
pgsql@customers_slave$ psql -d template1 -c "alter \
user slony with password 'slony_user_password';"
\end{verbatim}

Запускаем postmaster.

Внимание! Обычно требуется определённый владелец для реплицируемой БД. В этом случае необходимо завести его тоже!
\begin{verbatim}
pgsql@customers_slave$ createuser -a -d customers_owner
pgsql@customers_slave$ psql -d template1 -c "alter \
user customers_owner with password 'customers_owner_password';"
\end{verbatim}

Эти две команды можно запускать с customers\_master, к командной строке в этом случае нужно добавить 
<<-h customers\_slave>>, чтобы все операции выполнялись на slave.

На slave, как и на master, также нужно установить Slony.

\subsubsection{Инициализация БД и plpgsql на slave}

Следующие команды выполняются от пользователя slony. Скорее всего для выполнения каждой из них потребуется 
ввести пароль (slony\_user\_password). Итак:
\begin{verbatim}
slony@customers_master$ createdb -O customers_owner \
-h customers_slave.com customers
slony@customers_master$ createlang -d customers \
-h customers_slave.com plpgsql
\end{verbatim}

Внимание! Все таблицы, которые будут добавлены в replication set должны иметь primary key. 
Если какая-то из таблиц не удовлетворяет этому условию, задержитесь на этом шаге и дайте каждой таблице primary key 
командой ALTER TABLE ADD PRIMARY KEY.

Если столбца который мог бы стать primary key не находится, добавьте новый столбец типа serial (ALTER TABLE ADD COLUMN), 
и заполните его значениями. Настоятельно НЕ рекомендую использовать <<table add key>> slonik-a.

Продолжаем.
Создаём таблицы и всё остальное на slave:
\begin{verbatim}
slony@customers_master$ pg_dump -s customers | \
psql -U slony -h customers_slave.com customers
\end{verbatim}

pg\_dump -s сдампит только структуру нашей БД.

pg\_dump -s customers должен пускать без пароля, а вот для psql -U slony -h customers\_slave.com 
customers придётся набрать пароль (slony\_user\_pass). Важно: я подразумеваю что сейчас на мастер-хосте 
ещё не установлен Slony (речь не про make install), то есть в БД нет таблиц sl\_*, триггеров и прочего. 
Если есть, то возможно два варианта:
\begin{itemize}
\item добавляется узел в уже функционирующую систему репликации (читайте раздел 5)
\item это ошибка :-) Тогда до переноса структуры на slave выполните следующее:
\begin{verbatim}
slonik <<EOF
cluster name = customers_slave;
node Y admin conninfo = 'dbname=customers host=customers_master.com 
port=5432 user=slony password=slony_user_pass';
uninstall node (id = Y);
echo 'okay';
EOF
\end{verbatim}
Y~--- число. Любое. Важно: если это действительно ошибка, cluster name может иметь какой-то другое значение, например T1 
(default). Нужно его выяснить и сделать uninstall.

Если структура уже перенесена (и это действительно ошибка), сделайте uninstall с обоих узлов (с master и slave). 
\end{itemize}

\subsubsection{Инициализация кластера}
Если Сейчас мы имеем два сервера PgSQL которые свободно <<видят>> друг друга по сети, 
на одном из них находится мастер-база с данными, на другом~--- только структура.

На мастер-хосте запускаем такой скрипт:
\begin{verbatim}
#!/bin/sh

CLUSTER=customers_rep

DBNAME1=customers
DBNAME2=customers

HOST1=customers_master.com
HOST2=customers_slave.com

PORT1=5432
PORT2=5432

SLONY_USER=slony

slonik <<EOF
cluster name = $CLUSTER;
node 1 admin conninfo = 'dbname=$DBNAME1 host=$HOST1 port=$PORT1 
user=slony password=slony_user_password';
node 2 admin conninfo = 'dbname=$DBNAME2 host=$HOST2 
port=$PORT2 user=slony password=slony_user_password';
init cluster ( id = 1, comment = 'Customers DB 
replication cluster' );

echo 'Create set';

create set ( id = 1, origin = 1, comment = 'Customers 
DB replication set' );

echo 'Adding tables to the subscription set';

echo ' Adding table public.customers_sales...';
set add table ( set id = 1, origin = 1, id = 4, full qualified 
name = 'public.customers_sales', comment = 'Table public.customers_sales' );
echo ' done';

echo ' Adding table public.customers_something...';
set add table ( set id = 1, origin = 1, id = 5, full qualified 
name = 'public.customers_something, 
comment = 'Table public.customers_something );
echo ' done';

echo 'done adding';
store node ( id = 2, comment = 'Node 2, $HOST2' );
echo 'stored node';
store path ( server = 1, client = 2, conninfo = 'dbname=$DBNAME1 host=$HOST1 
port=$PORT1 user=slony password=slony_user_password' );
echo 'stored path';
store path ( server = 2, client = 1, conninfo = 'dbname=$DBNAME2 host=$HOST2 
port=$PORT2 user=slony password=slony_user_password' );

store listen ( origin = 1, provider = 1, receiver = 2 );
store listen ( origin = 2, provider = 2, receiver = 1 );
EOF
\end{verbatim}

Здесь мы инициализируем кластер, создаём репликационный набор, включаем в него две таблицы. 
Важно: нужно перечислить все таблицы, которые нужно реплицировать, id таблицы в наборе должен быть уникальным, 
таблицы должны иметь primary key.

Важно: replication set запоминается раз и навсегда. Чтобы добавить узел в схему репликации не нужно заново инициализировать set.

Важно: если в набор добавляется или удаляется таблица нужно переподписать все узлы. 
То есть сделать unsubscribe и subscribe заново. 

\subsubsection{Подписываем slave-узел на replication set}
Скрипт:
\begin{verbatim}
#!/bin/sh

CLUSTER=customers_rep

DBNAME1=customers
DBNAME2=customers

HOST1=customers_master.com
HOST2=customers_slave.com

PORT1=5432
PORT2=5432

SLONY_USER=slony

slonik <<EOF
cluster name = $CLUSTER;
node 1 admin conninfo = 'dbname=$DBNAME1 host=$HOST1 
port=$PORT1 user=slony password=slony_user_password';
node 2 admin conninfo = 'dbname=$DBNAME2 host=$HOST2 
port=$PORT2 user=slony password=slony_user_password';

echo'subscribing';
subscribe set ( id = 1, provider = 1, receiver = 2, forward = no);

EOF
\end{verbatim}

\subsubsection{Старт репликации}
Теперь, на обоих узлах необходимо запустить демона репликации.
\begin{verbatim}
slony@customers_master$ slon customers_rep \
"dbname=customers user=slony"
\end{verbatim}

и
\begin{verbatim}
slony@customers_slave$ slon customers_rep \
"dbname=customers user=slony"
\end{verbatim}

Сейчас слоны обменяются сообщениями и начнут передачу данных. Начальное наполнение происходит с помощью COPY, 
slave DB на это время полностью блокируется.

В среднем время актуализации данных на slave-системе составляет до 10-ти секунд. 
slon успешно обходит проблемы со связью и подключением к БД, и вообще требует к 
себе достаточно мало внимания. 

\subsection{Общие задачи}
\subsubsection{Добавление ещё одного узла в работающую схему репликации}
Выполнить~\ref{sec:slonyI}.1 и выполнить~\ref{sec:slonyI}.2.

Новый узел имеет id = 3. Находится на хосте customers\_slave3.com, <<видит>> мастер-сервер по сети и 
мастер может подключиться к его PgSQL. 

после дублирования структуры (п~\ref{sec:slonyI}.2) делаем следующее:
\begin{verbatim}
slonik <<EOF
cluster name = customers_slave;
node 3 admin conninfo = 'dbname=customers host=customers_slave3.com 
port=5432 user=slony password=slony_user_pass';
uninstall node (id = 3);
echo 'okay';
EOF
\end{verbatim}

Это нужно чтобы удалить схему, триггеры и процедуры, которые были сдублированы вместе с таблицами и структурой БД.

Инициализировать кластер не надо. Вместо этого записываем информацию о новом узле в сети:
\begin{verbatim}
#!/bin/sh

CLUSTER=customers_rep

DBNAME1=customers
DBNAME3=customers

HOST1=customers_master.com
HOST3=customers_slave3.com

PORT1=5432
PORT2=5432

SLONY_USER=slony

slonik <<EOF
cluster name = $CLUSTER;
node 1 admin conninfo = 'dbname=$DBNAME1 host=$HOST1 
port=$PORT1 user=slony password=slony_user_pass';
node 3 admin conninfo = 'dbname=$DBNAME3 
host=$HOST3 port=$PORT2 user=slony password=slony_user_pass';

echo 'done adding';

store node ( id = 3, comment = 'Node 3, $HOST3' );
echo 'sored node';
store path ( server = 1, client = 3, conninfo = 'dbname=$DBNAME1 
host=$HOST1 port=$PORT1 user=slony password=slony_user_pass' );
echo 'stored path';
store path ( server = 3, client = 1, conninfo = 'dbname=$DBNAME3 
host=$HOST3 port=$PORT2 user=slony password=slony_user_pass' );

echo 'again';
store listen ( origin = 1, provider = 1, receiver = 3 );
store listen ( origin = 3, provider = 3, receiver = 1 );

EOF
\end{verbatim}

Новый узел имеет id 3, потому что 2 уже есть и работает. Подписываем новый узел 3 на replication set:
\begin{verbatim}
#!/bin/sh

CLUSTER=customers_rep

DBNAME1=customers
DBNAME3=customers

HOST1=customers_master.com
HOST3=customers_slave3.com

PORT1=5432
PORT2=5432

SLONY_USER=slony

slonik <<EOF
cluster name = $CLUSTER;
node 1 admin conninfo = 'dbname=$DBNAME1 host=$HOST1 
port=$PORT1 user=slony password=slony_user_pass';
node 3 admin conninfo = 'dbname=$DBNAME3 host=$HOST3 
port=$PORT2 user=slony password=slony_user_pass';

echo'subscribing';
subscribe set ( id = 1, provider = 1, receiver = 3, forward = no);

EOF
\end{verbatim}

Теперь запускаем slon на новом узле, так же как и на остальных. Перезапускать slon на мастере не надо.
\begin{verbatim}
slony@customers_slave3$ slon customers_rep \
"dbname=customers user=slony"
\end{verbatim}

Репликация должна начаться как обычно. 

\subsection{Устранение неисправностей}
\subsubsection{Ошибка при добавлении узла в систему репликации}
Периодически, при добавлении новой машины в кластер возникает следующая ошибка: на новой ноде всё начинает 
жужжать и работать, имеющиеся же отваливаются с примерно следующей диагностикой:
\begin{verbatim}
%slon customers_rep "dbname=customers user=slony_user"
CONFIG main: slon version 1.0.5 starting up
CONFIG main: local node id = 3
CONFIG main: loading current cluster configuration
CONFIG storeNode: no_id=1 no_comment='CustomersDB 
replication cluster'
CONFIG storeNode: no_id=2 no_comment='Node 2, 
node2.example.com'
CONFIG storeNode: no_id=4 no_comment='Node 4, 
node4.example.com'
CONFIG storePath: pa_server=1 pa_client=3 
pa_conninfo="dbname=customers 
host=mainhost.com port=5432 user=slony_user 
password=slony_user_pass" pa_connretry=10
CONFIG storeListen: li_origin=1 li_receiver=3 
li_provider=1
CONFIG storeSet: set_id=1 set_origin=1 
set_comment='CustomersDB replication set'
WARN remoteWorker_wakeup: node 1 - no worker thread
CONFIG storeSubscribe: sub_set=1 sub_provider=1 sub_forward='f'
WARN remoteWorker_wakeup: node 1 - no worker thread
CONFIG enableSubscription: sub_set=1
WARN remoteWorker_wakeup: node 1 - no worker thread
CONFIG main: configuration complete - starting threads
CONFIG enableNode: no_id=1
CONFIG enableNode: no_id=2
CONFIG enableNode: no_id=4
ERROR remoteWorkerThread_1: "begin transaction; set 
transaction isolation level 
serializable; lock table "_customers_rep".sl_config_lock; select 
"_customers_rep".enableSubscription(1, 1, 4); 
notify "_customers_rep_Event"; notify "_customers_rep_Confirm"; 
insert into "_customers_rep".sl_event (ev_origin, ev_seqno, 
ev_timestamp, ev_minxid, ev_maxxid, ev_xip, 
ev_type , ev_data1, ev_data2, ev_data3, ev_data4 ) values 
('1', '219440', 
'2005-05-05 18:52:42.708351', '52501283', '52501292', 
'''52501283''', 'ENABLE_SUBSCRIPTION', 
'1', '1', '4', 'f'); insert into "_customers_rep".
sl_confirm (con_origin, con_received, 
con_seqno, con_timestamp) values (1, 3, '219440', 
CURRENT_TIMESTAMP); commit transaction;" 
PGRES_FATAL_ERROR ERROR: insert or update on table 
"sl_subscribe" violates foreign key 
constraint "sl_subscribe-sl_path-ref"
DETAIL: Key (sub_provider,sub_receiver)=(1,4) 
is not present in table "sl_path".
INFO remoteListenThread_1: disconnecting from 
'dbname=customers host=mainhost.com 
port=5432 user=slony_user password=slony_user_pass'
%
\end{verbatim}


Это означает что в служебной таблице \_<имя кластера>.sl\_path;, например 
\_customers\_rep.sl\_path на уже имеющихся узлах отсутствует информация о новом узле. В данном случае, 
id нового узла 4, пара (1,4) в sl\_path отсутствует.

Видимо, это баг Slony. Как избежать этого и последующих ручных вмешательств пока не ясно.

Чтобы это устранить, нужно выполнить на каждом из имеющихся узлов приблизительно следующий запрос 
(добавить путь, в данном случае (1,4)):
\begin{verbatim}
slony_user@masterhost$ psql -d customers -h _every_one_of_slaves -U slony
customers=# insert into _customers_rep.sl_path 
values ('1','4','dbname=customers host=mainhost.com 
port=5432 user=slony_user password=slony_user_password,'10');
\end{verbatim}

Если возникают затруднения, да и вообще для расширения кругозора можно посмотреть на служебные таблицы 
и их содержимое. Они не видны обычно и находятся в рамках пространства имён \_<имя кластера>, 
например \_customers\_rep. 

\subsubsection{Что делать если репликация со временем начинает тормозить}
В процессе эксплуатации наблюдаю как со временем растёт нагрузка на master-сервере, в списке активных бекендов~--- 
постоянные SELECT-ы со слейвов. В pg\_stat\_activity видим примерно такие запросы:
\begin{verbatim}
select ev_origin, ev_seqno, ev_timestamp, ev_minxid, ev_maxxid, ev_xip, 
ev_type, ev_data1, ev_data2, ev_data3, ev_data4, ev_data5, ev_data6, 
ev_data7, ev_data8 from "_customers_rep".sl_event e where 
(e.ev_origin = '2' and e.ev_seqno > '336996') or 
(e.ev_origin = '3' and e.ev_seqno > '1712871') or 
(e.ev_origin = '4' and e.ev_seqno > '721285') or 
(e.ev_origin = '5' and e.ev_seqno > '807715') or 
(e.ev_origin = '1' and e.ev_seqno > '3544763') or 
(e.ev_origin = '6' and e.ev_seqno > '2529445') or 
(e.ev_origin = '7' and e.ev_seqno > '2512532') or 
(e.ev_origin = '8' and e.ev_seqno > '2500418') or 
(e.ev_origin = '10' and e.ev_seqno > '1692318') 
order by e.ev_origin, e.ev_seqno;
\end{verbatim}

Не забываем что \_customers\_rep~--- имя схемы из примера, у вас будет другое имя.

Таблица sl\_event почему-то разрастается со временем, замедляя выполнение этих 
запросов до неприемлемого времени. Удаляем ненужные записи:
\begin{verbatim}
delete from _customers_rep.sl_event where 
ev_timestamp<NOW()-'1 DAY'::interval;
\end{verbatim}

Производительность должна вернуться к изначальным значениям. 
Возможно имеет смысл почистить таблицы \_customers\_rep.sl\_log\_* где 
вместо звёздочки подставляются натуральные числа, по-видимому по 
количеству репликационных сетов, так что \_customers\_rep.sl\_log\_1 
точно должна существовать. 