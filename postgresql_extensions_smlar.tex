\section{Smlar}
\textbf{Лицензия}: Open Source

\textbf{Ссылка}: \href{http://sigaev.ru/git/gitweb.cgi?p=smlar.git;a=blob;hb=HEAD;f=README}{sigaev.ru}

Поиск похожестей в больших базах данных является важным вопросом в настоящее время для таких систем как блоги (похожие статьи), интернет-магазины (похожие продукты), хостинг изображений (похожие изображения, поиск дубликатов изображений) и т.д. PostgreSQL позволяет сделать такой поиск более легким. Прежде всего, необходимо понять, как мы будем вычислять сходство двух объектов.

\subsection{Похожесть}

Любой объект может быть описан как список характеристик. Например, статья в блоге может быть описана тегами, продукт в интернет-магазине может быть описан размером, весом, цветом и т.д. Это означает, что для каждого объекта можно создать цифровую подпись~--- массив чисел, описывающих объект (отпечатки пальцев\footnote{http://en.wikipedia.org/wiki/Fingerprint}, n-grams\footnote{http://en.wikipedia.org/wiki/N-gram}). Тоесть нужно создать массив из цифр для описания каждого объекта. Что делать дальше?