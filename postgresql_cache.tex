\chapter{Кэширование в PostgreSQL}
\begin{epigraphs}
\qitem{Чтобы что-то узнать, нужно уже что-то знать.}{Станислав Лем}
\end{epigraphs}
\section{Введение}
Кэш или кеш~--- промежуточный буфер с быстрым доступом, содержащий информацию, которая может быть запрошена с наибольшей вероятностью.
Кэширование SELECT запросов позволяет повысить производительность приложений и снизить нагрузку на PostgreSQL. 
Преимущества кэширования особенно заметны в случае с относительно маленькими таблицами, имеющими статические данные, 
например, справочными таблицами. 

Многие СУБД могут кэшировать SQL запросы, и данная возможность идет у них, в основном, <<из коробки>>. 
PostgreSQL не обладает подобным функционалом. Почему? 
Во-первых, мы теряем транзакционную чистоту происходящего в базе. Что это значит?
Управление конкурентным доступом с помощью многоверсионности (MVCC~--- MultiVersion Concurrency Control)~--- 
один из механизмов обеспечения одновременного конкурентного доступа к БД, заключающийся в предоставлении каждому 
пользователю <<снимка>> БД, обладающего тем свойством, что вносимые данным пользователем изменения в БД невидимы 
другим пользователям до момента фиксации транзакции. Этот способ управления позволяет добиться того, что пишущие 
транзакции не блокируют читающих, и читающие транзакции не блокируют пишущих. 
При использовании кэширования, которому нет дела к транзакциям СУБД, <<снимки>> БД могут быть с неверними данними.
Во-вторых, кеширование результатов запросов, в основном, должно происходить на стороне приложения, а не СУБД. 
В таком случае управление кэшированием может работать более гибко (включать и отключать его где потребуется для приложения), а 
СУБД будет заниматся своей непосредственной целью~--- хранение и предоставление целосности данных.