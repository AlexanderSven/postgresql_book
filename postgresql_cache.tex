\chapter{Кеширование в PostgreSQL}
\begin{epigraphs}
\qitem{Чтобы что-то узнать, нужно уже что-то знать.}{Станислав Лем}
\end{epigraphs}
\section{Введение}
Кэш или кеш~--- промежуточный буфер с быстрым доступом, содержащий информацию, которая может быть запрошена с наибольшей вероятностью.
Кэширование SELECT запросов позволяет повысить производительность приложений и снизить нагрузку на PostgreSQL. 
Преимущества кэширования особенно заметны в случае с относительно маленькими таблицами, имеющими статические данные, 
например, справочными таблицами. 