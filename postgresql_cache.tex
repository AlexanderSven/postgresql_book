\chapter{Кэширование в PostgreSQL}
\begin{epigraphs}
\qitem{Чтобы что-то узнать, нужно уже что-то знать.}{Станислав Лем}
\end{epigraphs}
\section{Введение}
Кэш или кеш~--- промежуточный буфер с быстрым доступом, содержащий информацию, которая может быть запрошена с наибольшей вероятностью.
Кэширование SELECT запросов позволяет повысить производительность приложений и снизить нагрузку на PostgreSQL. 
Преимущества кэширования особенно заметны в случае с относительно маленькими таблицами, имеющими статические данные, 
например, справочными таблицами. 

Многие СУБД могут кэшировать SQL запросы, и данная возможность идет у них, в основном, <<из коробки>>. 
PostgreSQL не обладает подобным функционалом. Почему? 
Во-первых, мы теряем транзакционную чистоту происходящего в базе. Что это значит?
Управление конкурентным доступом с помощью многоверсионности (MVCC~--- MultiVersion Concurrency Control)~--- 
один из механизмов обеспечения одновременного конкурентного доступа к БД, заключающийся в предоставлении каждому 
пользователю <<снимка>> БД, обладающего тем свойством, что вносимые данным пользователем изменения в БД невидимы 
другим пользователям до момента фиксации транзакции. Этот способ управления позволяет добиться того, что пишущие 
транзакции не блокируют читающих, и читающие транзакции не блокируют пишущих. 
При использовании кэширования, которому нет дела к транзакциям СУБД, <<снимки>> БД могут быть с неверними данними.
Во-вторых, кеширование результатов запросов, в основном, должно происходить на стороне приложения, а не СУБД. 
В таком случае управление кэшированием может работать более гибко (включать и отключать его где потребуется для приложения), а 
СУБД будет заниматся своей непосредственной целью~--- хранение и предоставление целосности данных.

Но, несмотря на все эти минуси, многим разработчикам требуется кэширование на уровне базы данных. 
Для организации кэширования существует два инструмента для PostgreSQL:
\begin{itemize}
\item Pgmemcache (с memcached)
\item Pgpool-II
\end{itemize}


\section{Pgmemcache}
Memcached\footnote{http://memcached.org/}~--- 
компьютерная программа, реализующая сервис кэширования данных в оперативной памяти на основе 
парадигмы распределенной хеш-таблицы. С помощью клиентской библиотеки позволяет кэшировать данные в оперативной 
памяти одного или нескольких из множества доступных серверов. Распределение реализуется путем сегментирования 
данных по значению хэша ключа по аналогии с сокетами хэш-таблицы. Клиентская библиотека используя ключ данных 
вычисляет хэш и использует его для выбора соответствующего сервера. Ситуация сбоя сервера трактуется как промах 
кэша, что позволяет повышать отказоустойчивость комплекса за счет наращивания количества memcached серверов и возможности 
производить их горячую замену.

Pgmemcache\footnote{http://pgfoundry.org/projects/pgmemcache/}~--- это 
PostgreSQL API библиотека на основе libmemcached для взаимодействия с memcached. С помощью данной библиотеки 
PostgreSQL может записывать, считывать, искать и удалять данные из memcached. Попробуем, что из себя представляет данный тип кэширования.


\subsection{Установка}
Во время написания этой главы была доступна 2.0.4 версия 
pgmemcache\footnote{http://pgfoundry.org/frs/download.php/2672/pgmemcache\_2.0.4.tar.bz2}. 
Pgmemcache будет устанавливатся и настраиватся на PostgreSQL версии 8.4, операционная система~--- Ubuntu Server 10.10.

