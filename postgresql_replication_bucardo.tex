\section{Bucardo}
\subsection{Введение}
Bucardo~--- асинхронная master-master или master-slave репликация PostgreSQL, которая написана на Perl. 
Система очень гибкая, поддерживает несколько видов синхронизации и обработки конфликтов.

\subsection{Установка}
Установку будем проводить на Ubuntu Server. Сначала нам нужно установить DBIx::Safe Perl модуль.
\begin{lstlisting}[label=lst:bucardo1,caption=Установка]
sudo aptitude install libdbix-safe-perl
\end{lstlisting}

Для других систем можно поставить из исходников\footnote{http://search.cpan.org/CPAN/authors/id/T/TU/TURNSTEP/}:
\begin{lstlisting}[label=lst:bucardo2,caption=Установка]
tar xvfz DBIx-Safe-1.2.5.tar.gz
cd DBIx-Safe-1.2.5
perl Makefile.PL
make && make test && sudo make install
\end{lstlisting}

Теперь ставим сам Bucardo. Скачиваем\footnote{http://bucardo.org/wiki/Bucardo\#Obtaining\_Bucardo} его и инсталируем:
\begin{lstlisting}[label=lst:bucardo3,caption=Установка]
tar xvfz Bucardo-4.4.0.tar.gz
cd Bucardo-4.4.0
perl Makefile.PL
make
sudo make install
\end{lstlisting}

Для работы Bucardo потребуется установить поддержку pl/perlu языка PostgreSQL.
\begin{lstlisting}[label=lst:bucardo4,caption=Установка]
sudo aptitude install postgresql-plperl-8.4
\end{lstlisting}

Можем приступать к настройке.

\subsection{Настройка}
\subsubsection{Инициализация Bucardo}
Запускаем установку командой:
\begin{lstlisting}[label=lst:bucardo5,caption=Инициализация Bucardo]
bucardo_ctl install
\end{lstlisting}

Bucardo покажет настройки подключения к PostgreSQL, которые можно будет изменить:
\begin{lstlisting}[label=lst:bucardo6,caption=Инициализация Bucardo]
This will install the bucardo database into an existing Postgres cluster.
Postgres must have been compiled with Perl support,
and you must connect as a superuser

We will create a new superuser named 'bucardo',
and make it the owner of a new database named 'bucardo'

Current connection settings:
1. Host:          <none>
2. Port:          5432
3. User:          postgres
4. Database:      postgres
5. PID directory: /var/run/bucardo
\end{lstlisting}

Когда вы измените требуемые настройки и подтвердите установку, Bucardo создаст пользователя bucardo и базу данных bucardo.
Данный пользователь должен иметь право логинится через Unix socket, поэтому лучше заранее дать ему такие права в pg\_hda.conf.

\subsubsection{Настройка баз данных}
Теперь нам нужно настроить базы данных, с которыми будет работать Bucardo. 
Пусть у нас будет master\_db и slave\_db. Сначала настроим мастер:
\begin{lstlisting}[label=lst:bucardo7,caption=Настройка баз данных]
bucardo_ctl add db master_db name=master
bucardo_ctl add all tables herd=all_tables
bucardo_ctl add all sequences herd=all_tables
\end{lstlisting}

Первой командой мы указали базу данных и дали ей имя master (для того, что в реальной жизни master\_db и slave\_db 
имеют одинаковое название и их нужно Bucardo отличать). Второй и третей командой мы указали реплицыровать все 
таблицы и последовательности, обьеденив их в групу all\_tables.

Дальше добавляем slave\_db:
\begin{lstlisting}[label=lst:bucardo8,caption=Настройка баз данных]
bucardo_ctl add db slave_db name=replica port=6543 host=slave_host
\end{lstlisting}

Мы назвали replica базу данных в Bucardo.

\subsubsection{Настройка синхронизации}
Теперь нам нужно настроить синхронизацию между этими базами данных. Делается это командой (master-slave):
\begin{lstlisting}[label=lst:bucardo9,caption=Настройка синхронизации]
bucardo_ctl add sync delta type=pushdelta source=all_tables targetdb=replica
\end{lstlisting}

Данной командой мы установим Bucardo тригеры в PostgreSQL. А теперь по параметрам:
\begin{itemize}
\item \textbf{type}

Это тип синхронизации. Существует 3 типа:
\begin{itemize}
\item \textbf{Fullcopy}. Полное копирование.
\item \textbf{Pushdelta}. Master-slave репликация.
\item \textbf{Swap}. Master-master репликация. 
Для работы в таком режиме потребуется указать как Bucardo должен решать конфликты синхронизации.
Для этого в таблице <<goat>> (в которой находятся таблицы и последовательности) нужно в <<standard\_conflict>> 
поле поставить значение (это значение может быть разным для разных таблиц и последовательностей):
\begin{itemize}
\item source~--- при конфликте мы копируем данные с source (master\_db в нашем случае).
\item target~--- при конфликте мы копируем данные с target (slave\_db в нашем случае).
\item skip~--- конфликт мы просто не реплицируем. Не рекомендуется.
\item random~--- каждая БД имеет одинаковый шанс, что её изменение будет взято для решение конфликта.
\item latest~--- запись, которая была последней изменена решает конфликт.
\item abort~--- синхронизация прерывается.
\end{itemize}
\end{itemize}

\item \textbf{source}

Источник синхронизации.

\item \textbf{targetdb}

БД, в которум производим репликацию.
\end{itemize}

Для master-master:
\begin{lstlisting}[label=lst:bucardo10,caption=Настройка синхронизации]
bucardo_ctl add sync delta type=swap source=all_tables targetdb=replica
\end{lstlisting}

\subsubsection{Запуск/Остановка репликации}
Запуск репликации:
\begin{lstlisting}[label=lst:bucardo11,caption=Запуск репликации]
bucardo_ctl start
\end{lstlisting}

Остановка репликации:
\begin{lstlisting}[label=lst:bucardo12,caption=Остановка репликации]
bucardo_ctl stop
\end{lstlisting}

\subsection{Общие задачи}

\subsubsection{Просмотр значений конфигурации}
Просто используя эту команду:
\begin{lstlisting}[label=lst:bucardo13,caption=Просмотр значений конфигурации]
bucardo_ctl show all
\end{lstlisting}

\subsubsection{Изменения значений конфигурации}
\begin{lstlisting}[label=lst:bucardo14,caption=Изменения значений конфигурациии]
bucardo_ctl set name=value
\end{lstlisting}

Например:
\begin{lstlisting}[label=lst:bucardo15,caption=Изменения значений конфигурации]
bucardo_ctl set syslog_facility=LOG_LOCAL3
\end{lstlisting}

\subsubsection{Перегрузка конфигурации}
\begin{lstlisting}[label=lst:bucardo16,caption=Перегрузка конфигурации]
bucardo_ctl reload_config
\end{lstlisting}

Более полный список команд~--- http://bucardo.org/wiki/Bucardo\_ctl

