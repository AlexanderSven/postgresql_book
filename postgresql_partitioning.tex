\chapter{Партиционирование}
\begin{epigraphs}
\qitem{Решая какую-либо проблему, всегда полезно заранее знать правильный ответ. 
При условии, конечно, что вы уверены в наличии самой проблемы.}{Народная мудрость}
\end{epigraphs}
\section{Введение}
Партиционирование (partitioning, секционирование)~--- это разбиение больших структур баз данных (таблицы, индексы) на меньшие кусочки. 
Звучит сложно, но на практике все просто.

Скорее всего у Вас есть несколько огромных таблиц (обычно всю нагрузку обеспечивают всего несколько таблиц СУБД из всех имеющихся). 
Причем чтение в большинстве случаев приходится только на самую последнюю их часть (т.е. активно читаются те данные, которые 
недавно появились). Примером тому может служить блог~--- на первую страницу (это последние 5\dots10 постов) приходится 40\dots50\% 
всей нагрузки, или новостной портал (суть одна и та же), или системы личных сообщений… впрочем понятно. Партиционирование 
таблицы позволяет базе данных делать интеллектуальную выборку~--- сначала СУБД уточнит, какой партиции соответствует Ваш запрос 
(если это реально) и только потом сделает этот запрос, применительно к нужной партиции (или нескольким партициям). Таким образом, 
в рассмотренном случае, Вы распределите нагрузку на таблицу по ее партициям. Следовательно выборка типа 
<<SELECT * FROM articles ORDER BY id DESC LIMIT 10>> будет выполняться только над последней партицией, которая значительно 
меньше всей таблицы.

Итак, партиционирование дает ряд преимуществ:
\begin{itemize}
\item На определенные виды запросов (которые, в свою очередь, создают основную нагрузку на СУБД) мы можем улучшить производительность.
\item Массовое удаление может быть произведено путем удаления одной или нескольких партиций 
(DROP TABLE гораздо быстрее, чем массовый DELETE).
\item Редко используемые данные могут быть перенесены в другое хранилище.
\end{itemize}

\section{Теория}
На текущий момент PostgreSQL поддерживает два критерия для создания партиций:
\begin{itemize}
\item Партиционирование по диапазону значений (range)~--- таблица разбивается на <<диапазоны>> значений по полю или набору полей 
в таблице, без перекрытия диапазонов значений, отнесенных к различным партициям. Например, диапазоны дат.
\item Партиционирование по списку значений (list)~--- таблица разбивается по спискам ключевые значения для каждой партиции.
\end{itemize}

Чтобы настроить партиционирование таблици, достаточно выполните следующие действия:
\begin{itemize}
\item Создается <<мастер>> таблица, из которой все партиции будут наследоваться. Эта таблица не будет содержать данные. 
Так же не нужно ставить никаких ограничений на таблицу, если конечно они не будут дублироватся на партиции.
\item Создайте несколько <<дочерних>> таблиц, которые наследуют от <<мастер>> таблицы. 
\item Добавить в <<дочерние>> таблицы значения, по которым они будут партициями. 
Стоить заметить, что значения партиций не должны пересекатся. Например:
\begin{lstlisting}[language=SQL,label=lst:partitioning1,caption=Пример неверного задлания значений партиций]
CHECK ( outletID BETWEEN 100 AND 200 )
CHECK ( outletID BETWEEN 200 AND 300 )
\end{lstlisting}
неверно заданы партиции, поскольку не понятно какой партиции пренадлежит значение 200. 
\item Для каждой партиции создать индекс по ключевому полю (или нескольким), а также указать любые другие требуемые индексы.
\item При необходимости, создать триггер или правило для перенаправления данных с <<мастер>> таблици в соответствующую партицию.
\item Убедиться, что параметр <<constraint\_exclusion>> не отключен в postgresql.conf. Если его не включить, то запросы не будут 
оптимизированы при работе с партиционирование.
\end{itemize}

\section{Практика использования}
Теперь начнем с практического примера. Представим, что в нашей системе есть таблица, в которую мы собираем данные о 
посещаемости нашего ресурса. На любой запрос пользователя наша система логирует действия в эту таблицу. 
И, например, в начале каждого месяца (неделю) нам нужно создавать отчет за предыдущий месяц (неделю). При этом, 
логи нужно хранить в течении 3 лет.
Данные в такой таблице накапливаются быстро, если система активно используется. И вот, когда таблица уже с милионами, а то, 
и милиардами записей, создавать отчеты становится все сложнее (да и чистка старых записей становится не легким делом). 
Работа с такой таблицей создает огромную нагрузку на СУБД. 
Тут нам на помощь и приходит партиционирование.

\subsection{Настройка}
Для примера, мы имеем следующию таблицу:
\begin{lstlisting}[language=SQL,label=lst:partitioning2,caption=<<Мастер>> таблица]
CREATE TABLE my_logs (
    id              SERIAL PRIMARY KEY,
    user_id         INT NOT NULL,
    logdate         TIMESTAMP NOT NULL,
    data            TEXT,
    some_state      INT
);
\end{lstlisting}

Поскольку нам нужны отчеты каждый месяц, мы будем делить партиции по месяцам. Это поможет нам быстрее создавать 
отчеты и чистить старые данные. 

<<Мастер>> таблица будет <<my\_logs>>, структуру которой мы указали выше. Далее создадим <<дочерние>> таблици (партиции):
\begin{lstlisting}[language=SQL,label=lst:partitioning3,caption=<<Дочерние>> таблици]
CREATE TABLE my_logs2010m10 (
    CHECK ( logdate >= DATE '2010-10-01' AND logdate < DATE '2010-11-01' )
) INHERITS (my_logs);
CREATE TABLE my_logs2010m11 (
    CHECK ( logdate >= DATE '2010-11-01' AND logdate < DATE '2010-12-01' )
) INHERITS (my_logs);
CREATE TABLE my_logs2010m12 (
    CHECK ( logdate >= DATE '2010-12-01' AND logdate < DATE '2011-01-01' )
) INHERITS (my_logs);
CREATE TABLE my_logs2011m01 (
    CHECK ( logdate >= DATE '2011-01-01' AND logdate < DATE '2010-02-01' )
) INHERITS (my_logs);
\end{lstlisting}

Данными командами мы создаем таблицы <<my\_logs2010m10>>, <<my\_logs2010m11>> и т.д., которые копируют структуру с <<мастер>> 
таблици (кроме индексов). Также с помощью <<CHECK>> мы задаем диапазон значений, который будет попадать в эту партицию 
(хочу опять напомнить, что диапазоны значений партиций не должны пересекатся!). Поскольку партиционирование будет работать 
по полю <<logdate>>, мы создадим индекс на это поле на всех партициях:
\begin{lstlisting}[language=SQL,label=lst:partitioning4,caption=Создание индексов]
CREATE INDEX my_logs2010m10_logdate ON my_logs2010m10 (logdate);
CREATE INDEX my_logs2010m11_logdate ON my_logs2010m11 (logdate);
CREATE INDEX my_logs2010m12_logdate ON my_logs2010m12 (logdate);
CREATE INDEX my_logs2011m01_logdate ON my_logs2011m01 (logdate);
\end{lstlisting}

Далее для удобства создадим функцию, которая будет перенаправлять новые данные с <<мастер>> таблици в соответствующую партицию.
\begin{lstlisting}[language=SQL,label=lst:partitioning5,caption=Функция для перенаправления]
CREATE OR REPLACE FUNCTION my_logs_insert_trigger()
RETURNS TRIGGER AS $$
BEGIN
    IF ( NEW.logdate >= DATE '2010-10-01' AND
         NEW.logdate < DATE '2010-11-01' ) THEN
        INSERT INTO my_logs2010m10 VALUES (NEW.*);
    ELSIF ( NEW.logdate >= DATE '2010-11-01' AND
            NEW.logdate < DATE '2010-12-01' ) THEN
        INSERT INTO my_logs2010m11 VALUES (NEW.*);
    ELSIF ( NEW.logdate >= DATE '2010-12-01' AND
            NEW.logdate < DATE '2011-01-01' ) THEN
        INSERT INTO my_logs2010m12 VALUES (NEW.*);
    ELSIF ( NEW.logdate >= DATE '2011-01-01' AND
            NEW.logdate < DATE '2011-02-01' ) THEN
        INSERT INTO my_logs2011m01 VALUES (NEW.*);
    ELSE
        RAISE EXCEPTION 'Date out of range.  Fix the my_logs_insert_trigger() function!';
    END IF;
    RETURN NULL;
END;
$$
LANGUAGE plpgsql;
\end{lstlisting}

В функции ничего особенного нет: идет проверка поля <<logdate>>, по которой направляются данные в нужную партицию. 
При не нахождении требуемой партиции~--- вызываем ошибку. Теперь осталось создать триггер на <<мастер>> таблицу 
для автоматического вызова данной функции:
\begin{lstlisting}[language=SQL,label=lst:partitioning6,caption=Триггер]
CREATE TRIGGER insert_my_logs_trigger
    BEFORE INSERT ON my_logs
    FOR EACH ROW EXECUTE PROCEDURE my_logs_insert_trigger();
\end{lstlisting}

Партиционирование настроено и теперь мы готовы приступить к тестированию.

\subsection{Тестирование}
Для начала добавим данные в нашу таблицу <<my\_logs>>:
\begin{lstlisting}[language=SQL,label=lst:partitioning7,caption=Данные]
INSERT INTO my_logs (user_id,logdate, data, some_state) VALUES(1, '2010-10-30', '30.10.2010 data', 1);
INSERT INTO my_logs (user_id,logdate, data, some_state) VALUES(2, '2010-11-10', '10.11.2010 data2', 1);
INSERT INTO my_logs (user_id,logdate, data, some_state) VALUES(1, '2010-12-15', '15.12.2010 data3', 1);
\end{lstlisting}

Теперь проверим где они хранятся:
\begin{lstlisting}[language=SQL,label=lst:partitioning8,caption=<<Мастер>> таблица чиста]
partitioning_test=# SELECT * FROM ONLY my_logs;
 id | user_id | logdate | data | some_state 
----+---------+---------+------+------------
(0 rows)
\end{lstlisting}
Как видим в <<мастер>> таблицу данные не попали~--- она чиста. Теперь проверим а есть ли вообще данные:
\begin{lstlisting}[language=SQL,label=lst:partitioning9,caption=Проверка данных]
partitioning_test=# SELECT * FROM my_logs;
 id | user_id |       logdate       |       data       | some_state 
----+---------+---------------------+------------------+------------
  1 |       1 | 2010-10-30 00:00:00 | 30.10.2010 data  |          1
  2 |       2 | 2010-11-10 00:00:00 | 10.11.2010 data2 |          1
  3 |       1 | 2010-12-15 00:00:00 | 15.12.2010 data3 |          1
(3 rows)
\end{lstlisting}

Данные при этом выводятся без проблем. Проверим партиции, правильно ли хранятся данные:
\begin{lstlisting}[language=SQL,label=lst:partitioning10,caption=Проверка хранения данных]
partitioning_test=# Select * from my_logs2010m10;
 id | user_id |       logdate       |      data       | some_state 
----+---------+---------------------+-----------------+------------
  1 |       1 | 2010-10-30 00:00:00 | 30.10.2010 data |          1
(1 row)

partitioning_test=# Select * from my_logs2010m11;
 id | user_id |       logdate       |       data       | some_state 
----+---------+---------------------+------------------+------------
  2 |       2 | 2010-11-10 00:00:00 | 10.11.2010 data2 |          1
(1 row)
\end{lstlisting}

Отлично! Данные хранятся на требуемых нам партициях. При этом запросы к таблице <<my\_logs>> менять не нужно:
\begin{lstlisting}[language=SQL,label=lst:partitioning11,caption=Проверка запросов]
partitioning_test=# SELECT * FROM my_logs WHERE user_id = 2;
 id | user_id |       logdate       |       data       | some_state 
----+---------+---------------------+------------------+------------
  2 |       2 | 2010-11-10 00:00:00 | 10.11.2010 data2 |          1
(1 row)

partitioning_test=# SELECT * FROM my_logs WHERE data LIKE '%0.1%';
 id | user_id |       logdate       |       data       | some_state 
----+---------+---------------------+------------------+------------
  1 |       1 | 2010-10-30 00:00:00 | 30.10.2010 data  |          1
  2 |       2 | 2010-11-10 00:00:00 | 10.11.2010 data2 |          1
(2 rows)
\end{lstlisting}

\subsection{Управление партициями}
Обычно при работе с партиционированием старые партиции перестают получать данные и остаются неизменными. 
Это дает огоромное приемущество над работой с данными через партиции. 
Например, нам нужно удалить старые логи за 2008 год, 10 месяц. Нам достаточно выполить:
\begin{lstlisting}[language=SQL,label=lst:partitioning12,caption=Чистка логов]
DROP TABLE my_logs2008m10;
\end{lstlisting}
поскольку <<DROP TABLE>> работает гораздо быстрее, чем удаление милионов записей индивидуально через <<DELETE>>.
Другой вариант, который более предпочтителен, просто удалить партицию из партиционирования, 
тем самым оставив данные в СУБД, но уже не доступные через <<мастер>> таблицу: 
\begin{lstlisting}[language=SQL,label=lst:partitioning13,caption=Удаляем партицию из партиционирования]
ALTER TABLE my_logs2008m10 NO INHERIT my_logs; 
\end{lstlisting}
Это удобно, если мы хотим эти данные потом перенести в другое хранилище или просто сохранить.

\subsection{Важность <<constraint\_exclusion>> для партиционирования}
Параметр <<constraint\_exclusion>> отвечает за оптимизацию запросов, что повышает производительность для 
партиционированых таблиц. Например, выпоним простой запрос:
\begin{lstlisting}[language=SQL,label=lst:partitioning14,caption=<<constraint\_exclusion>> OFF]
partitioning_test=# SET constraint_exclusion = off;
partitioning_test=# EXPLAIN SELECT * FROM my_logs WHERE logdate > '2010-12-01';

                                            QUERY PLAN                                             
---------------------------------------------------------------------------------------------------
 Result  (cost=6.81..104.66 rows=1650 width=52)
   ->  Append  (cost=6.81..104.66 rows=1650 width=52)
         ->  Bitmap Heap Scan on my_logs  (cost=6.81..20.93 rows=330 width=52)
               Recheck Cond: (logdate > '2010-12-01 00:00:00'::timestamp without time zone)
               ->  Bitmap Index Scan on my_logs_logdate  (cost=0.00..6.73 rows=330 width=0)
                     Index Cond: (logdate > '2010-12-01 00:00:00'::timestamp without time zone)
         ->  Bitmap Heap Scan on my_logs2010m10 my_logs  (cost=6.81..20.93 rows=330 width=52)
               Recheck Cond: (logdate > '2010-12-01 00:00:00'::timestamp without time zone)
               ->  Bitmap Index Scan on my_logs2010m10_logdate  (cost=0.00..6.73 rows=330 width=0)
                     Index Cond: (logdate > '2010-12-01 00:00:00'::timestamp without time zone)
         ->  Bitmap Heap Scan on my_logs2010m11 my_logs  (cost=6.81..20.93 rows=330 width=52)
               Recheck Cond: (logdate > '2010-12-01 00:00:00'::timestamp without time zone)
               ->  Bitmap Index Scan on my_logs2010m11_logdate  (cost=0.00..6.73 rows=330 width=0)
                     Index Cond: (logdate > '2010-12-01 00:00:00'::timestamp without time zone)
         ->  Bitmap Heap Scan on my_logs2010m12 my_logs  (cost=6.81..20.93 rows=330 width=52)
               Recheck Cond: (logdate > '2010-12-01 00:00:00'::timestamp without time zone)
               ->  Bitmap Index Scan on my_logs2010m12_logdate  (cost=0.00..6.73 rows=330 width=0)
                     Index Cond: (logdate > '2010-12-01 00:00:00'::timestamp without time zone)
         ->  Bitmap Heap Scan on my_logs2011m01 my_logs  (cost=6.81..20.93 rows=330 width=52)
               Recheck Cond: (logdate > '2010-12-01 00:00:00'::timestamp without time zone)
               ->  Bitmap Index Scan on my_logs2011m01_logdate  (cost=0.00..6.73 rows=330 width=0)
                     Index Cond: (logdate > '2010-12-01 00:00:00'::timestamp without time zone)
(22 rows)
\end{lstlisting}

Как видно через команду <<EXPLAIN>>, данный запрос сканирует все партиции на наличие данных в них, что не логично, 
поскольку данное условие <<logdate > 2010-12-01>> говорит о том, что данные должны братся только с партицый, 
где подходит такое условие. А теперь включим <<constraint\_exclusion>>:
\begin{lstlisting}[language=SQL,label=lst:partitioning15,caption=<<constraint\_exclusion>> ON]
partitioning_test=# SET constraint_exclusion = on;
SET
partitioning_test=# EXPLAIN SELECT * FROM my_logs WHERE logdate > '2010-12-01';
                                            QUERY PLAN                                             
---------------------------------------------------------------------------------------------------
 Result  (cost=6.81..41.87 rows=660 width=52)
   ->  Append  (cost=6.81..41.87 rows=660 width=52)
         ->  Bitmap Heap Scan on my_logs  (cost=6.81..20.93 rows=330 width=52)
               Recheck Cond: (logdate > '2010-12-01 00:00:00'::timestamp without time zone)
               ->  Bitmap Index Scan on my_logs_logdate  (cost=0.00..6.73 rows=330 width=0)
                     Index Cond: (logdate > '2010-12-01 00:00:00'::timestamp without time zone)
         ->  Bitmap Heap Scan on my_logs2010m12 my_logs  (cost=6.81..20.93 rows=330 width=52)
               Recheck Cond: (logdate > '2010-12-01 00:00:00'::timestamp without time zone)
               ->  Bitmap Index Scan on my_logs2010m12_logdate  (cost=0.00..6.73 rows=330 width=0)
                     Index Cond: (logdate > '2010-12-01 00:00:00'::timestamp without time zone)
(10 rows)
\end{lstlisting}

Как мы видим, теперь запрос работает правильно, и сканирует только партиции, что подходят под условие запроса.
Но включать <<constraint\_exclusion>> не желательно для баз, где нет партиционирования, 
поскольку команда <<CHECK>> будет проверятся на всех запросах, даже простых, а значит производительность сильно упадет.
Начиная с 8.4 версии PostgreSQL <<constraint\_exclusion>> может быть <<on>>, <<off>> и <<partition>>. По умолчанию 
(и рекомендуется) ставить 
<<constraint\_exclusion>> не <<on>>, и не <<off>>, а <<partition>>, который будет проверять <<CHECK>> только на 
партиционированых таблицах. 

\section{Заключение}
Партиционирование~--- одна из самых простых и менее безболезненных методов уменьшения нагрузки на СУБД. 
Именно на этот вариант стоит посмотреть сперва, и если он не подходит по каким либо причинам~--- переходить к более сложным.
Но если в системе есть таблица, у которой актуальны только новые данные, но огромное количество старых (не актуальных)
данных дает 50\% или более нагрузки на СУБД~--- Вам стоит внедрить партиционирование.