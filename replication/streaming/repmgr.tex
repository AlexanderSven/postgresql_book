\subsection{Repmgr}

\href{http://www.repmgr.org/}{Repmgr}~--- набор инструментов для управления потоковой репликацией и восстановления после сбоя кластера PostgreSQL серверов. Он автоматизирует настройку резервных серверов, мониторинг репликации, а также помогает выполнять задачи администрированию кластера, такие как отказоустойчивость (failover) или переключение мастера-слейва (слейв становится мастером, а мастер - слейвом). Repmgr работает с версии PostgreSQL 9.3 и выше.

Repmgr состоит из двух утилит:

\begin{itemize}
  \item \lstinline!repmgr!~--- инструмент командной строки (cli), который используется для административных задач, таких как:
  \begin{itemize}
    \item создание слейвов;
    \item переключение слейва в режим мастера;
    \item переключение между собой мастер и слейв серверов;
    \item отображение состояния кластера;
  \end{itemize}
  \item \lstinline!repmgrd!~--- демон, который мониторит кластер серверов и выполняет такие задачи:
  \begin{itemize}
    \item мониторинг и логирование эффективности репликации;
    \item автоматическое переключение слейва в мастер при обнаружении проблем у текущего мастера (failover);
    \item посылка сообщений о событиях в кластере через заданые пользователем скрипты;
  \end{itemize}
\end{itemize}

\subsubsection{Пример использования: автоматическое переключение слейва в мастер}

Для использования failover потребуется добавить \lstinline!repmgr_funcs! в \lstinline!postgresql.conf!:

\begin{lstlisting}[language=Bash,label=lst:repmgr1,caption=repmgr\_funcs]
shared_preload_libraries = 'repmgr_funcs'
\end{lstlisting}

И добавить настройки в \lstinline!repmgr.conf!:

\begin{lstlisting}[language=Bash,label=lst:repmgr2,caption=repmgr.conf]
failover=automatic
promote_command='repmgr standby promote -f /etc/repmgr.conf --log-to-file'
follow_command='repmgr standby follow -f /etc/repmgr.conf --log-to-file'
\end{lstlisting}

Для демонстрации автоматического failover, настроен кластер с тремя узлами репликации (один мастер и два слейв сервера), так что таблица \lstinline!repl_nodes! выглядит следующим образом:

\begin{lstlisting}[language=SQL,label=lst:repmgr3,caption=repl\_nodes]
# SELECT id, type, upstream_node_id, priority, active FROM repmgr_test.repl_nodes ORDER BY id;
 id |  type   | upstream_node_id | priority | active
----+---------+------------------+----------+--------
  1 | master  |                  |      100 | t
  2 | standby |                1 |      100 | t
  3 | standby |                1 |      100 | t
(3 rows)
\end{lstlisting}

После запуска \lstinline!repmgrd! демона на каждом сервере в режиме ожидания, убеждаемся что он мониторит кластер:

\begin{lstlisting}[language=Bash,label=lst:repmgr4,caption=logs]
[2016-01-05 13:15:40] [INFO] checking cluster configuration with schema 'repmgr_test'
[2016-01-05 13:15:40] [INFO] checking node 2 in cluster 'test'
[2016-01-05 13:15:40] [INFO] reloading configuration file and updating repmgr tables
[2016-01-05 13:15:40] [INFO] starting continuous standby node monitoring
\end{lstlisting}

Теперь остановим мастер базу:

\begin{lstlisting}[language=Bash,label=lst:repmgr5,caption=Остановка текущего мастера]
pg_ctl -D /path/to/node1/data -m immediate stop
\end{lstlisting}

\lstinline!repmgrd! автоматически замечает падение мастера и переключает один из слейвов в мастер:

\begin{lstlisting}[language=Bash,label=lst:repmgr6,caption=Переключение слейва в мастер]
[2016-01-06 18:32:58] [WARNING] connection to upstream has been lost, trying to recover... 15 seconds before failover decision
[2016-01-06 18:33:03] [WARNING] connection to upstream has been lost, trying to recover... 10 seconds before failover decision
[2016-01-06 18:33:08] [WARNING] connection to upstream has been lost, trying to recover... 5 seconds before failover decision
...
[2016-01-06 18:33:18] [NOTICE] this node is the best candidate to be the new master, promoting...
...
[2016-01-06 18:33:20] [NOTICE] STANDBY PROMOTE successful
\end{lstlisting}

Также переключает оставшийся слейв на новый мастер:

\begin{lstlisting}[language=Bash,label=lst:repmgr7,caption=Переключение слейва на новый мастер]
[2016-01-06 18:32:58] [WARNING] connection to upstream has been lost, trying to recover... 15 seconds before failover decision
[2016-01-06 18:33:03] [WARNING] connection to upstream has been lost, trying to recover... 10 seconds before failover decision
[2016-01-06 18:33:08] [WARNING] connection to upstream has been lost, trying to recover... 5 seconds before failover decision
...
[2016-01-06 18:33:23] [NOTICE] node 2 is the best candidate for new master, attempting to follow...
[2016-01-06 18:33:23] [INFO] changing standby's master
...
[2016-01-06 18:33:25] [NOTICE] node 3 now following new upstream node 2
\end{lstlisting}

Таблица \lstinline!repl_nodes! будет обновлена, чтобы отразить новую ситуацию~--- старый мастер \lstinline!node1! помечен как неактивный, и слейв \lstinline!node3! теперь работает от нового мастера \lstinline!node2!:

\begin{lstlisting}[language=SQL,label=lst:repmgr8,caption=Результат после failover]
# SELECT id, type, upstream_node_id, priority, active from repl_nodes ORDER BY id;
 id |  type   | upstream_node_id | priority | active
----+---------+------------------+----------+--------
  1 | master  |                  |      100 | f
  2 | master  |                  |      100 | t
  3 | standby |                2 |      100 | t
(3 rows)
\end{lstlisting}

В таблицу \lstinline!repl_events! будут добавлены записи того, что произошло с каждым сервером во время failover:

\begin{lstlisting}[language=SQL,label=lst:repmgr9,caption=Результат после failover]
# SELECT node_id, event, successful, details from repmgr_test.repl_events where event_timestamp>='2016-01-06 18:30';
 node_id |          event           | successful |                         details
---------+--------------------------+------------+----------------------------------------------------------
       2 | standby_promote          | t          | node 2 was successfully promoted to master
       2 | repmgrd_failover_promote | t          | node 2 promoted to master; old master 1 marked as failed
       3 | repmgrd_failover_follow  | t          | node 3 now following new upstream node 2
(3 rows)
\end{lstlisting}

\subsubsection{Заключение}

Более подробно по функционалу, его настройках и ограничениях доступно в \href{https://github.com/2ndQuadrant/repmgr/blob/master/README.md}{официальном репозитории}.
