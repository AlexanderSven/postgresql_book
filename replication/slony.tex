\section{Slony-I}
\label{sec:slonyI}

\href{http://slony.info/}{Slony} это система репликации реального времени, позволяющая организовать синхронизацию нескольких серверов PostgreSQL по сети. Slony использует триггеры PostgreSQL для привязки к событиям INSERT/DELETE/UPDATE и хранимые процедуры для выполнения действий.

Система Slony с точки зрения администратора состоит из двух главных компонент: репликационного демона \lstinline!slony! и административной консоли \lstinline!slonik!. Администрирование системы сводится к общению со \lstinline!slonik!-ом, демон \lstinline!slon! только следит за собственно процессом репликации.

Все команды slonik принимает на свой stdin. До начала выполнения скрипт slonik-a проверяется на соответствие синтаксису, если обнаруживаются ошибки, скрипт не выполняется, так что можно не волноваться если slonik сообщает о syntax error, ничего страшного не произошло. И он ещё ничего не сделал. Скорее всего.

\subsection{Установка}

Установка на Ubuntu производится простой командой:

\begin{lstlisting}[label=lst:slony1,caption=Установка]
$ sudo aptitude install slony1-2-bin
\end{lstlisting}

\subsection{Настройка}
\label{subsec:slonyI-settings}

Рассмотрим установку на гипотетическую базу данных customers. Исходные данные:

\begin{itemize}
  \item \lstinline!customers!~--- база данных;
  \item \lstinline!master_host!~--- хост master базы;
  \item \lstinline!slave_host!~--- хост slave базы;
  \item \lstinline!customers_rep!~--- имя кластера;
\end{itemize}

\subsubsection{Подготовка master базы}
\label{subsec:slonyI-settings-1}

Для начала нужно создать пользователя в базе, под которым будет действовать Slony. По умолчанию, и отдавая должное системе, этого пользователя обычно называют slony.

\begin{lstlisting}[label=lst:slony2,caption=Подготовка master-сервера]
$ createuser -a -d slony
$ psql -d template1 -c "ALTER USER slony WITH PASSWORD 'slony_user_password';"
\end{lstlisting}

Также на каждом из узлов лучше завести системного пользователя slony, чтобы запускать от его имени репликационного демона slon. В дальнейшем подразумевается, что он (и пользователь и slon) есть на каждом из узлов кластера.

\subsubsection{Подготовка slave базы}
\label{subsec:slonyI-settings-2}

Здесь рассматривается, что серверы кластера соединены посредством сети. Необходимо чтобы с каждого из серверов можно было установить соединение с PostgreSQL на master хосте, и наоборот. То есть, команда:

\begin{lstlisting}[label=lst:slony3,caption=Подготовка одного slave-сервера]
anyuser@customers_slave$ psql -d customers \
-h customers_master.com -U slony
\end{lstlisting}

должна подключать нас к мастер-серверу (после ввода пароля, желательно).

Теперь устанавливаем на slave-хост сервер PostgreSQL. Следующего обычно не требуется, сразу после установки Postgres <<up and ready>>, но в случае каких-то ошибок можно начать <<с чистого листа>>, выполнив следующие команды (предварительно сохранив конфигурационные файлы и остановив postmaster):

\begin{lstlisting}[label=lst:slony4,caption=Подготовка одного slave-сервера]
pgsql@customers_slave$ rm -rf $PGDATA
pgsql@customers_slave$ mkdir $PGDATA
pgsql@customers_slave$ initdb -E UTF8 -D $PGDATA
pgsql@customers_slave$ createuser -a -d slony
pgsql@customers_slave$ psql -d template1 -c "alter \
user slony with password 'slony_user_password';"
\end{lstlisting}

Далее запускаем postmaster. Обычно требуется определённый владелец для реплицируемой БД. В этом случае необходимо создать его тоже:

\begin{lstlisting}[label=lst:slony5,caption=Подготовка одного slave-сервера]
pgsql@customers_slave$ createuser -a -d customers_owner
pgsql@customers_slave$ psql -d template1 -c "alter \
user customers_owner with password 'customers_owner_password';"
\end{lstlisting}

Эти две команды можно запускать с \lstinline!customers_master!, к командной строке в этом случае нужно добавить \lstinline!-h customers_slave!, чтобы все операции выполнялись на slave.

На slave, как и на master, также нужно установить Slony.

\subsubsection{Инициализация БД и plpgsql на slave}

Следующие команды выполняются от пользователя slony. Скорее всего для выполнения каждой из них потребуется ввести пароль (\lstinline!slony_user_password!):

\begin{lstlisting}[label=lst:slony6,caption=Инициализация БД и plpgsql на slave]
slony@customers_master$ createdb -O customers_owner \
-h customers_slave.com customers
slony@customers_master$ createlang -d customers \
-h customers_slave.com plpgsql
\end{lstlisting}

Внимание! Все таблицы, которые будут добавлены в replication set должны иметь primary key. Если какая-то из таблиц не удовлетворяет этому условию, задержитесь на этом шаге и дайте каждой таблице primary key командой \lstinline!ALTER TABLE ADD PRIMARY KEY!. Если столбца который мог бы стать primary key не находится, добавьте новый столбец типа serial (\lstinline!ALTER TABLE ADD COLUMN!), и заполните его значениями. Настоятельно НЕ рекомендую использовать \lstinline!table add key! slonik-a.

Далее создаём таблицы и всё остальное на slave базе:

\begin{lstlisting}[label=lst:slony7,caption=Инициализация БД и plpgsql на slave]
slony@customers_master$ pg_dump -s customers | \
psql -U slony -h customers_slave.com customers
\end{lstlisting}

\lstinline!pg_dump -s! сдампит только структуру нашей БД.

\lstinline!pg_dump -s customers! должен пускать без пароля, а вот для \lstinline!psql -U slony -h customers_slave.com customers! придётся набрать пароль (\lstinline!slony_user_pass!). Важно: подразумевается что сейчас на мастер-хосте ещё не установлен Slony (речь не про \lstinline!make install!), то есть в БД нет таблиц \lstinline!sl_*!, триггеров и прочего.

\subsubsection{Инициализация кластера}

Сейчас мы имеем два сервера PostgreSQL которые свободно <<видят>> друг друга по сети, на одном из них находится мастер-база с данными, на другом~--- только структура базы. Далее мастер-хосте запускаем скрипт:

\begin{lstlisting}[label=lst:slony9,caption=Инициализация кластера]
#!/bin/sh

CLUSTER=customers_rep

DBNAME1=customers
DBNAME2=customers

HOST1=customers_master.com
HOST2=customers_slave.com

PORT1=5432
PORT2=5432

SLONY_USER=slony

slonik <<EOF
cluster name = $CLUSTER;
node 1 admin conninfo = 'dbname=$DBNAME1 host=$HOST1 port=$PORT1
user=slony password=slony_user_password';
node 2 admin conninfo = 'dbname=$DBNAME2 host=$HOST2
port=$PORT2 user=slony password=slony_user_password';
init cluster ( id = 1, comment = 'Customers DB
replication cluster' );

echo 'Create set';

create set ( id = 1, origin = 1, comment = 'Customers
DB replication set' );

echo 'Adding tables to the subscription set';

echo ' Adding table public.customers_sales...';
set add table ( set id = 1, origin = 1, id = 4, full qualified
name = 'public.customers_sales', comment = 'Table public.customers_sales' );
echo ' done';

echo ' Adding table public.customers_something...';
set add table ( set id = 1, origin = 1, id = 5, full qualified
name = 'public.customers_something,
comment = 'Table public.customers_something );
echo ' done';

echo 'done adding';
store node ( id = 2, comment = 'Node 2, $HOST2' );
echo 'stored node';
store path ( server = 1, client = 2, conninfo = 'dbname=$DBNAME1 host=$HOST1
port=$PORT1 user=slony password=slony_user_password' );
echo 'stored path';
store path ( server = 2, client = 1, conninfo = 'dbname=$DBNAME2 host=$HOST2
port=$PORT2 user=slony password=slony_user_password' );

store listen ( origin = 1, provider = 1, receiver = 2 );
store listen ( origin = 2, provider = 2, receiver = 1 );
EOF
\end{lstlisting}

Здесь инициализируется кластер, создается replication set, включаются в него две таблицы. Нужно перечислить все таблицы, которые нужно реплицировать. Replication set запоминается раз и навсегда. Чтобы добавить узел в схему репликации не нужно заново инициализировать set. Если в набор добавляется или удаляется таблица нужно переподписать все узлы. То есть сделать \lstinline!unsubscribe! и \lstinline!subscribe! заново.

\subsubsection{Подписываем slave-узел на replication set}

Далее запускаем на слейве:

\begin{lstlisting}[label=lst:slony10,caption=Подписываем slave-узел на replication set]
#!/bin/sh

CLUSTER=customers_rep

DBNAME1=customers
DBNAME2=customers

HOST1=customers_master.com
HOST2=customers_slave.com

PORT1=5432
PORT2=5432

SLONY_USER=slony

slonik <<EOF
cluster name = $CLUSTER;
node 1 admin conninfo = 'dbname=$DBNAME1 host=$HOST1
port=$PORT1 user=slony password=slony_user_password';
node 2 admin conninfo = 'dbname=$DBNAME2 host=$HOST2
port=$PORT2 user=slony password=slony_user_password';

echo'subscribing';
subscribe set ( id = 1, provider = 1, receiver = 2, forward = no);

EOF
\end{lstlisting}

\subsubsection{Старт репликации}

Теперь, на обоих узлах необходимо запустить демона репликации.

\begin{lstlisting}[label=lst:slony11,caption=Старт репликации]
slony@customers_master$ slon customers_rep \
"dbname=customers user=slony"
\end{lstlisting}

и

\begin{lstlisting}[label=lst:slony12,caption=Старт репликации]
slony@customers_slave$ slon customers_rep \
"dbname=customers user=slony"
\end{lstlisting}

Cлоны обменяются сообщениями и начнут передачу данных. Начальное наполнение происходит с помощью \lstinline!COPY! команды, слейв база в это время полностью блокируется.

\subsection{Общие задачи}

\subsubsection{Добавление ещё одного узла в работающую схему репликации}

Требуется выполнить \ref{subsec:slonyI-settings-1} и \ref{subsec:slonyI-settings-2} этапы. Новый узел имеет id = 3. Находится на хосте \lstinline!customers_slave3.com!, <<видит>> мастер-сервер по сети и мастер может подключиться к его PostgreSQL. После дублирования структуры (п~\ref{subsec:slonyI-settings}.2) делается следующее:

\begin{lstlisting}[label=lst:slony13,caption=Общие задачи]
slonik <<EOF
cluster name = customers_slave;
node 3 admin conninfo = 'dbname=customers host=customers_slave3.com
port=5432 user=slony password=slony_user_pass';
uninstall node (id = 3);
echo 'okay';
EOF
\end{lstlisting}

Это нужно чтобы удалить схему, триггеры и процедуры, которые были сдублированы вместе с таблицами и структурой БД. Инициализировать кластер не надо. Вместо этого записываем информацию о новом узле в сети:

\begin{lstlisting}[label=lst:slony14,caption=Общие задачи]
#!/bin/sh

CLUSTER=customers_rep

DBNAME1=customers
DBNAME3=customers

HOST1=customers_master.com
HOST3=customers_slave3.com

PORT1=5432
PORT2=5432

SLONY_USER=slony

slonik <<EOF
cluster name = $CLUSTER;
node 1 admin conninfo = 'dbname=$DBNAME1 host=$HOST1
port=$PORT1 user=slony password=slony_user_pass';
node 3 admin conninfo = 'dbname=$DBNAME3
host=$HOST3 port=$PORT2 user=slony password=slony_user_pass';

echo 'done adding';

store node ( id = 3, comment = 'Node 3, $HOST3' );
echo 'sored node';
store path ( server = 1, client = 3, conninfo = 'dbname=$DBNAME1
host=$HOST1 port=$PORT1 user=slony password=slony_user_pass' );
echo 'stored path';
store path ( server = 3, client = 1, conninfo = 'dbname=$DBNAME3
host=$HOST3 port=$PORT2 user=slony password=slony_user_pass' );

echo 'again';
store listen ( origin = 1, provider = 1, receiver = 3 );
store listen ( origin = 3, provider = 3, receiver = 1 );

EOF
\end{lstlisting}

Новый узел имеет id 3, потому что 2 уже работает. Подписываем новый узел 3 на replication set:

\begin{lstlisting}[label=lst:slony15,caption=Общие задачи]
#!/bin/sh

CLUSTER=customers_rep

DBNAME1=customers
DBNAME3=customers

HOST1=customers_master.com
HOST3=customers_slave3.com

PORT1=5432
PORT2=5432

SLONY_USER=slony

slonik <<EOF
cluster name = $CLUSTER;
node 1 admin conninfo = 'dbname=$DBNAME1 host=$HOST1
port=$PORT1 user=slony password=slony_user_pass';
node 3 admin conninfo = 'dbname=$DBNAME3 host=$HOST3
port=$PORT2 user=slony password=slony_user_pass';

echo'subscribing';
subscribe set ( id = 1, provider = 1, receiver = 3, forward = no);

EOF
\end{lstlisting}

Теперь запускаем slon на новом узле, так же как и на остальных. Перезапускать slon на мастере не надо.

\begin{lstlisting}[label=lst:slony16,caption=Общие задачи]
slony@customers_slave3$ slon customers_rep \
"dbname=customers user=slony"
\end{lstlisting}

Репликация должна начаться как обычно.

\subsection{Устранение неисправностей}

\subsubsection{Ошибка при добавлении узла в систему репликации}

Периодически, при добавлении новой машины в кластер возникает следующая ошибка: на новой ноде всё начинает жужжать и работать, имеющиеся же отваливаются с примерно следующей диагностикой:

\begin{lstlisting}[label=lst:slony17,caption=Устранение неисправностей]
%slon customers_rep "dbname=customers user=slony_user"
CONFIG main: slon version 1.0.5 starting up
CONFIG main: local node id = 3
CONFIG main: loading current cluster configuration
CONFIG storeNode: no_id=1 no_comment='CustomersDB
replication cluster'
CONFIG storeNode: no_id=2 no_comment='Node 2,
node2.example.com'
CONFIG storeNode: no_id=4 no_comment='Node 4,
node4.example.com'
CONFIG storePath: pa_server=1 pa_client=3
pa_conninfo="dbname=customers
host=mainhost.com port=5432 user=slony_user
password=slony_user_pass" pa_connretry=10
CONFIG storeListen: li_origin=1 li_receiver=3
li_provider=1
CONFIG storeSet: set_id=1 set_origin=1
set_comment='CustomersDB replication set'
WARN remoteWorker_wakeup: node 1 - no worker thread
CONFIG storeSubscribe: sub_set=1 sub_provider=1 sub_forward='f'
WARN remoteWorker_wakeup: node 1 - no worker thread
CONFIG enableSubscription: sub_set=1
WARN remoteWorker_wakeup: node 1 - no worker thread
CONFIG main: configuration complete - starting threads
CONFIG enableNode: no_id=1
CONFIG enableNode: no_id=2
CONFIG enableNode: no_id=4
ERROR remoteWorkerThread_1: "begin transaction; set
transaction isolation level
serializable; lock table "_customers_rep".sl_config_lock; select
"_customers_rep".enableSubscription(1, 1, 4);
notify "_customers_rep_Event"; notify "_customers_rep_Confirm";
insert into "_customers_rep".sl_event (ev_origin, ev_seqno,
ev_timestamp, ev_minxid, ev_maxxid, ev_xip,
ev_type , ev_data1, ev_data2, ev_data3, ev_data4 ) values
('1', '219440',
'2005-05-05 18:52:42.708351', '52501283', '52501292',
'''52501283''', 'ENABLE_SUBSCRIPTION',
'1', '1', '4', 'f'); insert into "_customers_rep".
sl_confirm (con_origin, con_received,
con_seqno, con_timestamp) values (1, 3, '219440',
CURRENT_TIMESTAMP); commit transaction;"
PGRES_FATAL_ERROR ERROR: insert or update on table
"sl_subscribe" violates foreign key
constraint "sl_subscribe-sl_path-ref"
DETAIL: Key (sub_provider,sub_receiver)=(1,4)
is not present in table "sl_path".
INFO remoteListenThread_1: disconnecting from
'dbname=customers host=mainhost.com
port=5432 user=slony_user password=slony_user_pass'
%
\end{lstlisting}

Это означает что в служебной таблице \lstinline!_<имя кластера>.sl_path!, например \lstinline!_customers_rep.sl_path! на уже имеющихся узлах отсутствует информация о новом узле. В данном случае, id нового узла 4, пара (1,4) в \lstinline!sl_path! отсутствует. Чтобы это устранить, нужно выполнить на каждом из имеющихся узлов приблизительно следующий запрос:

\begin{lstlisting}[label=lst:slony18,caption=Устранение неисправностей]
$ psql -d customers -h _every_one_of_slaves -U slony
customers=# insert into _customers_rep.sl_path
values ('1','4','dbname=customers host=mainhost.com
port=5432 user=slony_user password=slony_user_password,'10');
\end{lstlisting}

Если возникают затруднения, то можно посмотреть на служебные таблицы и их содержимое. Они не видны обычно и находятся в рамках пространства имён \lstinline!_<имя кластера>!, например \lstinline!_customers_rep!.

\subsubsection{Что делать если репликация со временем начинает тормозить}

В процессе эксплуатации может наблюдатся как со временем растёт нагрузка на master-сервере, в списке активных бекендов~--- постоянные SELECT-ы со слейвов. В \lstinline!pg_stat_activity! видны примерно такие запросы:

\begin{lstlisting}[label=lst:slony19,caption=Устранение неисправностей]
select ev_origin, ev_seqno, ev_timestamp, ev_minxid, ev_maxxid, ev_xip,
ev_type, ev_data1, ev_data2, ev_data3, ev_data4, ev_data5, ev_data6,
ev_data7, ev_data8 from "_customers_rep".sl_event e where
(e.ev_origin = '2' and e.ev_seqno > '336996') or
(e.ev_origin = '3' and e.ev_seqno > '1712871') or
(e.ev_origin = '4' and e.ev_seqno > '721285') or
(e.ev_origin = '5' and e.ev_seqno > '807715') or
(e.ev_origin = '1' and e.ev_seqno > '3544763') or
(e.ev_origin = '6' and e.ev_seqno > '2529445') or
(e.ev_origin = '7' and e.ev_seqno > '2512532') or
(e.ev_origin = '8' and e.ev_seqno > '2500418') or
(e.ev_origin = '10' and e.ev_seqno > '1692318')
order by e.ev_origin, e.ev_seqno;
\end{lstlisting}

где \lstinline!_customers_rep!~--- имя схемы из примера. Таблица \lstinline!sl_event! почему-то разрастается со временем, замедляя выполнение этих запросов до неприемлемого времени. Удаляем ненужные записи:

\begin{lstlisting}[label=lst:slony20,caption=Устранение неисправностей]
delete from _customers_rep.sl_event where
ev_timestamp<NOW()-'1 DAY'::interval;
\end{lstlisting}

Производительность должна вернуться к изначальным значениям. Возможно имеет смысл почистить таблицы \lstinline!_customers_rep.sl_log_*! где вместо звёздочки подставляются натуральные числа, по-видимому по количеству репликационных сетов, так что \lstinline!_customers_rep.sl_log_1! точно должна существовать.

