\section{Londiste}
\label{sec:londiste}

Londiste представляет собой движок для организации репликации, написанный на языке Python. Основные принципы: надежность и простота использования. Из-за этого данное решение имеет меньше функциональности, чем Slony-I. Londiste использует в качестве транспортного механизма очередь PgQ (описание этого более чем интересного проекта остается за рамками данной главы, поскольку он представляет интерес скорее для низкоуровневых программистов баз данных, чем для конечных пользователей~--- администраторов СУБД PostgreSQL). Отличительными особенностями решения являются:

\begin{itemize}
  \item возможность потабличной репликации;
  \item начальное копирование ничего не блокирует;
  \item возможность двухстороннего сравнения таблиц;
  \item простота установки;
\end{itemize}

К недостаткам можно отнести:

\begin{itemize}
  \item триггерная репликация, что ухудшает производительность базы;
\end{itemize}


\subsection{Установка}

Установка будет проводиться на Debian сервере. Поскольку Londiste~--- это часть Skytools, то нам нужно ставить этот пакет:

\begin{lstlisting}[label=lst:londiste1,caption=Установка]
% sudo aptitude install skytools
\end{lstlisting}

В некоторых системнах пакет может содержатся версия 2.x, которая не поддерживает каскадную репликацию, отказоустойчивость(failover) и переключение между серверами (switchover). По этой причине она не будет расматриваться. Скачать самую последнюю версию пакета можно с \href{http://pgfoundry.org/projects/skytools}{официального сайта}. На момент написания главы последняя версия была 3.2. Начнем установку:

\begin{lstlisting}[label=lst:londiste2,caption=Установка]
$ wget http://pgfoundry.org/frs/download.php/3622/skytools-3.2.tar.gz
$ tar zxvf skytools-3.2.tar.gz
$ cd skytools-3.2/
# пакеты для сборки deb
$ sudo aptitude install build-essential autoconf \
automake autotools-dev dh-make \
debhelper devscripts fakeroot xutils lintian pbuilder \
python-all-dev python-support xmlto asciidoc \
libevent-dev libpq-dev libtool
# python-psycopg нужен для работы Londiste
$ sudo aptitude install python-psycopg2 postgresql-server-dev-all
# данной командой собираем deb пакет
$ make deb
$ cd ../
# ставим skytools
$ dpkg -i *.deb
\end{lstlisting}

Для других систем можно собрать Skytools командами:

\begin{lstlisting}[label=lst:londiste3,caption=Установка]
$ ./configure
$ make
$ make install
\end{lstlisting}

Далее проверяем правильность установки:

\begin{lstlisting}[label=lst:londiste4,caption=Установка]
$ londiste3 -V
londiste3, Skytools version 3.2
$ pgqd -V
bad switch: usage: pgq-ticker [switches] config.file
Switches:
  -v        Increase verbosity
  -q        No output to console
  -d        Daemonize
  -h        Show help
  -V        Show version
 --ini      Show sample config file
  -s        Stop - send SIGINT to running process
  -k        Kill - send SIGTERM to running process
  -r        Reload - send SIGHUP to running process
\end{lstlisting}


\subsection{Настройка}

Обозначения:

\begin{itemize}
  \item \lstinline!master-host!~--- мастер база данных;
  \item \lstinline!slave1-host!, \lstinline!slave2-host!, \lstinline!slave3-host!, \lstinline!slave4-host!~--- слейв базы данных;
  \item \lstinline!l3simple!~--- название реплицируемой базы данных;
\end{itemize}

\subsubsection{Конфигурация репликаторов}

Сначала создается конфигурационный файл для master базы (конфиг будет \lstinline!/etc/skytools/master-londiste.ini!):

\begin{lstlisting}[label=lst:londiste-replica1,caption=Конфигурация репликаторов]
[londiste3]
job_name = master_l3simple
db = dbname=l3simple
queue_name = replika
logfile = /var/log/skytools/master_l3simple.log
pidfile = /var/pid/skytools/master_l3simple.pid

# Задержка между проверками наличия активности
# (новых пакетов данных) в секундах
loop_delay = 0.5
\end{lstlisting}

Инициализируем Londiste для master базы:

\begin{lstlisting}[label=lst:londiste-replica2,caption=Инициализируем Londiste]
$ londiste3 /etc/skytools/master-londiste.ini create-root master-node "dbname=l3simple host=master-host"
INFO plpgsql is installed
INFO Installing pgq
INFO   Reading from /usr/share/skytools3/pgq.sql
INFO pgq.get_batch_cursor is installed
INFO Installing pgq_ext
INFO   Reading from /usr/share/skytools3/pgq_ext.sql
INFO Installing pgq_node
INFO   Reading from /usr/share/skytools3/pgq_node.sql
INFO Installing londiste
INFO   Reading from /usr/share/skytools3/londiste.sql
INFO londiste.global_add_table is installed
INFO Initializing node
INFO Location registered
INFO Node "master-node" initialized for queue "replika" with type "root"
INFO Done
\end{lstlisting}

где \lstinline!master-server!~--- это имя провайдера (мастера базы).

Теперь запустим демон:

\begin{lstlisting}[label=lst:londiste-replica3,caption=Запускаем демон для master базы]
$ londiste3 -d /etc/skytools/master-londiste.ini worker
$ tail -f /var/log/skytools/master_l3simple.log
INFO {standby: 1}
INFO {standby: 1}
\end{lstlisting}

Если нужно перегрузить демон (например при изменении конфигурации), то можно воспользоватся параметром \lstinline!-r!:

\begin{lstlisting}[label=lst:londiste-replica4,caption=Перегрузка демона]
$ londiste3 /etc/skytools/master-londiste.ini -r
\end{lstlisting}

Для остановки демона есть параметр \lstinline!-s!:

\begin{lstlisting}[label=lst:londiste-replica5,caption=Остановка демона]
$ londiste3 /etc/skytools/master-londiste.ini -s
\end{lstlisting}

или если потребуется <<убить>> (\lstinline!kill!) демон:

\begin{lstlisting}[label=lst:londiste-replica6,caption=Остановка демона]
$ londiste3 /etc/skytools/master-londiste.ini -k
\end{lstlisting}

Для автоматизации данного процесса skytools3 имеет встроенный демон, который запускает все воркеры из директории \lstinline!/etc/skytools/!. Сама конфигурация демона находится в \lstinline!/etc/skytools.ini!. Что бы запустить все демоны londiste достаточно выполнить:

\begin{lstlisting}[label=lst:londiste-replica7,caption=Демон для ticker]
$ /etc/init.d/skytools3 start
INFO Starting master_l3simple
\end{lstlisting}

Перейдем к slave базе. Для начала нужно создать базу данных:

\begin{lstlisting}[label=lst:londiste-replica8,caption=Копирования структуры базы]
$ psql -h slave1-host -U postgres
# CREATE DATABASE l3simple;
\end{lstlisting}

Подключение должно быть <<trust>> (без паролей) между master и slave базами данных.

Далее создадим конфиг для slave базы (\lstinline!/etc/skytools/slave1-londiste.ini!):

\begin{lstlisting}[label=lst:londiste-replica9,caption=Создаём конфигурацию для slave]
[londiste3]
job_name = slave1_l3simple
db = dbname=l3simple
queue_name = replika
logfile = /var/log/skytools/slave1_l3simple.log
pidfile = /var/pid/skytools/slave1_l3simple.pid

# Задержка между проверками наличия активности
# (новых пакетов данных) в секундах
loop_delay = 0.5
\end{lstlisting}

Инициализируем Londiste для slave базы:

\begin{lstlisting}[label=lst:londiste-replica10,caption=Инициализируем Londiste для slave]
$ londiste3 /etc/skytools/slave1-londiste.ini create-leaf slave1-node "dbname=l3simple host=slave1-host" --provider="dbname=l3simple host=master-host"
\end{lstlisting}

Теперь можем запустить демон:

\begin{lstlisting}[label=lst:londiste-replica11,caption=Запускаем демон для slave базы]
$ londiste3 -d /etc/skytools/slave1-londiste.ini worker
\end{lstlisting}

Или же через главный демон:

\begin{lstlisting}[label=lst:londiste-replica12,caption=Запускаем демон для slave базы]
$ /etc/init.d/skytools3 start
INFO Starting master_l3simple
INFO Starting slave1_l3simple
\end{lstlisting}


\subsubsection{Создаём конфигурацию для PgQ ticker}

Londiste требуется PgQ ticker для работы с мастер базой данных, который может быть запущен и на другой машине. Но, конечно, лучше его запускать на той же, где и master база данных. Для этого мы настраиваем специальный конфиг для ticker демона (конфиг будет \lstinline!/etc/skytools/pgqd.ini!):

\begin{lstlisting}[label=lst:londiste-replica13,caption=PgQ ticker конфиг]
[pgqd]
logfile = /var/log/skytools/pgqd.log
pidfile = /var/pid/skytools/pgqd.pid
\end{lstlisting}

Запускаем демон:

\begin{lstlisting}[label=lst:londiste-replica11,caption=Запускаем PgQ ticker]
$ pgqd -d /etc/skytools/pgqd.ini
$ tail -f /var/log/skytools/pgqd.log
LOG Starting pgqd 3.2
LOG auto-detecting dbs ...
LOG l3simple: pgq version ok: 3.2
\end{lstlisting}

Или же через глобальный демон:

\begin{lstlisting}[label=lst:londiste-replica12,caption=Запускаем PgQ ticker]
$ /etc/init.d/skytools3 restart
INFO Starting master_l3simple
INFO Starting slave1_l3simple
INFO Starting pgqd
LOG Starting pgqd 3.2
\end{lstlisting}

Теперь можно увидеть статус кластера:

\begin{lstlisting}[label=lst:londiste-replica13,caption=Статус кластера]
$ londiste3 /etc/skytools/master-londiste.ini status
Queue: replika   Local node: slave1-node

master-node (root)
  |                           Tables: 0/0/0
  |                           Lag: 44s, Tick: 5
  +--: slave1-node (leaf)
                              Tables: 0/0/0
                              Lag: 44s, Tick: 5

$ londiste3 /etc/skytools/master-londiste.ini members
Member info on master-node@replika:
node_name        dead             node_location
---------------  ---------------  --------------------------------
master-node      False            dbname=l3simple host=master-host
slave1-node      False            dbname=l3simple host=slave1-host
\end{lstlisting}

Но репликация еще не запущенна: требуется добавить таблици в очередь, которые мы хотим реплицировать. Для этого используем команду \lstinline!add-table!:

\begin{lstlisting}[label=lst:londiste-replica13,caption=Добавляем таблицы]
$ londiste3 /etc/skytools/master-londiste.ini add-table --all
$ londiste3 /etc/skytools/slave1-londiste.ini add-table --all --create-full
\end{lstlisting}

В данном примере используется параметр \lstinline!--all!, который означает все таблицы, но вместо него вы можете перечислить список конкретных таблиц, если не хотите реплицировать все. Если имена таблиц отличаются на master и slave, то можно использовать \lstinline!--dest-table! параметр при добавлении таблиц на slave базе. Также, если вы не перенесли струкруру таблиц заранее с master на slave базы, то это можно сделать автоматически через \lstinline!--create! параметр (или \lstinline!--create-full!, если нужно перенести полностью всю схему таблицы).

Подобным образом добавляем последовательности (\lstinline!sequences!) для репликации:

\begin{lstlisting}[label=lst:londiste-replica14,caption=Добавляем последовательности]
$ londiste3 /etc/skytools/master-londiste.ini add-seq --all
$ londiste3 /etc/skytools/slave1-londiste.ini add-seq --all
\end{lstlisting}

Но последовательности должны на slave базе созданы заранее (тут не поможет \lstinline!--create-full! для таблиц). Поэтому иногда проще перенести точную копию структуры master базы на slave:

\begin{lstlisting}[label=lst:londiste-replica-dump1,caption=Клонирование структуры базы]
$ pg_dump -s -npublic l3simple | psql -hslave1-host l3simple
\end{lstlisting}

Далее проверяем состояние репликации:

\begin{lstlisting}[label=lst:londiste-replica15,caption=Статус кластера]
$ londiste3 /etc/skytools/master-londiste.ini status
Queue: replika   Local node: master-node

master-node (root)
  |                           Tables: 4/0/0
  |                           Lag: 18s, Tick: 12
  +--: slave1-node (leaf)
                              Tables: 0/4/0
                              Lag: 18s, Tick: 12
\end{lstlisting}

Как можно заметить, возле <<Table>> содержится три цифры (x/y/z). Каждая обозначает:

\begin{itemize}
  \item x - количество таблиц в состоянии <<ok>> (replicated). На master базе указывает, что она в норме, а на slave базах - таблица синхронизирована с master базой;
  \item y - количество таблиц в состоянии half (initial copy, not finnished), у master должно быть 0, а у slave базах это указывает количество таблиц в процессе копирования;
  \item z - количество таблиц в состоянии ignored (table not replicated locally), у master должно быть 0, а у slave базах это количество таблиц, которые не добавлены для репликации с мастера (т.е. master отдает на репликацию эту таблицу, но slave их просто не забирает).
\end{itemize}

Через небольшой интервал времени все таблици должны синхронизироватся:

\begin{lstlisting}[label=lst:londiste-replica16,caption=Статус кластера]
$ londiste3 /etc/skytools/master-londiste.ini status
Queue: replika   Local node: master-node

master-node (root)
  |                           Tables: 4/0/0
  |                           Lag: 31s, Tick: 20
  +--: slave1-node (leaf)
                              Tables: 4/0/0
                              Lag: 31s, Tick: 20
\end{lstlisting}

Дополнительно Londiste позволяет просмотреть состояние таблиц и последовательностей на master и slave базах:

\begin{lstlisting}[label=lst:londiste-replica17,caption=Статус таблиц и последовательностей]
$ londiste3 /etc/skytools/master-londiste.ini tables
Tables on node
table_name               merge_state      table_attrs
-----------------------  ---------------  ---------------
public.pgbench_accounts  ok
public.pgbench_branches  ok
public.pgbench_history   ok
public.pgbench_tellers   ok

$ londiste3 /etc/skytools/master-londiste.ini seqs
Sequences on node
seq_name                        local            last_value
------------------------------  ---------------  ---------------
public.pgbench_history_hid_seq  True             33345
\end{lstlisting}



\subsubsection{Проверка}

Для проверки будем использовать \lstinline!pgbench! утилиту. Запустим добавление данных в таблицу и смотрим в логи одновлеменно:

\begin{lstlisting}[label=lst:londiste-check1,caption=Проверка]
$ pgbench -T 10 -c 5 l3simple
$ tail -f /var/log/skytools/slave1_l3simple.log
INFO {count: 1508, duration: 0.307, idle: 0.0026}
INFO {count: 1572, duration: 0.3085, idle: 0.002}
INFO {count: 1600, duration: 0.3086, idle: 0.0026}
INFO {count: 36, duration: 0.0157, idle: 2.0191}
\end{lstlisting}

Как видно по логам slave база успешно реплицируется с master базой.


\subsubsection{Каскадная репликация}

Каскадная репликация позволяет реплицировать данные с одного слейва на другой. Создадим конфиг для второго slave (конфиг будет \lstinline!/etc/skytools/slave2-londiste.ini!):

\begin{lstlisting}[label=lst:londiste-cascade1,caption=Конфиг для slave2]
[londiste3]
job_name = slave2_l3simple
db = dbname=l3simple host=slave2-host
queue_name = replika
logfile = /var/log/skytools/slave2_l3simple.log
pidfile = /var/pid/skytools/slave2_l3simple.pid

# Задержка между проверками наличия активности
# (новых пакетов данных) в секундах
loop_delay = 0.5
\end{lstlisting}

Для создания slave, от которого можно реплицировать другие базы данных используется команда \lstinline!create-branch! вместо \lstinline!create-leaf! (root, корень - master нода, предоставляет информацию для репликации; branch, ветка - нода с копией данных, с которой можно реплицировать; leaf, лист - нода с копией данными, но реплицировать с нее уже не возможно):

\begin{lstlisting}[label=lst:londiste-cascade2,caption=Инициализируем slave2]
$ psql -hslave2-host -d postgres -c "CREATE DATABASE l3simple;"
$ pg_dump -s -npublic l3simple | psql -hslave2-host l3simple
$ londiste3 /etc/skytools/slave2-londiste.ini create-branch slave2-node "dbname=l3simple host=slave2-host" --provider="dbname=l3simple host=master-host"
INFO plpgsql is installed
INFO Installing pgq
INFO   Reading from /usr/share/skytools3/pgq.sql
INFO pgq.get_batch_cursor is installed
INFO Installing pgq_ext
INFO   Reading from /usr/share/skytools3/pgq_ext.sql
INFO Installing pgq_node
INFO   Reading from /usr/share/skytools3/pgq_node.sql
INFO Installing londiste
INFO   Reading from /usr/share/skytools3/londiste.sql
INFO londiste.global_add_table is installed
INFO Initializing node
INFO Location registered
INFO Location registered
INFO Subscriber registered: slave2-node
INFO Location registered
INFO Location registered
INFO Location registered
INFO Node "slave2-node" initialized for queue "replika" with type "branch"
INFO Done
\end{lstlisting}

Далее добавляем все таблици и последовательности:

\begin{lstlisting}[label=lst:londiste-cascade3,caption=Инициализируем slave2]
$ londiste3 /etc/skytools/slave2-londiste.ini add-table --all
$ londiste3 /etc/skytools/slave2-londiste.ini add-seq --all
\end{lstlisting}

И запускаем новый демон:

\begin{lstlisting}[label=lst:londiste-cascade4,caption=Инициализируем slave2]
$ /etc/init.d/skytools3 start
INFO Starting master_l3simple
INFO Starting slave1_l3simple
INFO Starting slave2_l3simple
INFO Starting pgqd
LOG Starting pgqd 3.2
\end{lstlisting}

Повторим вышеперечисленные операции для slave3 и slave4, только поменяем provider для них:

\begin{lstlisting}[label=lst:londiste-cascade5,caption=Инициализируем slave3 и slave4]
$ londiste3 /etc/skytools/slave3-londiste.ini create-branch slave3-node "dbname=l3simple host=slave3-host" --provider="dbname=l3simple host=slave2-host"
$ londiste3 /etc/skytools/slave4-londiste.ini create-branch slave4-node "dbname=l3simple host=slave4-host" --provider="dbname=l3simple host=slave3-host"
\end{lstlisting}

В результате получаем такую картину с кластером:

\begin{lstlisting}[label=lst:londiste-cascade6,caption=Кластер с каскадной репликацией]
$ londiste3 /etc/skytools/slave4-londiste.ini status
Queue: replika   Local node: slave4-node

master-node (root)
  |                                Tables: 4/0/0
  |                                Lag: 9s, Tick: 49
  +--: slave1-node (leaf)
  |                                Tables: 4/0/0
  |                                Lag: 9s, Tick: 49
  +--: slave2-node (branch)
     |                             Tables: 4/0/0
     |                             Lag: 9s, Tick: 49
     +--: slave3-node (branch)
        |                          Tables: 4/0/0
        |                          Lag: 9s, Tick: 49
        +--: slave4-node (branch)
                                   Tables: 4/0/0
                                   Lag: 9s, Tick: 49
\end{lstlisting}

Londiste позволяет <<на лету>> изменять топологию кластера. Например, изменим <<provider>> для slave4:

\begin{lstlisting}[label=lst:londiste-cascade7,caption=Изменяем топологию]
$ londiste3 /etc/skytools/slave4-londiste.ini change-provider --provider="dbname=l3simple host=slave2-host"
$ londiste3 /etc/skytools/slave4-londiste.ini status
Queue: replika   Local node: slave4-node

master-node (root)
  |                                Tables: 4/0/0
  |                                Lag: 12s, Tick: 56
  +--: slave1-node (leaf)
  |                                Tables: 4/0/0
  |                                Lag: 12s, Tick: 56
  +--: slave2-node (branch)
     |                             Tables: 4/0/0
     |                             Lag: 12s, Tick: 56
     +--: slave3-node (branch)
     |                             Tables: 4/0/0
     |                             Lag: 12s, Tick: 56
     +--: slave4-node (branch)
                                   Tables: 4/0/0
                                   Lag: 12s, Tick: 56
\end{lstlisting}

Также топологию можно менять с сторони репликатора через команду \lstinline!takeover!:

\begin{lstlisting}[label=lst:londiste-cascade8,caption=Изменяем топологию]
$ londiste3 /etc/skytools/slave3-londiste.ini takeover slave4-node
$ londiste3 /etc/skytools/slave4-londiste.ini status
Queue: replika   Local node: slave4-node

master-node (root)
  |                                Tables: 4/0/0
  |                                Lag: 9s, Tick: 49
  +--: slave1-node (leaf)
  |                                Tables: 4/0/0
  |                                Lag: 9s, Tick: 49
  +--: slave2-node (branch)
     |                             Tables: 4/0/0
     |                             Lag: 9s, Tick: 49
     +--: slave3-node (branch)
        |                          Tables: 4/0/0
        |                          Lag: 9s, Tick: 49
        +--: slave4-node (branch)
                                   Tables: 4/0/0
                                   Lag: 9s, Tick: 49
\end{lstlisting}

Через команду \lstinline!drop-node! можно удалить slave из кластера:

\begin{lstlisting}[label=lst:londiste-cascade9,caption=Удаляем ноду]
$ londiste3 /etc/skytools/slave4-londiste.ini drop-node slave4-node
$ londiste3 /etc/skytools/slave3-londiste.ini status
Queue: replika   Local node: slave3-node

master-node (root)
  |                                Tables: 4/0/0
  |                                Lag: 9s, Tick: 49
  +--: slave1-node (leaf)
  |                                Tables: 4/0/0
  |                                Lag: 9s, Tick: 49
  +--: slave2-node (branch)
     |                             Tables: 4/0/0
     |                             Lag: 9s, Tick: 49
     +--: slave3-node (branch)
                                   Tables: 4/0/0
                                   Lag: 9s, Tick: 49
\end{lstlisting}

Команда \lstinline!tag-dead! может использоваться, что бы указать slave как не живой (прекратить на него репликацию), а через команду \lstinline!tag-alive! его можно вернуть в кластер.




\subsection{Общие задачи}

\subsubsection{Проверка состояния слейвов}

Данный запрос на мастере дает некоторую информацию о каждой очереди и слейве:

\begin{lstlisting}[language=SQL,label=lst:londiste21,caption=Проверка состояния слейвов]
# SELECT queue_name, consumer_name, lag, last_seen FROM pgq.get_consumer_info();
 queue_name |     consumer_name      |       lag       |    last_seen
------------+------------------------+-----------------+-----------------
 replika    | .global_watermark      | 00:03:37.108259 | 00:02:33.013915
 replika    | slave1_l3simple        | 00:00:32.631509 | 00:00:32.533911
 replika    | .slave1-node.watermark | 00:03:37.108259 | 00:03:05.01431
\end{lstlisting}

где \lstinline!lag! столбец показывает отставание от мастера в синхронизации, \lstinline!last_seen!~--- время последней запроса от слейва. Значение этого столбца не должно быть больше, чем 60 секунд для конфигурации по умолчанию.

\subsubsection{Удаление очереди всех событий из мастера}

При работе с Londiste может потребоваться удалить все ваши настройки для того, чтобы начать все заново. Для PGQ, чтобы остановить накопление данных, используйте следующие API:

\begin{lstlisting}[label=lst:londiste22,caption=Удаление очереди всех событий из мастера]
SELECT pgq.unregister_consumer('queue_name', 'consumer_name');
\end{lstlisting}

\subsubsection{Добавление столбца в таблицу}

Добавляем в следующей последовательности:

\begin{enumerate}
 \item добавить поле на все слейвы;
 \item BEGIN; -- на мастере;
 \item добавить поле на мастере;
 \item COMMIT;
\end{enumerate}

\subsubsection{Удаление столбца из таблицы}

\begin{enumerate}
 \item BEGIN; -- на мастере;
 \item удалить поле на мастере;
 \item COMMIT;
 \item Проверить \lstinline!lag!, когда londiste пройдет момент удаления поля;
 \item удалить поле на всех слейвах;
\end{enumerate}

Хитрость тут в том, чтобы удалить поле на слейвах только тогда, когда больше нет событий в очереди на это поле.


\subsection{Устранение неисправностей}

\subsubsection{Londiste пожирает процессор и lag растет}

Это происходит, например, если во время сбоя админ забыл перезапустить ticker. Или когда вы сделали большой UPDATE или DELETE в одной транзакции, но теперь что бы реализовать каждое событие в этом запросе создаются транзакции на слейвах \dots

Следующий запрос позволяет подсчитать, сколько событий пришло в \lstinline!pgq.subscription! в колонках \lstinline!sub_last_tick! и \lstinline!sub_next_tick!.

\begin{lstlisting}[language=SQL,label=lst:londiste24,caption=Устранение неисправностей]
SELECT count(*)
  FROM pgq.event_1,
    (SELECT tick_snapshot
      FROM pgq.tick
      WHERE tick_id BETWEEN 5715138 AND 5715139
    ) as t(snapshots)
WHERE txid_visible_in_snapshot(ev_txid, snapshots);
\end{lstlisting}

На практике это было более чем 5 миллионов и 400 тысяч событий. Чем больше событий с базы данных требуется обработать Londiste, тем больше ему требуется памяти для этого. Возможно сообщить Londiste не загружать все события сразу. Достаточно добавить в INI конфиг PgQ ticker следующую настройку:

\begin{lstlisting}[label=lst:londiste25,caption=Устранение неисправностей]
pgq_lazy_fetch = 500
\end{lstlisting}

Теперь Londiste будет брать максимум 500 событий в один пакет запросов. Остальные попадут в следующие пакеты запросов.
