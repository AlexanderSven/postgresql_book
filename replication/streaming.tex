\section{Потоковая репликация (Streaming Replication)}

\href{https://wiki.postgresql.org/wiki/Streaming_Replication}{Потоковая репликация (Streaming Replication, SR)} дает возможность непрерывно отправлять и применять WAL (Write-Ahead Log) записи на резервные сервера для создания точной копии текущего. Данная функциональность появилась у PostgreSQL начиная с 9 версии. Этот тип репликации простой, надежный и, вероятней всего, будет использоваться в качестве стандартной репликации в большинстве высоконагруженных приложений, что используют PostgreSQL.

Отличительными особенностями решения являются:

\begin{itemize}
  \item репликация всего инстанса PostgreSQL;
  \item асинхронный или синхронный механизмы репликации;
  \item простота установки;
  \item мастер база данных может обслуживать огромное количество слейвов из-за минимальной нагрузки;
\end{itemize}

К недостаткам можно отнести:

\begin{itemize}
  \item невозможность реплицировать только определенную базу данных из всех на PostgreSQL инстансе;
\end{itemize}

\subsection{Установка}

Для начала нам потребуется PostgreSQL не ниже 9 версии. Все работы, как полагается, будут проводится на Linux.

\subsection{Настройка}

Обозначим мастер сервер как \lstinline!masterdb(192.168.0.10)! и слейв как \lstinline!slavedb(192.168.0.20)!.

\subsubsection{Предварительная настройка}

Для начала позволим определенному пользователю без пароля ходить по ssh. Пусть это будет \lstinline!postgres! юзер. Если же нет, то создаем набором команд:

\begin{lstlisting}[label=lst:streaming1,caption=Создаем пользователя userssh]
$ sudo groupadd userssh
$ sudo useradd -m -g userssh -d /home/userssh -s /bin/bash \
-c "user ssh allow" userssh
\end{lstlisting}

Дальше выполняем команды от имени пользователя (в данном случае \lstinline!postgres!):

\begin{lstlisting}[label=lst:streaming2,caption=Логинимся под пользователем postgres]
$ su postgres
\end{lstlisting}

Генерим RSA-ключ для обеспечения аутентификации в условиях отсутствия возможности использовать пароль:

\begin{lstlisting}[label=lst:streaming3,caption=Генерим RSA-ключ]
$ ssh-keygen -t rsa -P ""
Generating public/private rsa key pair.
Enter file in which to save the key (/var/lib/postgresql/.ssh/id_rsa):
Created directory '/var/lib/postgresql/.ssh'.
Your identification has been saved in /var/lib/postgresql/.ssh/id_rsa.
Your public key has been saved in /var/lib/postgresql/.ssh/id_rsa.pub.
The key fingerprint is:
16:08:27:97:21:39:b5:7b:86:e1:46:97:bf:12:3d:76 postgres@localhost
\end{lstlisting}

И добавляем его в список авторизованных ключей:

\begin{lstlisting}[label=lst:streaming4,caption=Добавляем его в список авторизованных ключей]
$ cat $HOME/.ssh/id_rsa.pub >> $HOME/.ssh/authorized_keys
\end{lstlisting}

Проверить работоспособность соединения можно просто написав:

\begin{lstlisting}[label=lst:streaming5,caption=Пробуем зайти на ssh без пароля]
$ ssh localhost
\end{lstlisting}

Не забываем предварительно инициализировать \lstinline!sshd!:

\begin{lstlisting}[label=lst:streaming6,caption=Запуск sshd]
$ $/etc/init.d/sshd start
\end{lstlisting}

После успешно проделаной операции скопируйте \lstinline!$HOME/.ssh! на \lstinline!slavedb!. Теперь мы должны иметь возможность без пароля заходить с мастера на слейв и со слейва на мастер через ssh.

Также отредактируем \lstinline!pg_hba.conf! на мастере и слейве, разрешив им друг к другу доступ без пароля (тут добавляется роль r\lstinline!eplication!):

\begin{lstlisting}[label=lst:streaming7,caption=Мастер pg\_hba.conf]
host  replication  all  192.168.0.20/32  trust
\end{lstlisting}

\begin{lstlisting}[label=lst:streaming8,caption=Слейв pg\_hba.conf]
host  replication  all  192.168.0.10/32  trust
\end{lstlisting}

Не забываем после этого перегрузить postgresql на обоих серверах.

\subsubsection{Настройка мастера}

Для начала настроим masterdb. Установим параметры в \lstinline!postgresql.conf! для репликации:

\begin{lstlisting}[label=lst:streaming9,caption=Настройка мастера]
# To enable read-only queries on a standby server, wal_level must be set to
# "hot_standby". But you can choose "archive" if you never connect to the
# server in standby mode.
wal_level = hot_standby

# Set the maximum number of concurrent connections from the standby servers.
max_wal_senders = 5

# To prevent the primary server from removing the WAL segments required for
# the standby server before shipping them, set the minimum number of segments
# retained in the pg_xlog directory. At least wal_keep_segments should be
# larger than the number of segments generated between the beginning of
# online-backup and the startup of streaming replication. If you enable WAL
# archiving to an archive directory accessible from the standby, this may
# not be necessary.
wal_keep_segments = 32

# Enable WAL archiving on the primary to an archive directory accessible from
# the standby. If wal_keep_segments is a high enough number to retain the WAL
# segments required for the standby server, this may not be necessary.
archive_mode    = on
archive_command = 'cp %p /path_to/archive/%f'
\end{lstlisting}

Давайте по порядку:

\begin{itemize}
  \item \lstinline!wal_level = hot_standby!~--- сервер начнет писать в WAL логи так же как и при режиме <<archive>>, добавляя информацию, необходимую для восстановления транзакции (можно также поставить \lstinline!archive!, но тогда сервер не может быть слейвом при необходимости);
  \item \lstinline!max_wal_senders = 5!~--- максимальное количество слейвов;
  \item \lstinline!wal_keep_segments = 32!~--- минимальное количество файлов c WAL сегментами в \lstinline!pg_xlog! директории;
  \item \lstinline!archive_mode = on!~--- позволяем сохранять WAL сегменты в указанное переменной \lstinline!archive_command! хранилище. В данном случае в директорию \lstinline!/path/to/archive/!;
\end{itemize}

По умолчанию репликация асинхронная. В версии 9.1 добавили параметр \lstinline!synchronous_standby_names!, который включает синхронную репликацию. В данные параметр передается \lstinline!application_name!, который используется на слейвах в \lstinline!recovery.conf!:

\begin{lstlisting}[label=lst:streaming91,caption=recovery.conf для синхронной репликации на слейве]
restore_command = 'cp /mnt/server/archivedir/%f %p'               # e.g. 'cp /mnt/server/archivedir/%f %p'
standby_mode = on
primary_conninfo = 'host=masterdb port=59121 user=replication password=replication application_name=newcluster'            # e.g. 'host=localhost port=5432'
trigger_file = '/tmp/trig_f_newcluster'
\end{lstlisting}

После изменения параметров перегружаем PostgreSQL сервер. Теперь перейдем к \lstinline!slavedb!.

\subsubsection{Настройка слейва}
\label{subsec:streaming-slave-settings}

Для начала нам потребуется создать на \lstinline!slavedb! точную копию \lstinline!masterdb!. Перенесем данные с помощью <<Онлайн бекапа>>.

Переместимся на \lstinline!masterdb! сервер и выполним в консоли:

\begin{lstlisting}[label=lst:streaming10,caption=Выполняем на мастере]
$ psql -c "SELECT pg_start_backup('label', true)"
\end{lstlisting}

Теперь нам нужно перенести данные с мастера на слейв. Выполняем на мастере:

\begin{lstlisting}[label=lst:streaming11,caption=Выполняем на мастере]
$ rsync -C -a --delete -e ssh --exclude postgresql.conf --exclude postmaster.pid \
--exclude postmaster.opts --exclude pg_log --exclude pg_xlog \
--exclude recovery.conf master_db_datadir/ slavedb_host:slave_db_datadir/
\end{lstlisting}

где

\begin{itemize}
  \item \lstinline!master_db_datadir!~--- директория с postgresql данными на masterdb;
  \item \lstinline!slave_db_datadir!~--- директория с postgresql данными на slavedb;
  \item \lstinline!slavedb_host!~--- хост slavedb(в нашем случае - 192.168.1.20);
\end{itemize}

После копирования данных с мастера на слейв, остановим онлайн бекап. Выполняем на мастере:

\begin{lstlisting}[label=lst:streaming12,caption=Выполняем на мастере]
$ psql -c "SELECT pg_stop_backup()"
\end{lstlisting}

Для версии PostgreSQL 9.1+ можно воспользоватся командой \lstinline!pg_basebackup! (копирует базу на \lstinline!slavedb! подобным образом):

\begin{lstlisting}[label=lst:streaming122,caption=Выполняем на слейве]
$ pg_basebackup -R -D /srv/pgsql/standby --host=192.168.0.10 --port=5432
\end{lstlisting}

Устанавливаем такие же данные в конфиге \lstinline!postgresql.conf!, что и у мастера (чтобы при падении мастера слейв мог его заменить). Так же установим дополнительный параметр:

\begin{lstlisting}[label=lst:streaming13,caption=Конфиг слейва]
hot_standby = on
\end{lstlisting}

Внимание! Если на мастере поставили \lstinline!wal_level = archive!, тогда параметр оставляем по умолчанию (\lstinline!hot_standby = off!).

Далее на \lstinline!slavedb! в директории с данными PostgreSQL создадим файл \lstinline!recovery.conf! с таким содержимым:

\begin{lstlisting}[label=lst:streaming14,caption=Конфиг recovery.conf]
# Specifies whether to start the server as a standby. In streaming replication,
# this parameter must to be set to on.
standby_mode          = 'on'

# Specifies a connection string which is used for the standby server to connect
# with the primary.
primary_conninfo      = 'host=192.168.0.10 port=5432 user=postgres'

# Specifies a trigger file whose presence should cause streaming replication to
# end (i.e., failover).
trigger_file = '/path_to/trigger'

# Specifies a command to load archive segments from the WAL archive. If
# wal_keep_segments is a high enough number to retain the WAL segments
# required for the standby server, this may not be necessary. But
# a large workload can cause segments to be recycled before the standby
# is fully synchronized, requiring you to start again from a new base backup.
restore_command = 'scp masterdb_host:/path_to/archive/%f "%p"'
\end{lstlisting}

где

\begin{itemize}
  \item \lstinline!standby_mode='on'!~--- указываем серверу работать в режиме слейв;
  \item \lstinline!primary_conninfo!~--- настройки соединения слейва с мастером;
  \item \lstinline!trigger_file!~--- указываем триггер файл, при наличии которого будет остановлена репликация;
  \item \lstinline!restore_command!~--- команда, которой будут восстанавливаться WAL логи. В нашем случае через scp копируем с masterdb (\lstinline!masterdb_host! - хост masterdb);
\end{itemize}

Теперь можем запустить PostgreSQL на \lstinline!slavedb!.

\subsubsection{Тестирование репликации}

В результате можем посмотреть отставание слейвов от мастера с помощью таких команд:

\begin{lstlisting}[label=lst:streaming15,caption=Тестирование репликации]
$ psql -c "SELECT pg_current_xlog_location()" -h192.168.0.10 (masterdb)
 pg_current_xlog_location
--------------------------
 0/2000000
(1 row)

$ psql -c "select pg_last_xlog_receive_location()" -h192.168.0.20 (slavedb)
 pg_last_xlog_receive_location
-------------------------------
 0/2000000
(1 row)

$ psql -c "select pg_last_xlog_replay_location()" -h192.168.0.20 (slavedb)
 pg_last_xlog_replay_location
------------------------------
 0/2000000
(1 row)
\end{lstlisting}

Начиная с версии 9.1 добавили дополнительные view для просмотра состояния репликации. Теперь master знает все состояния slaves:

\begin{lstlisting}[label=lst:streaming151,caption=Состояние слейвов]
# SELECT * from pg_stat_replication ;
  procpid | usesysid |   usename   | application_name | client_addr | client_hostname | client_port |        backend_start         |   state   | sent_location | write_location | flush_location | replay_location | sync_priority | sync_state
 ---------+----------+-------------+------------------+-------------+-----------------+-------------+------------------------------+-----------+---------------+----------------+----------------+-----------------+---------------+------------
    17135 |    16671 | replication | newcluster       | 127.0.0.1   |                 |       43745 | 2011-05-22 18:13:04.19283+02 | streaming | 1/30008750    | 1/30008750     | 1/30008750     | 1/30008750      |             1 | sync
\end{lstlisting}

Также с версии 9.1 добавили view \lstinline!pg_stat_database_conflicts!, с помощью которой на слейв базах можно просмотреть сколько запросов было отменено и по каким причинам:

\begin{lstlisting}[label=lst:streaming152,caption=Состояние слейва]
# SELECT * from pg_stat_database_conflicts ;
  datid |  datname  | confl_tablespace | confl_lock | confl_snapshot | confl_bufferpin | confl_deadlock
 -------+-----------+------------------+------------+----------------+-----------------+----------------
      1 | template1 |                0 |          0 |              0 |               0 |              0
  11979 | template0 |                0 |          0 |              0 |               0 |              0
  11987 | postgres  |                0 |          0 |              0 |               0 |              0
  16384 | marc      |                0 |          0 |              1 |               0 |              0
\end{lstlisting}

Еще проверить работу репликации можно с помощью утилиты \lstinline!ps!:

\begin{lstlisting}[label=lst:streaming16,caption=Тестирование репликации]
[masterdb] $ ps -ef | grep sender
postgres  6879  6831  0 10:31 ?        00:00:00 postgres: wal sender process postgres 127.0.0.1(44663) streaming 0/2000000

[slavedb] $ ps -ef | grep receiver
postgres  6878  6872  1 10:31 ?        00:00:01 postgres: wal receiver process   streaming 0/2000000
\end{lstlisting}

Давайте проверим реприкацию и выполним на мастере:

\begin{lstlisting}[language=SQL,label=lst:streaming17,caption=Выполняем на мастере]
$ psql test_db
test_db=# create table test3(id int not null primary key,name varchar(20));
NOTICE:  CREATE TABLE / PRIMARY KEY will create implicit index "test3_pkey" for table "test3"
CREATE TABLE
test_db=# insert into test3(id, name) values('1', 'test1');
INSERT 0 1
test_db=#
\end{lstlisting}

Теперь проверим на слейве результат:

\begin{lstlisting}[language=SQL,label=lst:streaming18,caption=Выполняем на слейве]
$ psql test_db
test_db=# select * from test3;
 id | name
----+-------
  1 | test1
(1 row)
\end{lstlisting}

Как видим, таблица с данными успешно скопирована с мастера на слейв. Более подробно по настройке данной репликации можно почитать из \href{https://wiki.postgresql.org/wiki/Streaming_Replication}{официальной wiki}.

\subsection{Общие задачи}

\subsubsection{Переключение на слейв при падении мастера}

Достаточно создать триггер файл (\lstinline!trigger_file!) на слейве, который перестанет читать данные с мастера.

\subsubsection{Остановка репликации на слейве}

Создать триггер файл (\lstinline!trigger_file!) на слейве. Также с версии 9.1 добавили функции \lstinline!pg_xlog_replay_pause()! и \lstinline!pg_xlog_replay_resume()! для остановки и возобновления репликации.

\subsubsection{Перезапуск репликации после сбоя}

Повторяем операции из раздела <<\nameref{subsec:streaming-slave-settings}>>. Хочется заметить, что мастер при этом не нуждается в остановке при выполнении данной задачи.

\subsubsection{Перезапуск репликации после сбоя слейва}

Перезагрузить PostgreSQL на слейве после устранения сбоя.

\subsubsection{Повторно синхронизировать репликации на слейве}

Это может потребоваться, например, после длительного отключения от мастера. Для этого останавливаем PostgreSQL на слейве и повторяем операции из раздела <<\nameref{subsec:streaming-slave-settings}>>.


\subsection{Repmgr}

\href{http://www.repmgr.org/}{Repmgr}~--- набор инструментов для управления потоковой репликацией и восстановления после сбоя кластера PostgreSQL серверов. Он автоматизирует настройку резервных серверов, мониторинг репликации, а также помогает выполнять задачи администрированию кластера, такие как отказоустойчивость (failover) или переключение мастера-слейва (слейв становится мастером, а мастер - слейвом). Repmgr работает с версии PostgreSQL 9.3 и выше.

Repmgr состоит из двух утилит:

\begin{itemize}
  \item \lstinline!repmgr!~--- инструмент командной строки (cli), который используется для административных задач, таких как:
  \begin{itemize}
    \item создание слейвов;
    \item переключение слейва в режим мастера;
    \item переключение между собой мастер и слейв серверов;
    \item отображение состояния кластера;
  \end{itemize}
  \item \lstinline!repmgrd!~--- демон, который мониторит кластер серверов и выполняет такие задачи:
  \begin{itemize}
    \item мониторинг и логирование эффективности репликации;
    \item автоматическое переключение слейва в мастер при обнаружении проблем у текущего мастера (failover);
    \item посылка сообщений о событиях в кластере через заданые пользователем скрипты;
  \end{itemize}
\end{itemize}

\subsubsection{Пример использования: автоматическое переключение слейва в мастер}

Для использования failover потребуется добавить \lstinline!repmgr_funcs! в \lstinline!postgresql.conf!:

\begin{lstlisting}[language=Bash,label=lst:repmgr1,caption=repmgr\_funcs]
shared_preload_libraries = 'repmgr_funcs'
\end{lstlisting}

И добавить настройки в \lstinline!repmgr.conf!:

\begin{lstlisting}[language=Bash,label=lst:repmgr2,caption=repmgr.conf]
failover=automatic
promote_command='repmgr standby promote -f /etc/repmgr.conf --log-to-file'
follow_command='repmgr standby follow -f /etc/repmgr.conf --log-to-file'
\end{lstlisting}

Для демонстрации автоматического failover, настроен кластер с тремя узлами репликации (один мастер и два слейв сервера), так что таблица \lstinline!repl_nodes! выглядит следующим образом:

\begin{lstlisting}[language=SQL,label=lst:repmgr3,caption=repl\_nodes]
# SELECT id, type, upstream_node_id, priority, active FROM repmgr_test.repl_nodes ORDER BY id;
 id |  type   | upstream_node_id | priority | active
----+---------+------------------+----------+--------
  1 | master  |                  |      100 | t
  2 | standby |                1 |      100 | t
  3 | standby |                1 |      100 | t
(3 rows)
\end{lstlisting}

После запуска \lstinline!repmgrd! демона на каждом сервере в режиме ожидания, убеждаемся что он мониторит кластер:

\begin{lstlisting}[language=Bash,label=lst:repmgr4,caption=logs]
[2016-01-05 13:15:40] [INFO] checking cluster configuration with schema 'repmgr_test'
[2016-01-05 13:15:40] [INFO] checking node 2 in cluster 'test'
[2016-01-05 13:15:40] [INFO] reloading configuration file and updating repmgr tables
[2016-01-05 13:15:40] [INFO] starting continuous standby node monitoring
\end{lstlisting}

Теперь остановим мастер базу:

\begin{lstlisting}[language=Bash,label=lst:repmgr5,caption=Остановка текущего мастера]
pg_ctl -D /path/to/node1/data -m immediate stop
\end{lstlisting}

\lstinline!repmgrd! автоматически замечает падение мастера и переключает один из слейвов в мастер:

\begin{lstlisting}[language=Bash,label=lst:repmgr6,caption=Переключение слейва в мастер]
[2016-01-06 18:32:58] [WARNING] connection to upstream has been lost, trying to recover... 15 seconds before failover decision
[2016-01-06 18:33:03] [WARNING] connection to upstream has been lost, trying to recover... 10 seconds before failover decision
[2016-01-06 18:33:08] [WARNING] connection to upstream has been lost, trying to recover... 5 seconds before failover decision
...
[2016-01-06 18:33:18] [NOTICE] this node is the best candidate to be the new master, promoting...
...
[2016-01-06 18:33:20] [NOTICE] STANDBY PROMOTE successful
\end{lstlisting}

Также переключает оставшийся слейв на новый мастер:

\begin{lstlisting}[language=Bash,label=lst:repmgr7,caption=Переключение слейва на новый мастер]
[2016-01-06 18:32:58] [WARNING] connection to upstream has been lost, trying to recover... 15 seconds before failover decision
[2016-01-06 18:33:03] [WARNING] connection to upstream has been lost, trying to recover... 10 seconds before failover decision
[2016-01-06 18:33:08] [WARNING] connection to upstream has been lost, trying to recover... 5 seconds before failover decision
...
[2016-01-06 18:33:23] [NOTICE] node 2 is the best candidate for new master, attempting to follow...
[2016-01-06 18:33:23] [INFO] changing standby's master
...
[2016-01-06 18:33:25] [NOTICE] node 3 now following new upstream node 2
\end{lstlisting}

Таблица \lstinline!repl_nodes! будет обновлена, чтобы отразить новую ситуацию~--- старый мастер \lstinline!node1! помечен как неактивный, и слейв \lstinline!node3! теперь работает от нового мастера \lstinline!node2!:

\begin{lstlisting}[language=SQL,label=lst:repmgr8,caption=Результат после failover]
# SELECT id, type, upstream_node_id, priority, active from repl_nodes ORDER BY id;
 id |  type   | upstream_node_id | priority | active
----+---------+------------------+----------+--------
  1 | master  |                  |      100 | f
  2 | master  |                  |      100 | t
  3 | standby |                2 |      100 | t
(3 rows)
\end{lstlisting}

В таблицу \lstinline!repl_events! будут добавлены записи того, что произошло с каждым сервером во время failover:

\begin{lstlisting}[language=SQL,label=lst:repmgr9,caption=Результат после failover]
# SELECT node_id, event, successful, details from repmgr_test.repl_events where event_timestamp>='2016-01-06 18:30';
 node_id |          event           | successful |                         details
---------+--------------------------+------------+----------------------------------------------------------
       2 | standby_promote          | t          | node 2 was successfully promoted to master
       2 | repmgrd_failover_promote | t          | node 2 promoted to master; old master 1 marked as failed
       3 | repmgrd_failover_follow  | t          | node 3 now following new upstream node 2
(3 rows)
\end{lstlisting}

\subsubsection{Заключение}

Более подробно по функционалу, его настройках и ограничениях доступно в \href{https://github.com/2ndQuadrant/repmgr/blob/master/README.md}{официальном репозитории}.

\subsection{Patroni}

\href{https://github.com/zalando/patroni}{Patroni}~--- это демон на Python, позволяющий автоматически обслуживать кластеры PostgreSQL с потоковой репликацией.

Особенности:

\begin{itemize}
  \item Использует потоковую PostgreSQL репликацию (асинхронная и синхронная репликация);
  \item Интеграция с \href{https://kubernetes.io/}{Kubernetes};
  \item Поддержания актуальности кластера и выборов мастера используются распределенные \href{https://en.wikipedia.org/wiki/Distributed_control_system}{DCS} хранилища (поддерживаются \href{https://zookeeper.apache.org/}{Zookeeper}, \href{https://coreos.com/etcd}{Etcd} или \href{https://www.consul.io/}{Consul});
  \item Автоматическое <<service discovery>> и динамическая реконфигурация кластера;
  \item Состояние кластера можно получить как запросами в DCS, так и напрямую к Patroni через HTTP запросы;
\end{itemize}

Информация по настройке и использованию Patroni находится в \href{https://patroni.readthedocs.io/en/latest/}{официальной документации} проекта.

\subsection{Stolon}

\href{https://github.com/sorintlab/stolon}{Stolon}

TODO

