\section{PostgreSQL Bi-Directional Replication (BDR)}
\label{sec:bdr}

\href{https://2ndquadrant.com/en/resources/bdr/}{BDR (Bi-Directional Replication)} это новая функциональность добавленая в ядро PostgreSQL которая предоставляет расширенные средства для репликации. На данный момент это реализовано в виде небольшого патча и модуля для 9.4 версии. Заявлено что полностью будет только в PostgreSQL 9.6 (разработчики решили не заниматься поддержкой патча для 9.5, а сосредоточиться на добавление патчей в сам PostgreSQL). BDR позволяет создавать географически распределенные асинхронные мульти-мастер конфигурации используя для этого встроенную логическую потоковую репликацию LLSR (Logical Log Streaming Replication).

BDR не является инструментом для кластеризации, т.к. здесь нет каких-либо глобальных менеджеров блокировок или координаторов транзакций. Каждый узел не зависит от других, что было бы невозможно в случае использования менеджеров блокировки. Каждый из узлов содержит локальную копию данных идентичную данным на других узлах. Запросы также выполняются только локально. При этом каждый из узлов внутренне консистентен в любое время, целиком же группа серверов является согласованной в конечном счете (eventually consistent). Уникальность BDR заключается в том что она непохожа ни на встроенную потоковую репликацию, ни на существующие trigger-based решения (Londiste, Slony, Bucardo).

Самым заметным отличием от потоковой репликации является то, что BDR (LLSR) оперирует базами (per-database replication), а классическая PLSR реплицирует целиком инстанс (per-cluster replication), т.е. все базы внутри инстанса. Существующие ограничения и особенности:

\begin{itemize}
  \item Все изменения данных вызываемые \lstinline!INSERT/DELETE/UPDATE! реплицируются (\lstinline!TRUNCATE! на момент написания статьи пока не реализован);
  \item Большинство операции изменения схемы (DDL) реплицируются успешно. Неподдерживаемые DDL фиксируются модулем репликации и отклоняются с выдачей ошибкой (на момент написания не работал \lstinline!CREATE TABLE ... AS!);
  \item Определения таблиц, типов, расширений и т.п. должны быть идентичными между upstream и downstream мастерами;
  \item Действия которые отражаются в WAL, но непредставляются в виде логических изменений не реплицируются на другой узел (запись полных страниц, вакуумация таблиц и т.п.). Таким образом логическая потоковая репликация (LLSR) избавлена от некоторой части накладных расходов которые присутствуют в физической потоковой репликации PLSR (тем не менее это не означает что LLSR требуется меньшая пропускная способность сети чем для PLSR);
\end{itemize}

Небольшое примечание: временная остановка репликации осуществляется выключением downstream мастера. Однако стоит отметить что остановленная реплика приводит к тому что upstream мастер продолжит накапливать WAL журналы что в свою очередь может привести к неконтролируемому расходу пространства на диске. Поэтому крайне не рекомендуется надолго выключать реплику. Удаление реплики навсегда осуществляется через удаление конфигурации BDR на downstream сервере с последующим перезапуском downstream мастера. Затем нужно удалить соответствующий слот репликации на upstream мастере с помощью функции \lstinline!pg_drop_replication_slot('slotname')!. Доступные слоты можно просмотреть с помощью функции \lstinline!pg_get_replication_slots!.

На текущий момент собрать BDR можно из исходников по \href{https://wiki.postgresql.org/wiki/BDR_Quick_Start}{данному мануалу}. С официальным принятием данных патчей в ядро PostgreSQL данный раздел про BDR будет расширен и дополнен.