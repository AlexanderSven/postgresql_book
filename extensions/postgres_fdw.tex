\section{Postgres\_fdw}

\href{https://www.postgresql.org/docs/current/static/postgres-fdw.html}{Postgres\_fdw}~--- расширение, которое позволяет подключить PostgreSQL к PostgreSQL, которые могут находится на разных хостах.

\subsection{Установка и использование}

Для начала инициализируем расширение в базе данных:

\begin{lstlisting}[language=SQL,label=lst:postgresfdw1,caption=Инициализация postgres\_fdw]
# CREATE EXTENSION postgres_fdw;
\end{lstlisting}

Далее создадим сервер подключений, который будет содержать данные для подключения к другой PostgreSQL базе:

\begin{lstlisting}[language=SQL,label=lst:postgresfdw2,caption=Создание сервера]
# CREATE SERVER slave_db
FOREIGN DATA WRAPPER postgres_fdw
OPTIONS (host 'slave.example.com', dbname 'exampledb', port '5432');
\end{lstlisting}

После этого нужно создать \lstinline!USER MAPPING!, которое создаёт сопоставление пользователя на внешнем сервере:

\begin{lstlisting}[language=SQL,label=lst:postgresfdw3,caption=USER MAPPING]
# CREATE USER MAPPING FOR admin
SERVER slave_db
OPTIONS (user 'admin', password 'password');
\end{lstlisting}

Теперь можно импортировать таблицы:

\begin{lstlisting}[language=SQL,label=lst:postgresfdw4,caption=Импорт таблицы]
# CREATE FOREIGN TABLE fdw_users (
  id             serial,
  username       text not null,
  password       text,
  created_on     timestamptz not null,
  last_logged_on timestamptz not null
)
SERVER slave_db
OPTIONS (schema_name 'public', table_name 'users');
\end{lstlisting}

Для того, чтобы не импортировать каждую таблицу отдельно, можно воспользоваться \lstinline!IMPORT FOREIGN SCHEMA! командой:

\begin{lstlisting}[language=SQL,label=lst:postgresfdw5,caption=Импортируем таблицы]
# IMPORT FOREIGN SCHEMA public
LIMIT TO (users, pages)
FROM SERVER slave_db INTO fdw;
\end{lstlisting}

Теперь можно проверить таблицы:

\begin{lstlisting}[language=SQL,label=lst:postgresfdw6,caption=SELECT]
# SELECT * FROM fdw_users LIMIT 1;
-[ RECORD 1 ]--+---------------------------------
id             | 1
username       | 64ec7083d7facb7c5d97684e7f415b65
password       | b82af3966b49c9ef0f7829107db642bc
created_on     | 2017-02-21 05:07:25.619561+00
last_logged_on | 2017-02-19 21:03:35.651561+00
\end{lstlisting}

По умолчанию, из таблиц можно не только читать, но и изменять в них данные (\lstinline!INSERT/UPDATE/DELETE!). \lstinline!updatable! опция может использовать для подключения к серверу в режиме <<только на чтение>>:

\begin{lstlisting}[language=SQL,label=lst:postgresfdw7,caption=Read-only mode]
# ALTER SERVER slave_db
OPTIONS (ADD updatable 'false');
ALTER SERVER
# DELETE FROM fdw_users WHERE id < 10;
ERROR:  foreign table "fdw_users" does not allow deletes
\end{lstlisting}

Данную опцию можно установить не только на уровне сервера, но и на уровне отдельных таблиц:

\begin{lstlisting}[language=SQL,label=lst:postgresfdw8,caption=Read-only mode для таблицы]
# ALTER FOREIGN TABLE fdw_users
OPTIONS (ADD updatable 'false');
\end{lstlisting}


\subsection{Postgres\_fdw vs DBLink}

Как можно было заметить, postgres\_fdw и dblink выполняют одну и ту же работу~--- подключение одной PostgreSQL базы к другой. Что лучше использовать в таком случае?

PostgreSQL FDW (Foreign Data Wrapper) более новый и рекомендуемый метод подключения к другим источникам данных. Хотя функциональность dblink похожа на FDW, последний является более SQL совместимым и может обеспечивать улучшеную производительность по сравнению с dblink подключениями. Кроме того, в отличии от postgres\_fdw, dblink не способен сделать данные <<только на чтение>>. Это может быть достаточно важно, если требуется обеспечить, чтобы данные с другой базы нельзя было изменять.

В dblink подключения работают только в течение работы сесии и их требуется пересоздавать каждый раз. Postgres\_fdw создает постоянное подключение к другой базе данных. Это может быть как хорошо, так плохо, в зависимости от потребностей.

Из положительных сторон dblink можно отнести множество полезных комманд, которые позволяю использовать его для программирования полезного функционала. Также dblink работает в версиях PostgreSQL 8.3 и выше, в то время как postgres\_fdw работает только в PostgreSQL 9.3 и выше (такое может возникнуть, если нет возможности обновить PostgreSQL базу).
