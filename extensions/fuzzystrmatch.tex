\section{Fuzzystrmatch}
\textbf{Лицензия}: Open Source

Fuzzystrmatch предоставляет несколько функций для определения сходства и расстояния между строками. Функция soundex используется для согласования сходно звучащих имен путем преобразования их в одинаковый код. Функция difference преобразует две строки в soundex код, а затем сообщает количество совпадающих позиций кода. В soundex код состоит из четырех символов, поэтому результат будет от нуля до четырех: 0~--- не совпадают, 4~--- точное совпадение (таким образом, функция названа неверно~--- как название лучше подходит similarity):

\begin{lstlisting}[label=lst:ext_fuzzystrmatch1,caption=soundex]
# CREATE EXTENSION fuzzystrmatch;
CREATE EXTENSION
# SELECT soundex('hello world!');
 soundex
---------
 H464
(1 row)

# SELECT soundex('Anne'), soundex('Ann'), difference('Anne', 'Ann');
 soundex | soundex | difference
---------+---------+------------
 A500    | A500    |          4
(1 row)

# SELECT soundex('Anne'), soundex('Andrew'), difference('Anne', 'Andrew');
 soundex | soundex | difference
---------+---------+------------
 A500    | A536    |          2
(1 row)

# SELECT soundex('Anne'), soundex('Margaret'), difference('Anne', 'Margaret');
 soundex | soundex | difference
---------+---------+------------
 A500    | M626    |          0
(1 row)

# CREATE TABLE s (nm text);
CREATE TABLE
# INSERT INTO s VALUES ('john'), ('joan'), ('wobbly'), ('jack');
INSERT 0 4
# SELECT * FROM s WHERE soundex(nm) = soundex('john');
  nm
------
 john
 joan
(2 rows)

# SELECT * FROM s WHERE difference(s.nm, 'john') > 2;
  nm
------
 john
 joan
 jack
(3 rows)
\end{lstlisting}

Функция levenshtein вычисляет расстояние Левенштейна\footnote{http://en.wikipedia.org/wiki/Levenshtein\_distance} между двумя строками. levenshtein\_less\_equal ускоряется функцию levenshtein для маленьких значений расстояния:

\begin{lstlisting}[label=lst:ext_fuzzystrmatch2,caption=levenshtein]
# SELECT levenshtein('GUMBO', 'GAMBOL');
 levenshtein
-------------
           2
(1 row)

# SELECT levenshtein('GUMBO', 'GAMBOL', 2, 1, 1);
 levenshtein
-------------
           3
(1 row)

# SELECT levenshtein_less_equal('extensive', 'exhaustive', 2);
 levenshtein_less_equal
------------------------
                      3
(1 row)

test=# SELECT levenshtein_less_equal('extensive', 'exhaustive', 4);
 levenshtein_less_equal
------------------------
                      4
(1 row)
\end{lstlisting}

Функция metaphone, как и soundex, построена на идее создания кода для строки: две строки, которые будут считатся похожими, будут иметь одинаковые коды. Последним параметром указывается максимальная длина metaphone кода. Функция dmetaphone вычисляет два <<как звучит>> кода для строки~--- <<первичный>> и <<альтернативный>>:

\begin{lstlisting}[label=lst:ext_fuzzystrmatch3,caption=metaphone]
# SELECT metaphone('GUMBO', 4);
 metaphone
-----------
 KM
(1 row)
# SELECT dmetaphone('postgresql');
 dmetaphone
------------
 PSTK
(1 row)

# SELECT dmetaphone_alt('postgresql');
 dmetaphone_alt
----------------
 PSTK
(1 row)
\end{lstlisting}

