\section{PLV8}
\textbf{Лицензия}: Open Source

\textbf{Ссылка}: \href{http://code.google.com/p/plv8js/}{code.google.com/p/plv8js}

PLV8 является расширением, которое предоставляет PostgreSQL процедурный язык с движком V8 JavaScript. С помощью этого расширения можно писать в PostgreSQL JavaScript функции, которые можно вызывать из SQL.

\subsection{Скорость работы}

V8\footnote{http://en.wikipedia.org/wiki/V8\_(JavaScript\_engine)} компилирует JavaScript код непосредственно в собственный машинный код и с помощью этого достигается высокая скорость работы. Для примера расмотрим расчет числа Фибоначи. Вот функция написана на plpgsql:

\begin{lstlisting}[label=lst:plv8js1,caption=Фибоначи на plpgsql]
CREATE OR REPLACE FUNCTION
psqlfib(n int) RETURNS int AS $$
 BEGIN
     IF n < 2 THEN
         RETURN n;
     END IF;
     RETURN psqlfib(n-1) + psqlfib(n-2);
 END;
$$ LANGUAGE plpgsql IMMUTABLE STRICT;
\end{lstlisting}

Замерим скорость её работы:

\begin{lstlisting}[label=lst:plv8js2,caption=Скорость расчета числа Фибоначи на plpgsql]
SELECT n, psqlfib(n) FROM generate_series(0,30,5) as n;
 n  | psqlfib 
----+---------
  0 |       0
  5 |       5
 10 |      55
 15 |     610
 20 |    6765
 25 |   75025
 30 |  832040
(7 rows)

Time: 16003,257 ms
\end{lstlisting}

Теперь сделаем тоже самое, но с использованием PLV8:

\begin{lstlisting}[label=lst:plv8js3,caption=Фибоначи на plv8]
CREATE OR REPLACE FUNCTION
fib(n int) RETURNS int as $$

  function fib(n) {
    return n<2 ? n : fib(n-1) + fib(n-2)
  }
  return fib(n)

$$ LANGUAGE plv8 IMMUTABLE STRICT;
\end{lstlisting}

Замерим скорость работы:

\begin{lstlisting}[label=lst:plv8js4,caption=Скорость расчета числа Фибоначи на plv8]
SELECT n, fib(n) FROM generate_series(0,30,5) as n;
 n  |  fib   
----+--------
  0 |      0
  5 |      5
 10 |     55
 15 |    610
 20 |   6765
 25 |  75025
 30 | 832040
(7 rows)

Time: 59,254 ms
\end{lstlisting}

Как видим PLV8 приблизительно в 270 (16003.257/59.254) раз быстрее plpgsql. Можно ускорить работу расчета числа Фибоначи на PLV8 за счет кеширования:

\begin{lstlisting}[label=lst:plv8js5,caption=Фибоначи на plv8]
CREATE OR REPLACE FUNCTION
fib1(n int) RETURNS int as $$
  var memo = {0: 0, 1: 1};
  function fib(n) {
    if(!(n in memo))
      memo[n] = fib(n-1) + fib(n-2)
    return memo[n]
  }
  return fib(n);
$$ LANGUAGE plv8 IMMUTABLE STRICT;
\end{lstlisting}

Замерим скорость работы:

\begin{lstlisting}[label=lst:plv8js6,caption=Скорость расчета числа Фибоначи на plv8]
SELECT n, fib1(n) FROM generate_series(0,30,5) as n;
 n  |  fib1  
----+--------
  0 |      0
  5 |      5
 10 |     55
 15 |    610
 20 |   6765
 25 |  75025
 30 | 832040
(7 rows)

Time: 0,766 ms
\end{lstlisting}

Теперь расчет на PLV8 приблизительно в 20892 (16003.257/0.766) раза быстрее, чем на plpgsql.

\subsection{Использование}

Одно из полезных применение PLV8 является создание из PostgreSQL документоориенторованое хранилище. Для хранения неструктурированных данных можно использовать hstore, но у него есть свои недостатки:

\begin{itemize}
\item нет вложенности
\item все данные (ключ и значение по ключу) это строка
\end{itemize}

Для хранения данных многие документоориентированые базы данных используют JSON (MongoDB, CouchDB, Couchbase и т.д.). Для этого начиная с PostgreSQL 9.2 добавлен тип данных JSON. Такой тип можно добавить для PostgreSQL 9.1 и ниже используя PLV8 и DOMAIN:

\begin{lstlisting}[label=lst:plv8js7,caption=Создание типа JSON]
CREATE OR REPLACE FUNCTION 
valid_json(json text)
RETURNS BOOLEAN AS $$
  try { 
    JSON.parse(json); return true; 
  } catch(e) { 
    return false;
  }
$$ LANGUAGE plv8 IMMUTABLE STRICT;

CREATE DOMAIN json AS TEXT 
CHECK(valid_json(VALUE));
\end{lstlisting}

Функция <<valid\_json>> используется для проверки JSON данных. Пример использования:

\begin{lstlisting}[label=lst:plv8js8,caption=Проверка JSON]
$ INSERT INTO members 
VALUES('not good json');
ERROR:  value for domain json 
violates check constraint "json_check"
$ INSERT INTO members 
VALUES('{"good": "json", "is": true}');
INSERT 0 1
$ select * from members;
	    profile            
------------------------------
  {"good": "json", "is": true}
(1 row)
\end{lstlisting}

Расмотрим пример использования JSON для хранения данных и PLV8 для их поиска. Для начала создадим таблицу и заполним её данными:

\begin{lstlisting}[label=lst:plv8js9,caption=Таблица с JSON полем]
$ CREATE TABLE members ( id SERIAL, profile json );
$ SELECT count(*) FROM members;
  count  
---------
 1000000
(1 row)

Time: 201.109 ms
\end{lstlisting}

В <<profile>> поле мы записали приблизительно такую структуру JSON:

\begin{lstlisting}[label=lst:plv8js10,caption=JSON структура]
{                                  +
  "name": "Litzy Satterfield",     +
  "age": 24,                       +
  "siblings": 2,                   +
  "faculty": false,                +
  "numbers": [                     +
    {                              +
      "type":   "work",            +
      "number": "684.573.3783 x368"+
    },                             +
    {                              +
      "type":   "home",            +
      "number": "625.112.6081"     +
    }                              +
  ]                                +
}
\end{lstlisting}

Теперь создадим функцию для вывода значения по ключу из JSON (в данном случае ожидаем цифру):

\begin{lstlisting}[label=lst:plv8js11,caption=Функция для JSON]
CREATE OR REPLACE FUNCTION get_numeric(json_raw json, key text)
RETURNS numeric AS $$
  var o = JSON.parse(json_raw);
  return o[key];
$$ LANGUAGE plv8 IMMUTABLE STRICT;
\end{lstlisting}

Теперь мы можем произвести поиск по таблице фильтруя по значениям ключей <<age>>, <<siblings>> или другим циферным:

\begin{lstlisting}[label=lst:plv8js12,caption=Поиск по данным JSON]
$ SELECT * FROM members WHERE get_numeric(profile, 'age') = 36;
Time: 9340.142 ms
$ SELECT * FROM members WHERE get_numeric(profile, 'siblings') = 1;
Time: 14320.032 ms
\end{lstlisting}

Поиск работает, но скорость очень маленькая. Чтобы улучшить скорость, нужно создать функциональные индексы:

\begin{lstlisting}[label=lst:plv8js13,caption=Создание индексов]
CREATE INDEX member_age ON members (get_numeric(profile, 'age'));
CREATE INDEX member_siblings ON members (get_numeric(profile, 'siblings'));
\end{lstlisting}

С индексами скорость поиска по JSON станет достаточно высокая:

\begin{lstlisting}[label=lst:plv8js14,caption=Поиск по данным JSON с индексами]
$ SELECT * FROM members WHERE get_numeric(profile, 'age') = 36;
Time: 57.429 ms
$ SELECT * FROM members WHERE get_numeric(profile, 'siblings') = 1;
Time: 65.136 ms
$ SELECT count(*) from members where  get_numeric(profile, 'age') = 26 and get_numeric(profile, 'siblings') = 1;
Time: 106.492 ms
\end{lstlisting}

Получилось отличное документоориентированое хранилище из PostgreSQL.

PLV8 позволяет использовать некоторые JavaScript библиотеки внутри PostgreSQL. Вот пример рендера Mustache\footnote{http://mustache.github.com/} темплейтов:

\begin{lstlisting}[label=lst:plv8js15,caption=Функция для рендера Mustache темплейтов]
CREATE OR REPLACE FUNCTION mustache(template text, view json)
RETURNS text as $$
  // …400 lines of mustache.js…
  return Mustache.render(template, JSON.parse(view))
$$ LANGUAGE plv8 IMMUTABLE STRICT;
\end{lstlisting}

\begin{lstlisting}[label=lst:plv8js16,caption=Рендер темплейтов]
$ SELECT mustache(
  'hello {{#things}}{{.}} {{/things}}:) {{#data}}{{key}}{{/data}}',
  '{"things": ["world", "from", "postgresql"], "data": {"key": "and me"}}'
);
		mustache                
---------------------------------------
  hello world from postgresql :) and me
(1 row)

Time: 0.837 ms
\end{lstlisting}

Этот пример показывает как можно использовать PLV8. В действительности рендерить Mustache в PostgreSQL не лучшая идея.

\subsection{Вывод}

PLV8 расширение предоставляет PostgreSQL процедурный язык с движком V8 JavaScript, с помощью которого можно работать с JavaScript билиотеками, индексировать JSON данные и использовать его как более быстрый язык.