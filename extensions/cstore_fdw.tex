\section{Cstore\_fdw}

\href{https://citusdata.github.io/cstore\_fdw/}{Cstore\_fdw} расширение реализует модель хранения данных на базе \href{https://en.wikipedia.org/wiki/Column-oriented\_DBMS}{семейства столбцов} (column-oriented systems) для PostgreSQL. Такое хранение данных   обеспечивает заметные преимущества для аналитических задач (\href{https://ru.wikipedia.org/wiki/OLAP}{OLAP}, \href{https://en.wikipedia.org/wiki/Data\_warehouse}{data warehouse}), поскольку требуется считывать меньше данных с диска (благодаря формату хранения и компресии). Расширение использует \href{https://cwiki.apache.org/confluence/display/Hive/LanguageManual+ORC#LanguageManualORC-ORCFileFormat}{Optimized Row Columnar} (ORC) формат для размещения данных на диске, который имеет следующие преимущества:

\begin{itemize}
  \item Уменьшение (сжатие) размера данных в памяти и на диске в 2-4 раза. Можно добавить в расширение другой кодек для сжатия (алгоритм Лемпеля-Зива, LZ присутствует в расширении);
  \item Считывание с диска только тех данных, которые требуются. Повышается производительность по I/O диска для других запросов;
  \item Хранение минимального/максимального значений для групп полей (skip index, индекс с пропусками), что помогает пропустить не требуемые данные на диске при выборке;
\end{itemize}

