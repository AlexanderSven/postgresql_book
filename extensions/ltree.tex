\section{Ltree}

\href{https://www.postgresql.org/docs/current/static/ltree.html}{Ltree}~--- расширение, которое позволяет хранить древовидные структуры в виде меток, а также предоставляет широкие возможности поиска по ним.

\subsection{Почему Ltree?}

\begin{itemize}
  \item Реализация алгоритма Materialized Path (достаточно быстрый как на запись, так и на чтение);
  \item Как правило данное решение будет быстрее, чем использовании CTE (Common Table Expressions) или рекурсивный функции (постоянно будут пересчитываться ветвления);
  \item Встроены механизмы поиска по дереву;
  \item Индексы;
\end{itemize}

\subsection{Установка и использование}

Для начала активируем расширение для базы данных:

\begin{lstlisting}[language=SQL,label=lst:pgltree1,caption=Ltree]
# CREATE EXTENSION ltree;
\end{lstlisting}

Далее создадим таблицу коментариев, которые будут хранится как дерево:

\begin{lstlisting}[language=SQL,label=lst:pgltree2,caption=Ltree]
CREATE TABLE comments (user_id integer, description text, path ltree);
INSERT INTO comments (user_id, description, path) VALUES ( 1, md5(random()::text), '0001');
INSERT INTO comments (user_id, description, path) VALUES ( 2, md5(random()::text), '0001.0001.0001');
INSERT INTO comments (user_id, description, path) VALUES ( 2, md5(random()::text), '0001.0001.0001.0001');
INSERT INTO comments (user_id, description, path) VALUES ( 1, md5(random()::text), '0001.0001.0001.0002');
INSERT INTO comments (user_id, description, path) VALUES ( 5, md5(random()::text), '0001.0001.0001.0003');
INSERT INTO comments (user_id, description, path) VALUES ( 6, md5(random()::text), '0001.0002');
INSERT INTO comments (user_id, description, path) VALUES ( 6, md5(random()::text), '0001.0002.0001');
INSERT INTO comments (user_id, description, path) VALUES ( 6, md5(random()::text), '0001.0003');
INSERT INTO comments (user_id, description, path) VALUES ( 8, md5(random()::text), '0001.0003.0001');
INSERT INTO comments (user_id, description, path) VALUES ( 9, md5(random()::text), '0001.0003.0002');
INSERT INTO comments (user_id, description, path) VALUES ( 11, md5(random()::text), '0001.0003.0002.0001');
INSERT INTO comments (user_id, description, path) VALUES ( 2, md5(random()::text), '0001.0003.0002.0002');
INSERT INTO comments (user_id, description, path) VALUES ( 5, md5(random()::text), '0001.0003.0002.0003');
INSERT INTO comments (user_id, description, path) VALUES ( 7, md5(random()::text), '0001.0003.0002.0002.0001');
INSERT INTO comments (user_id, description, path) VALUES ( 20, md5(random()::text), '0001.0003.0002.0002.0002');
INSERT INTO comments (user_id, description, path) VALUES ( 31, md5(random()::text), '0001.0003.0002.0002.0003');
INSERT INTO comments (user_id, description, path) VALUES ( 22, md5(random()::text), '0001.0003.0002.0002.0004');
INSERT INTO comments (user_id, description, path) VALUES ( 34, md5(random()::text), '0001.0003.0002.0002.0005');
INSERT INTO comments (user_id, description, path) VALUES ( 22, md5(random()::text), '0001.0003.0002.0002.0006');
\end{lstlisting}

Не забываем добавить индексы:

\begin{lstlisting}[language=SQL,label=lst:pgltree3,caption=Ltree]
# CREATE INDEX path_gist_comments_idx ON comments USING GIST(path);
# CREATE INDEX path_comments_idx ON comments USING btree(path);
\end{lstlisting}

В данном примере я создаю таблицу \lstinline!comments! с полем \lstinline!path!, которые и будет содержать полный путь к этому коментарию в дереве (я использую 4 цифры и точку для делителя узлов дерева). Для начала найдем все коментарии, у который путь начинается с \lstinline!0001.0003!:

\begin{lstlisting}[language=SQL,label=lst:pgltree4,caption=Ltree]
# SELECT user_id, path FROM comments WHERE path <@ '0001.0003';
 user_id |           path
---------+--------------------------
       6 | 0001.0003
       8 | 0001.0003.0001
       9 | 0001.0003.0002
      11 | 0001.0003.0002.0001
       2 | 0001.0003.0002.0002
       5 | 0001.0003.0002.0003
       7 | 0001.0003.0002.0002.0001
      20 | 0001.0003.0002.0002.0002
      31 | 0001.0003.0002.0002.0003
      22 | 0001.0003.0002.0002.0004
      34 | 0001.0003.0002.0002.0005
      22 | 0001.0003.0002.0002.0006
(12 rows)
\end{lstlisting}

И проверим как работают индексы:

\begin{lstlisting}[language=SQL,label=lst:pgltree5,caption=Ltree]
# SET enable_seqscan=false;
SET
# EXPLAIN ANALYZE SELECT user_id, path FROM comments WHERE path <@ '0001.0003';
                                                            QUERY PLAN
-----------------------------------------------------------------------------------------------------------------------------------
 Index Scan using path_gist_comments_idx on comments  (cost=0.00..8.29 rows=2 width=38) (actual time=0.023..0.034 rows=12 loops=1)
   Index Cond: (path <@ '0001.0003'::ltree)
 Total runtime: 0.076 ms
(3 rows)
\end{lstlisting}

Данную выборку можно сделать другим запросом:

\begin{lstlisting}[language=SQL,label=lst:pgltree6,caption=Ltree]
# SELECT user_id, path FROM comments WHERE path ~ '0001.0003.*';
user_id |           path
---------+--------------------------
       6 | 0001.0003
       8 | 0001.0003.0001
       9 | 0001.0003.0002
      11 | 0001.0003.0002.0001
       2 | 0001.0003.0002.0002
       5 | 0001.0003.0002.0003
       7 | 0001.0003.0002.0002.0001
      20 | 0001.0003.0002.0002.0002
      31 | 0001.0003.0002.0002.0003
      22 | 0001.0003.0002.0002.0004
      34 | 0001.0003.0002.0002.0005
      22 | 0001.0003.0002.0002.0006
(12 rows)
\end{lstlisting}

Не забываем про сортировку дерева:

\begin{lstlisting}[language=SQL,label=lst:pgltree7,caption=Ltree]
# INSERT INTO comments (user_id, description, path) VALUES ( 9, md5(random()::text), '0001.0003.0001.0001');
# INSERT INTO comments (user_id, description, path) VALUES ( 9, md5(random()::text), '0001.0003.0001.0002');
# INSERT INTO comments (user_id, description, path) VALUES ( 9, md5(random()::text), '0001.0003.0001.0003');
# SELECT user_id, path FROM comments WHERE path ~ '0001.0003.*';
user_id |           path
---------+--------------------------
       6 | 0001.0003
       8 | 0001.0003.0001
       9 | 0001.0003.0002
      11 | 0001.0003.0002.0001
       2 | 0001.0003.0002.0002
       5 | 0001.0003.0002.0003
       7 | 0001.0003.0002.0002.0001
      20 | 0001.0003.0002.0002.0002
      31 | 0001.0003.0002.0002.0003
      22 | 0001.0003.0002.0002.0004
      34 | 0001.0003.0002.0002.0005
      22 | 0001.0003.0002.0002.0006
       9 | 0001.0003.0001.0001
       9 | 0001.0003.0001.0002
       9 | 0001.0003.0001.0003
(15 rows)
# SELECT user_id, path FROM comments WHERE path ~ '0001.0003.*' ORDER by path;
 user_id |           path
---------+--------------------------
       6 | 0001.0003
       8 | 0001.0003.0001
       9 | 0001.0003.0001.0001
       9 | 0001.0003.0001.0002
       9 | 0001.0003.0001.0003
       9 | 0001.0003.0002
      11 | 0001.0003.0002.0001
       2 | 0001.0003.0002.0002
       7 | 0001.0003.0002.0002.0001
      20 | 0001.0003.0002.0002.0002
      31 | 0001.0003.0002.0002.0003
      22 | 0001.0003.0002.0002.0004
      34 | 0001.0003.0002.0002.0005
      22 | 0001.0003.0002.0002.0006
       5 | 0001.0003.0002.0003
(15 rows)
\end{lstlisting}

Для поиска можно использовать разные модификаторы. Пример использования <<или>> (\lstinline!|!):

\begin{lstlisting}[language=SQL,label=lst:pgltree8,caption=Ltree]
# SELECT user_id, path FROM comments WHERE path ~ '0001.*{1,2}.0001|0002.*' ORDER by path;
 user_id |           path
---------+--------------------------
       2 | 0001.0001.0001
       2 | 0001.0001.0001.0001
       1 | 0001.0001.0001.0002
       5 | 0001.0001.0001.0003
       6 | 0001.0002.0001
       8 | 0001.0003.0001
       9 | 0001.0003.0001.0001
       9 | 0001.0003.0001.0002
       9 | 0001.0003.0001.0003
       9 | 0001.0003.0002
      11 | 0001.0003.0002.0001
       2 | 0001.0003.0002.0002
       7 | 0001.0003.0002.0002.0001
      20 | 0001.0003.0002.0002.0002
      31 | 0001.0003.0002.0002.0003
      22 | 0001.0003.0002.0002.0004
      34 | 0001.0003.0002.0002.0005
      22 | 0001.0003.0002.0002.0006
       5 | 0001.0003.0002.0003
(19 rows)
\end{lstlisting}

Например, найдем прямых потомков от \lstinline!0001.0003!:

\begin{lstlisting}[language=SQL,label=lst:pgltree9,caption=Ltree]
# SELECT user_id, path FROM comments WHERE path ~ '0001.0003.*{1}' ORDER by path;
 user_id |      path
---------+----------------
       8 | 0001.0003.0001
       9 | 0001.0003.0002
(2 rows)
\end{lstlisting}

Можно также найти родителя для потомка <<0001.0003.0002.0002.0005>>:

\begin{lstlisting}[language=SQL,label=lst:pgltree10,caption=Ltree]
# SELECT user_id, path FROM comments WHERE path = subpath('0001.0003.0002.0002.0005', 0, -1) ORDER by path;
 user_id |        path
---------+---------------------
       2 | 0001.0003.0002.0002
(1 row)
\end{lstlisting}


\subsection{Заключение}

Ltree~--- расширение, которое позволяет хранить и удобно управлять Materialized Path в PostgreSQL.
