\section{ZomboDB}

\href{https://github.com/zombodb/zombodb}{ZomboDB}~--- PostgreSQL расширение, которое позволяет использовать \href{https://en.wikipedia.org/wiki/Elasticsearch}{Elasticsearch} индексы внутри базы (используется \href{https://www.postgresql.org/docs/current/static/indexam.html}{интерфейс для методов доступа индекса}). ZomboDB индекс для PostgreSQL ничем не отличается от стандартного btree индекса. Таким образом, стандартные команды SQL полностью поддерживаются, включая \lstinline!SELECT!, \lstinline!BEGIN!, \lstinline!COMMIT!, \lstinline!ABORT!, \lstinline!INSERT!, \lstinline!UPDATE!, \lstinline!DELETE!, \lstinline!COPY! и \lstinline!VACUUM! и данные индексы являются MVCC-безопасными.

На низком уровне ZomboDB индексы взаимодействуют с Elasticsearch сервером через HTTP запросы и автоматически синхронизируются в процессе изменения данных в PostgreSQL базе.

\subsection{Установка и использование}

ZomboDB состоит из двух частей: PostgreSQL расширения (написан на C и SQL/PLPGSQL) и Elasticsearch плагина (написан на Java).

После установки требуется добавить в \lstinline!postgresql.conf! zombodb библиотеку:

\begin{lstlisting}[language=Bash,label=lst:zombodb1,caption=Zombodb]
local_preload_libraries = 'zombodb.so'
\end{lstlisting}

И после перегрузки PostgreSQL активировать его для базы данных:

\begin{lstlisting}[language=SQL,label=lst:zombodb2,caption=Zombodb]
CREATE EXTENSION zombodb;
\end{lstlisting}

После этого требуется установить Elasticsearch плагин на все ноды сервера и изменить конфигурацию в \lstinline!elasticsearch.yml!:

\begin{lstlisting}[language=Bash,label=lst:zombodb3,caption=Elasticsearch]
threadpool.bulk.queue_size: 1024
threadpool.bulk.size: 12

http.compression: true

http.max_content_length: 1024mb
index.query.bool.max_clause_count: 1000000
\end{lstlisting}

Для примера создадим таблицу с продуктами и заполним её данными:

\begin{lstlisting}[language=SQL,label=lst:zombodb4,caption=Products table]
# CREATE TABLE products (
    id SERIAL8 NOT NULL PRIMARY KEY,
    name text NOT NULL,
    keywords varchar(64)[],
    short_summary phrase,
    long_description fulltext,
    price bigint,
    inventory_count integer,
    discontinued boolean default false,
    availability_date date
);

# COPY products FROM PROGRAM 'curl https://raw.githubusercontent.com/zombodb/zombodb/master/TUTORIAL-data.dmp';
\end{lstlisting}

\lstinline!zdb(record)! zombodb функция конвертирует запись в JSON формат (обертка поверх \lstinline!row_to_json(record)!):

\begin{lstlisting}[language=SQL,label=lst:zombodb5,caption=Zdb]
# SELECT zdb(products) FROM products WHERE id = 1;
                    zdb
-------------------------------------------------
{"id":1,"name":"Magical Widget","keywords":["magical","widget","round"],"short_summary":"A widget that is quite magical","long_description":"Magical Widgets come from the land of Magicville and are capable of things you can't imagine","price":9900,"inventory_count":42,"discontinued":false,"availability_date":"2015-08-31"}
\end{lstlisting}

\lstinline!zdb(regclass, tid)! zombodb функция, которая используется для статического определения ссылок на таблицу/индекс в контексте последовательного сканирования. Благодаря этим двум функциям можно создать zombodb индекс для \lstinline!products! таблицы:

\begin{lstlisting}[language=SQL,label=lst:zombodb6,caption=Zdb]
# CREATE INDEX idx_zdb_products
           ON products
        USING zombodb(zdb('products', products.ctid), zdb(products))
         WITH (url='http://localhost:9200/');
\end{lstlisting}

Теперь можно проверить работу индекса:

\begin{lstlisting}[language=SQL,label=lst:zombodb7,caption=Index usage]
# SELECT id, name, short_summary FROM products WHERE zdb('products', products.ctid) ==> 'sports or box';
 id |   name   |         short_summary
----+----------+--------------------------------
  2 | Baseball | It's a baseball
  4 | Box      | Just an empty box made of wood
(2 rows)
# EXPLAIN SELECT * FROM products WHERE zdb('products', ctid) ==> 'sports or box';
                                    QUERY PLAN
-----------------------------------------------------------------------------------
 Index Scan using idx_zdb_products on products  (cost=0.00..4.02 rows=2 width=153)
   Index Cond: (zdb('products'::regclass, ctid) ==> 'sports or box'::text)
(2 rows)
\end{lstlisting}

ZomboDB содержит набор функций для агрегационных запросов. Например, если нужно выбрать уникальный набор ключевых слов для всех продуктов в \lstinline!keywords! поле вместе с их количеством, то можно воспользоваться \lstinline!zdb_tally! функцией:

\begin{lstlisting}[language=SQL,label=lst:zombodb8,caption=Zdb\_tally]
# SELECT * FROM zdb_tally('products', 'keywords', '^.*', '', 5000, 'term');
         term          | count
-----------------------+-------
 alexander graham bell |     1
 baseball              |     1
 box                   |     1
 communication         |     1
 magical               |     1
 negative space        |     1
 primitive             |     1
 round                 |     2
 sports                |     1
 square                |     1
 widget                |     1
 wooden                |     1
(12 rows)
\end{lstlisting}

Более подробно про использование ZomboDB возможно ознакомиться в \href{https://github.com/zombodb/zombodb#quick-links}{официальной документации}.
