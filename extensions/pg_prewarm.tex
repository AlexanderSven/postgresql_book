\section{Pg\_prewarm}

Модуль \href{https://www.postgresql.org/docs/current/static/pgprewarm.html}{pg\_prewarm} обеспечивает удобный способ загрузки данных обьектов (таблиц, индексов, прочего) в буферный кэш PostgreSQL или операционной системы. Данный модуль добавлен в contrib начиная с PostgreSQL 9.4.


\subsection{Установка и использование}

Для начала нужно установить модуль:

\begin{lstlisting}[language=SQL,label=lst:pgprewarm1]
$ CREATE EXTENSION pg_prewarm;
\end{lstlisting}

После уставновки доступна функция \lstinline!pg_prewarm!:

\begin{lstlisting}[language=SQL,label=lst:pgprewarm2]
$ SELECT pg_prewarm('pgbench_accounts');
 pg_prewarm
------------
       4082
(1 row)
\end{lstlisting}

Первый аргумент~--- объект, который требуется предварительно загружать в память. Второй аргумент~--- <<режим>> загрузки в память, который может содержать такие варианты:

\begin{itemize}
  \item \lstinline!prefetch!~--- чтение файла ядром системы в асинхронном режиме (в фоновом режиме). Это позволяет положить содержимое файла в кэше ядра системы. Но этот режим не работает на всех платформах;
  \item \lstinline!read!~--- результат похож на \lstinline!prefetch!, но делается синхронно (а значит медленнее). Работает на всех платформах;
  \item \lstinline!buffer!~--- в этом режиме данные будут грузится в PostgreSQL \lstinline!shared_buffers!. Этот режим используется по умолчанию;
\end{itemize}

Третий аргумент называется <<fork>>. Про него не нужно беспокоиться. Возможные значения: <<main>> (используется по умолчанию), <<fsm>>, <<vm>>.

Четвертый и пятый аргументы указывают диапазон страниц для загрузки данных. По умолчанию загружается весь обьект в память, но можно решить, например, загрузить только последние 1000 страниц:

\begin{lstlisting}[language=SQL,label=lst:pgprewarm3]
$ SELECT pg_prewarm(
    'pgbench_accounts',
    first_block := (
        SELECT pg_relation_size('pgbench_accounts') / current_setting('block_size')::int4 - 1000
    )
);
\end{lstlisting}


\subsection{Заключение}

Pg\_prewarm~--- расширение, которое позволяет предварительно загрузить (<<подогреть>>) данные в буферной кэш PostgreSQL или операционной системы.
