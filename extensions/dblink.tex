\section{Dblink}

\href{https://www.postgresql.org/docs/current/static/dblink.html}{Dblink}~-- расширение, которое позволяет выполнять запросы к удаленным базам данных непосредственно из SQL, не прибегая к помощи внешних скриптов.

\subsection{Установка и использование}

Для начала инициализируем расширение в базе данных:

\begin{lstlisting}[language=SQL,label=lst:dblink1,caption=Инициализация dblink]
# CREATE EXTENSION dblink;
\end{lstlisting}

Для создания подключения к другой базе данных нужно использовать \lstinline!dblink_connect! функцию, где первым параметром указывается имя подключения, а вторым - опции подключения к базе:

\begin{lstlisting}[language=SQL,label=lst:dblink2,caption=Подключение через dblink]
# SELECT dblink_connect('slave_db', 'host=slave.example.com port=5432 dbname=exampledb user=admin password=password');
 dblink_connect
----------------
 OK
(1 row)
\end{lstlisting}

При успешном выполнении команды будет выведен ответ <<OK>>. Теперь можно попробовать считать данные из таблиц через \lstinline!dblink! функцию:

\begin{lstlisting}[language=SQL,label=lst:dblink3,caption=SELECT]
# SELECT *
FROM dblink('slave_db', 'SELECT id, username FROM users LIMIT 3')
AS dblink_users(id integer, username text);

 id |             username
----+----------------------------------
  1 | 64ec7083d7facb7c5d97684e7f415b65
  2 | 404c3b639a920b5ba814fc01353368f2
  3 | 153041f992e3eab6891f0e8da9d11f23
(3 rows)
\end{lstlisting}

По завершению работы с сервером, подключение требуется закрыть через функцию \lstinline!dblink_disconnect!:

\begin{lstlisting}[language=SQL,label=lst:dblinkclose1,caption=dblink\_disconnect]
# SELECT dblink_disconnect('slave_db');
 dblink_disconnect
-------------------
 OK
(1 row)
\end{lstlisting}


\subsection{Курсоры}

Dblink поддерживает \href{https://www.postgresql.org/docs/current/static/plpgsql-cursors.html}{курсоры}~--- инкапсулирующие запросы, которые позволяют получать результат запроса по нескольку строк за раз. Одна из причин использования курсоров заключается в том, чтобы избежать переполнения памяти, когда результат содержит большое количество строк.

Для открытия курсора используется функция \lstinline!dblink_open!, где первый параметр - название подключения, второй - название для курсора, а третий - сам запрос:

\begin{lstlisting}[language=SQL,label=lst:dblink4,caption=dblink\_open]
# SELECT dblink_open('slave_db', 'users', 'SELECT id, username FROM users');
 dblink_open
-------------
 OK
(1 row)
\end{lstlisting}

Для получения данных из курсора требуется использовать \lstinline!dblink_fetch!, где первый параметр - название подключения, второй - название для курсора, а третий - требуемое количество записей из курсора:

\begin{lstlisting}[language=SQL,label=lst:dblink5,caption=dblink\_fetch]
# SELECT id, username FROM dblink_fetch('slave_db', 'users', 2)
AS (id integer, username text);
 id |             username
----+----------------------------------
  1 | 64ec7083d7facb7c5d97684e7f415b65
  2 | 404c3b639a920b5ba814fc01353368f2
(2 rows)

# SELECT id, username FROM dblink_fetch('slave_db', 'users', 2)
AS (id integer, username text);
 id |             username
----+----------------------------------
  3 | 153041f992e3eab6891f0e8da9d11f23
  4 | 318c33458b4840f90d87ee4ea8737515
(2 rows)

# SELECT id, username FROM dblink_fetch('slave_db', 'users', 2)
AS (id integer, username text);
 id |             username
----+----------------------------------
  6 | 5b795b0e73b00220843f82c4d0f81f37
  8 | c2673ee986c23f62aaeb669c32261402
(2 rows)
\end{lstlisting}

После работы с курсором его нужно обязательно закрыть через \lstinline!dblink_close! функцию:

\begin{lstlisting}[language=SQL,label=lst:dblink6,caption=dblink\_close]
# SELECT dblink_close('slave_db', 'users');
 dblink_close
--------------
 OK
(1 row)
\end{lstlisting}


\subsection{Асинхронные запросы}

Последним вариантом для выполнения запросов в dblink является асинхронный запрос. При его использовании результаты не будут возвращены до полного выполнения результата запроса. Для создания асинхронного запроса используется \lstinline!dblink_send_query! функция:

\begin{lstlisting}[language=SQL,label=lst:dblink7,caption=dblink\_send\_query]
# SELECT * FROM dblink_send_query('slave_db', 'SELECT id, username FROM users') AS users;
 users
-------
    1
(1 row)
\end{lstlisting}

Результат получается через \lstinline!dblink_get_result! функцию:

\begin{lstlisting}[language=SQL,label=lst:dblink8,caption=dblink\_get\_result]
# SELECT id, username FROM dblink_get_result('slave_db')
AS (id integer, username text);
 id   |             username
------+----------------------------------
    1 | 64ec7083d7facb7c5d97684e7f415b65
    2 | 404c3b639a920b5ba814fc01353368f2
    3 | 153041f992e3eab6891f0e8da9d11f23
    4 | 318c33458b4840f90d87ee4ea8737515
    6 | 5b795b0e73b00220843f82c4d0f81f37
    8 | c2673ee986c23f62aaeb669c32261402
    9 | c53f14040fef954cd6e73b9aa2e31d0e
   10 | 2dbe27fd96cdb39f01ce115cf3c2a517
\end{lstlisting}
