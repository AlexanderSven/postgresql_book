\section{Pg\_partman}

Поскольку реализация партиционирования реализована не полноценно в PostgreSQL (для управлением партициями и данных в них приходится писать функции, тригеры и правила), то существует расширение, которое автоматизирует полностью данный процесс. \href{https://github.com/keithf4/pg\_partman}{PG Partition Manager}, он же pg\_partman, это расширение для создания и управления партициями и партициями партиций (sub-partitoning) в PostgreSQL. Поддерживает партрицирование по времени (time-based) или по последованности (serial-based). Для партрицированию по диапазону значений (range) существует отдельное расширение \href{https://github.com/moat/range\_partitioning}{Range Partitioning (range\_partitioning)}.

Текущая реализация поддерживается только INSERT операции, которые перенаправляют данные в нужную партицию. UPDATE операции, которые будут перемещать данные из одной партиции в другую не поддерживаются. При попытке вставить данные, на которые нет партиции, pg\_partman перемещает их в <<мастер>> (родительскую) таблицу. Данный вариант предпочтительнее, чем создавать автоматически новые партиции, поскольку это может привести к созданию десятков или сотен не нужных дочерных таблиц из-за ошибки в самих данных. Функция \lstinline!check_parent! позволят проверить попадение подобных данных в родительскую таблицу и решить, что с ними требуется делать (удалить или использовать \lstinline!partition_data_time/partition_data_id! для создания и переноса этих данных в партиции).

Данное расширение использует большинство атрибутов родительской таблици для создания партиций: индексы, внешние ключи (опционально), tablespace, constraints, privileges и ownership. Под такое условие попадают OID и UNLOGGED таблицы.

Партициями партиций (sub-partitoning) поддерживаются разных уровней: time->time, id->id, time->id и id->time. Нет лимитов на создания таких партиций, но стоит помнить что большое число партиций влияет на производительность родительской таблицы. Если размер партиций станет слишком большим, то придется увеличивать \lstinline!max_locks_per_transaction! параметр для базы данных (64 по умолчанию).

В PostgreSQL 9.4 появилась возможность создания пользовательских фоновых воркеров и динамически загружать их во время работы базы. Благодаря этому в pg\_partman есть собственный фоновый воркер, задача которого запускать \lstinline!run_maintenance! функцию каждый заданный промежуток времени. Если у Вас версия PostgreSQL ниже 9.4, то придется воспользоватся внешним планировщиком для выполнения данной функции (например cron). Задача данной функции проверять и автоматически создавать партиции и опционально чистить старые.

\subsection{Пример использования}

Для начала установим данное расширение:

\begin{lstlisting}[language=Bash,label=lst:pgpartman1,caption=Установка]
$ git clone https://github.com/keithf4/pg_partman.git
$ cd pg_partman/
$ make
$ sudo make install
\end{lstlisting}

Если не требуется использовать фоновый воркер, то можно собрать без него:

\begin{lstlisting}[language=Bash,label=lst:pgpartman2,caption=Установка]
$ sudo make NO_BGW=1 install
\end{lstlisting}

Для работы фонового воркера нужно загружать его на старте PostgreSQL. Для этого потребуется добавить настройки в postgresql.conf:

\begin{lstlisting}[language=Bash,label=lst:pgpartman3,caption=Настройки воркера]
shared_preload_libraries = 'pg_partman_bgw'     # (change requires restart)
pg_partman_bgw.interval = 3600
pg_partman_bgw.role = 'myrole'
pg_partman_bgw.dbname = 'mydatabase'
\end{lstlisting}

где:

\begin{itemize}
  \item \lstinline!pg_partman_bgw.dbname!~--- база данных, в которой будет выполняться \lstinline!run_maintenance! функция. Если нужно указать больше одной базы, то они указываются через запятую. Без этого параметра воркер не будет работать;
  \item \lstinline!pg_partman_bgw.interval!~--- количество секунд между вызовами \lstinline!run_maintenance! функции. По умолчанию 3600 (1 час);
  \item \lstinline!pg_partman_bgw.role!~--- роль для запуска \lstinline!run_maintenance! функции. По умолчанию postgres. Разрешена только одна роль;
  \item \lstinline!pg_partman_bgw.analyze!~--- запускать или нет \lstinline!ANALYZE! после создания партиций на родительскую таблицу. По умолчанию включено;
  \item \lstinline!pg_partman_bgw.jobmon!~--- разрешить или нет использовать \lstinline!pg_jobmon! расширение для мониторинга, что партрицирование работает без проблем. По умолчанию включено;
\end{itemize}

Далее подключаемся к базе данных и активируем расширение:

\begin{lstlisting}[language=SQL,label=lst:pgpartman4,caption=Настройка расширения]
# CREATE SCHEMA partman;
CREATE SCHEMA
# CREATE EXTENSION pg_partman SCHEMA partman;
CREATE EXTENSION
\end{lstlisting}

Теперь можно приступать к использованию расширения. Создадим и заполним таблицу тестовыми данными:

\begin{lstlisting}[language=SQL,label=lst:pgpartman5,caption=Данные]
# CREATE TABLE users (
    id             serial primary key,
    username       text not null unique,
    password       text,
    created_on     timestamptz not null,
    last_logged_on timestamptz not null
);

# INSERT INTO users (username, password, created_on, last_logged_on)
  SELECT
      md5(random()::text),
      md5(random()::text),
      now() - '1 years'::interval * random(),
      now() - '1 years'::interval * random()
  FROM
      generate_series(1, 10000);
\end{lstlisting}

Далее активируем расширение для поля \lstinline!created_on! с партицией на каждый год:

\begin{lstlisting}[language=SQL,label=lst:pgpartman6,caption=Партицирование]
# SELECT partman.create_parent('public.users', 'created_on', 'time', 'yearly');
 create_parent
---------------
 t
(1 row)
\end{lstlisting}

Указывание схемы в имени таблици обязательно, даже если она <<public>> (первый аргумент функции).

Поскольку родительская таблица уже была заполнена данными, перенесем данные из нее в партиции через \lstinline!partition_data_time! функцию:

\begin{lstlisting}[language=SQL,label=lst:pgpartman7,caption=Перенос данных в партиции]
# SELECT partman.check_parent();
     check_parent
----------------------
 (public.users,10000)
(1 row)

# SELECT partman.partition_data_time('public.users', 1000);
 partition_data_time
---------------------
               10000
(1 row)

# SELECT partman.check_parent();
 check_parent
--------------
(0 rows)

# SELECT * FROM ONLY users;
 id | username | password | created_on | last_logged_on
----+----------+----------+------------+----------------
(0 rows)

# \d+ users
                                                          Table "public.users"
     Column     |           Type           |                     Modifiers                      | Storage  | Stats target | Description
----------------+--------------------------+----------------------------------------------------+----------+--------------+-------------
 id             | integer                  | not null default nextval('users_id_seq'::regclass) | plain    |              |
 username       | text                     | not null                                           | extended |              |
 password       | text                     |                                                    | extended |              |
 created_on     | timestamp with time zone | not null                                           | plain    |              |
 last_logged_on | timestamp with time zone | not null                                           | plain    |              |
Indexes:
    "users_pkey" PRIMARY KEY, btree (id)
    "users_username_key" UNIQUE CONSTRAINT, btree (username)
Triggers:
    users_part_trig BEFORE INSERT ON users FOR EACH ROW EXECUTE PROCEDURE users_part_trig_func()
Child tables: users_p2012,
              users_p2013,
              users_p2014,
              users_p2015,
              users_p2016,
              users_p2017,
              users_p2018,
              users_p2019,
              users_p2020
\end{lstlisting}


В результате данные в таблице \lstinline!users! содержатся в партициях благодаря pg\_partman. Более подробно по функционалу расширения, его настройках и ограничениях доступно в  \href{https://github.com/keithf4/pg\_partman/blob/master/doc/pg\_partman.md}{официальной документации}.


