\chapter{Пул коннектов}
\section{Введение}
Программы для создания пула коннектов позволяют уменьшить накладные расходы на базу данных, 
когда огромное количество физических соединений тянет производительность PostgreSQL вниз. 
Это особенно важно на Windows, когда система ограничивает большое количество соединений. 
Это также важно для веб-приложений, где количество соединений может быть очень большим.

Программы, которые создают пулы соединений:
\begin{itemize}
\item PgBouncer
\item Pgpool
\end{itemize}

Также некоторые администраторы PostgreSQL с успехом используют Memcached для уменьшения работы БД 
за счет кэширования данных.

\section{PgBouncer}
Это пул коннектов для PostgreSQL от компании Skype. Существуют три режима управления.
\begin{itemize}
\item \textbf{Session Pooling.}
Клиенту выделяется соединение с сервером; оно приписано ему в течение всей сессии и 
возвращается в пул только после отсоединения клиента.
\item \textbf{Transaction Pooling.} 
Клиент владеет соединением только в течение транзакции.
\item \textbf{Statement Pooling.} 
Соединение возвращается назад в пул сразу после завершения запроса.
\end{itemize}

К достоинствам PgBouncer относится:
\begin{itemize}
\item малое потребление памяти (менее 2 КБ на соединение);
\item отсутствие привязки к одному серверу баз данных;
\item реконфигурация настроек без рестарта.
\end{itemize}

Базовая утилита запускается так:
\begin{verbatim}
$pgbouncer [-d][-R][-v][-u user] <pgbouncer.ini>
\end{verbatim}

Простой пример для конфига:
\begin{verbatim}
[databases]
template1 = host=127.0.0.1 port=5432 dbname=template1
[pgbouncer]
listen_port = 6543
listen_addr = 127.0.0.1
auth_type = md5
auth_file = userlist.txt
logfile = pgbouncer.log
pidfile = pgbouncer.pid
admin_users = someuser
\end{verbatim}

Нужно создать файл пользователей userlist.txt примерного содержания:"someuser" "same\_password\_as\_in\_server"

Админский доступ из консоли к базе данных pgbouncer:
\begin{verbatim}
$psql -h 127.0.0.1 -p 6543 pgbouncer
\end{verbatim}

Здесь можно получить различную статистическую информацию с помощью команды SHOW.

