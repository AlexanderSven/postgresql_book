\section{Londiste}
Londiste представляет собой движок для организации репликации, написанный на языке python. 
Основные принципы: надежность и простота использования. Из-за этого данное решение имеет меньше функциональности, 
чем Slony-I. Londiste использует в качестве транспортного механизма очередь PgQ  (описание этого более чем интересного 
проекта остается за рамками данной главы, поскольку он представляет интерес скорее для низкоуровневых программистов 
баз данных, чем для конечных пользователей~--- администраторов СУБД PostgreSQL). Отличительными особенностями решения являются:
\begin{itemize}
\item возможность потабличной репликации
\item начальное копирование ничего не блокирует
\item возможность двухстороннего сравнения таблиц
\item простота установки
\end{itemize}

К недостаткам можно отнести:
\begin{itemize}
\item отсутствие поддержки каскадной репликации, отказоустойчивости(failover) и переключение между 
серверами (switchover) (все это обещают к 3 версии реализовать
\footnote{http://skytools.projects.postgresql.org/skytools-3.0/doc/skytools3.html})
\end{itemize}


\subsection{Установка}
На серверах, которые мы настраиваем расматривается ОС Linux, а именно Ubuntu Server. 
Автор данной книги считает, что под другие операционные системы (кроме Windows) все мало чем будет отличаться, 
а держать кластера PostgreSQL под ОС Windows, по меньшей мере, неразумно.

Поскольку Londiste~--- это часть Skytools, то нам нужно ставить этот пакет. На таких системах, как Debian или Ubuntu skytools 
можно найти в репозитории пакетов и поставить одной командой:
\begin{verbatim}
$sudo aptitude install skytools
\end{verbatim}

Но все же лучше скачать самую последнюю версию пакета с официального сайта~--- http://pgfoundry.org/projects/skytools. 
На момент написания статьи последняя версия была 2.1.11. Итак, начнем:

\begin{verbatim}
$wget http://pgfoundry.org/frs/download.php/2561/
skytools-2.1.11.tar.gz
$tar zxvf skytools-2.1.11.tar.gz
$cd skytools-2.1.11/
# это для сборки deb пакета
$sudo aptitude install build-essential autoconf \ 
automake autotools-dev dh-make \ 
debhelper devscripts fakeroot xutils lintian pbuilder \
python-dev yada
# ставим пакет исходников для postgresql 8.4.x
$sudo aptitude install postgresql-server-dev-8.4
# python-psycopg нужен для работы Londiste
$sudo aptitude install python-psycopg2
# данной командой я собираю deb пакет для 
# postgresql 8.4.x (для 8.3.x например будет "make deb83")
$sudo make deb84
$cd ../
# ставим skytools
$dpkg -i skytools-modules-8.4_2.1.11_i386.deb 
skytools_2.1.11_i386.deb
\end{verbatim}

Для других систем можно собрать Skytools командами 

\begin{verbatim}
$./configure
$make
$make install
\end{verbatim}

Дальше проверим, что все у нас правильно установилось
\begin{verbatim}
$londiste.py -V
Skytools version 2.1.11
$pgqadm.py -V
Skytools version 2.1.11
\end{verbatim}

Если у Вас похожий вывод, значит все установленно правильно и можно приступать к настройке.


\subsection{Настройка}
Обозначения: 
\begin{itemize}
\item host1~--- мастер; 
\item host2~--- слейв;
\end{itemize}