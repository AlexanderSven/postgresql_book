\section{Londiste}
Londiste представляет собой движок для организации репликации, написанный на языке python. 
Основные принципы: надежность и простота использования. Из-за этого данное решение имеет меньше функциональности, 
чем Slony-I. Londiste использует в качестве транспортного механизма очередь PgQ  (описание этого более чем интересного 
проекта остается за рамками данной главы, поскольку он представляет интерес скорее для низкоуровневых программистов 
баз данных, чем для конечных пользователей~--- администраторов СУБД PostgreSQL). Отличительными особенностями решения являются:
\begin{itemize}
\item возможность потабличной репликации
\item начальное копирование ничего не блокирует
\item возможность двухстороннего сравнения таблиц
\item простота установки
\end{itemize}

К недостаткам можно отнести:
\begin{itemize}
\item отсутствие поддержки каскадной репликации, отказоустойчивости(failover) и переключение между 
серверами (switchover) (все это обещают к 3 версии реализовать
\footnote{http://skytools.projects.postgresql.org/skytools-3.0/doc/skytools3.html})
\end{itemize}


\subsection{Установка}
На серверах, которые мы настраиваем расматривается ОС Linux, а именно Ubuntu Server. 
Автор данной книги считает, что под другие операционные системы (кроме Windows) все мало чем будет отличаться, 
а держать кластера PostgreSQL под ОС Windows, по меньшей мере, неразумно.

Поскольку Londiste~--- это часть Skytools, то нам нужно ставить этот пакет. На таких системах, как Debian или Ubuntu skytools 
можно найти в репозитории пакетов и поставить одной командой:
\begin{verbatim}
$sudo aptitude install skytools
\end{verbatim}

Но все же лучше скачать самую последнюю версию пакета с официального сайта~--- http://pgfoundry.org/projects/skytools. 
На момент написания статьи последняя версия была 2.1.11. Итак, начнем:

\begin{verbatim}
$wget http://pgfoundry.org/frs/download.php/2561/
skytools-2.1.11.tar.gz
$tar zxvf skytools-2.1.11.tar.gz
$cd skytools-2.1.11/
# это для сборки deb пакета
$sudo aptitude install build-essential autoconf \ 
automake autotools-dev dh-make \ 
debhelper devscripts fakeroot xutils lintian pbuilder \
python-dev yada
# ставим пакет исходников для postgresql 8.4.x
$sudo aptitude install postgresql-server-dev-8.4
# python-psycopg нужен для работы Londiste
$sudo aptitude install python-psycopg2
# данной командой я собираю deb пакет для 
# postgresql 8.4.x (для 8.3.x например будет "make deb83")
$sudo make deb84
$cd ../
# ставим skytools
$dpkg -i skytools-modules-8.4_2.1.11_i386.deb 
skytools_2.1.11_i386.deb
\end{verbatim}

Для других систем можно собрать Skytools командами 

\begin{verbatim}
$./configure
$make
$make install
\end{verbatim}

Дальше проверим, что все у нас правильно установилось
\begin{verbatim}
$londiste.py -V
Skytools version 2.1.11
$pgqadm.py -V
Skytools version 2.1.11
\end{verbatim}

Если у Вас похожий вывод, значит все установленно правильно и можно приступать к настройке.


\subsection{Настройка}
Обозначения: 
\begin{itemize}
\item host1~--- мастер; 
\item host2~--- слейв;
\end{itemize}


\subsubsection{Настройка ticker-а}
Londiste требуется ticker для работы с мастер базой данных, который может быть запущен и на другой машине. 
Но, конечно, лучше его запускать на той же, где и мастер база данных. Для этого мы настраиваем специальный 
конфиг для ticker-а (пусть конфиг будет у нас /etc/skytools/db1-ticker.ini):
\begin{verbatim}
[pgqadm]
# название
job_name = db1-ticker

# мастер база данных 
db = dbname=P host=host1 

# Задержка между запусками обслуживания 
# (ротация очередей и т.п.) в секундах
maint_delay = 600

# Задержка между проверками наличия активности 
# (новых пакетов данных) в секундах
loop_delay = 0.1

# log и pid демона
logfile = /var/log/%(job_name)s.log
pidfile = /var/pid/%(job_name)s.pid
\end{verbatim}

Теперь необходимо инсталлировать служебный код (SQL) и запустить ticker как демона для базы данных. 
Делается это с помощью утилиты pgqadm.py следующими командами:
\begin{verbatim}
pgqadm.py /etc/skytools/db1-ticker.ini install
pgqadm.py /etc/skytools/db1-ticker.ini ticker -d
\end{verbatim}

Проверим, что в логах (/var/log/skytools/db1-tickers.log) всё нормально. На данном этапе там должны быть редкие записи (раз в минуту).

Если нам потребуется остановить ticker, мы можем воспользоватся этой командой:
\begin{verbatim}
pgqadm.py /etc/skytools/db1-ticker.ini ticker -s
\end{verbatim}
или если потребуется <<убить>> ticker:
\begin{verbatim}
pgqadm.py /etc/skytools/db1-ticker.ini ticker -k
\end{verbatim}

\subsubsection{Востанавливаем схему базы}
Londiste не умеет переносить изменения структуры базы данных. 
Поэтому на всех slave базах данных перед репликацией должна быть создана такая же структура БД, что и на мастере.

\subsubsection{Создаём конфигурацию репликатора}
Для каждой из реплицируемых баз создадим конфигурационные файлы 
(пусть конфиг будет у нас /etc/skytools/db1-londiste.ini):
\begin{verbatim}
[londiste]
# название
job_name = db1-londiste

# мастер база данных
provider_db = dbname=db1 port=5432 host=host1
# слейв база данных
subscriber_db = dbname=db1 host=host2

# Это будет использоваться в качестве 
# SQL-идентификатора, т.ч. не используйте
# точки и пробелы.
# ВАЖНО! Если есть живая репликация на другой слейв, 
# именуем очередь так-же
pgq_queue_name = db1-londiste-queue

# log и pid демона
logfile = /var/log/%(job_name)s.log
pidfile = /var/run/%(job_name)s.pid

# рзмер лога
log_size = 5242880
log_count = 3
\end{verbatim}

\subsubsection{Устанавливаем Londiste в базы на мастере и слейве}
Теперь необходимо установить служебный SQL для каждой из созданных в предыдущем
пункте конфигураций.

Устанавливаем код на стороне мастера:
\begin{verbatim}
londiste.py /etc/skytools/db1-londiste.ini provider install
\end{verbatim}
и подобным образом на стороне слейва:
\begin{verbatim}
londiste.py /etc/skytools/db1-londiste.ini subscriber install
\end{verbatim}

После этого пункта на мастере будут созданы очереди для репликации.

\subsubsection{Запускаем процессы Londiste}
Для каждой реплицируемой базы делаем:
\begin{verbatim}
londiste.py /etc/skytools/db1-londiste.ini replay -d
\end{verbatim}

Таким образом запустятся слушатели очередей репликации, но, т.к. мы ещё не
указывали какие таблицы хотим реплицировать, они пока будут работать в холостую.

Убедимся что в логах нет ошибок (/var/log/db1-londistes.log).

\subsubsection{Добавляем реплицируемые таблицы}
Для каждой конфигурации указываем что будем реплицировать с мастера:
\begin{verbatim}
londiste.py /etc/skytools/db1-londiste.ini provider add --all
\end{verbatim}
и что со слейва:
\begin{verbatim}
londiste.py /etc/skytools/db1-londiste.ini subscriber add --all
\end{verbatim}

В данном примере я использую спец-параметр <<--all>>, который означает все таблицы,
но вместо него вы можете перечислить список конкретных таблиц, если не хотите
реплицировать все.

\subsubsection{Добавляем реплицируемые последовательности (sequence)}
Так же для всех конфигураций.
Для мастера:
\begin{verbatim}
londiste.py /etc/skytools/db1-londiste.ini provider add-seq --all
\end{verbatim}
Для слейва:
\begin{verbatim}
londiste.py /etc/skytools/db1-londiste.ini subscriber add-seq --all
\end{verbatim}

Точно также как и с таблицами можно указать конкретные последовательности вместо <<--all>>.

\subsubsection{Проверка}
Итак, всё что надо сделано. Теперь Londiste запустит так называемый bulk copy
процесс, который массово (с помощью COPY) зальёт присутствующие на момент
добавления таблиц данные на слейв, а затем перейдёт в состояние обычной репликации.

Мониторим логи на предмет ошибок:
\begin{verbatim}
less /var/log/db1-londiste.log
\end{verbatim}

Если всё хорошо, смотрим состояние репликации. Данные уже синхронизированы для
тех таблиц, где статус отображается как "ok".
\begin{verbatim}
londiste.py /etc/skytools/db1-londiste.ini subscriber tables

   Table State
   public.table1 ok
   public.table2 ok
   public.table3 in-copy
   public.table4 -
   public.table5 -
   public.table6 -
   ...
\end{verbatim}

Для удобства представляю следующий трюк с уведомление в почту об окончании
первоначального копирования (мыло поменять на своё):
\begin{verbatim}
(
while [ $(
python londiste.py /etc/skytools/db1-londiste.ini subscriber tables |
tail -n+2 | awk '{print $2}' | grep -v ok | wc -l) -ne 0 ];
do sleep 60; done; echo '' | mail -s 'Replication done EOM' user@domain.com
) &
\end{verbatim}

\subsubsection{Проверка состояния слейвов}
Этот запрос на мастере дает некоторую информацию о каждой очереди и слейве.
\begin{verbatim}
SELECT queue_name, consumer_name, lag, last_seen
  FROM pgq.get_consumer_info();
\end{verbatim}

<<lag>> столбец показывает отставание от мастера в синхронизации, 
<<last\_seen>>~--- время последней запроса от слейва. Значение этого столбца не должно быть больше, 
чем 60 секунд для конфигурации по умолчанию.