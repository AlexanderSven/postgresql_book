\section{Londiste}
Londiste представляет собой движок для организации репликации, написанный на языке python. 
Основные принципы: надежность и простота использования. Из-за этого данное решение имеет меньше функциональности, 
чем Slony-I. Londiste использует в качестве транспортного механизма очередь PgQ  (описание этого более чем интересного 
проекта остается за рамками данной главы, поскольку он представляет интерес скорее для низкоуровневых программистов 
баз данных, чем для конечных пользователей~--- администраторов СУБД PostgreSQL). Отличительными особенностями решения являются:
\begin{itemize}
\item возможность потабличной репликации
\item начальное копирование ничего не блокирует
\item возможность двухстороннего сравнения таблиц
\item простота установки
\end{itemize}

К недостаткам можно отнести:
\begin{itemize}
\item отсутствие поддержки каскадной репликации, отказоустойчивости(failover) и переключение между 
серверами (switchover) (все это обещают к 3 версии реализовать
\footnote{http://skytools.projects.postgresql.org/skytools-3.0/doc/skytools3.html})
\end{itemize}

\subsection{Перед началом}
На серверах, которые мы настраиваем расматривается ОС Linux. 
Автор данной книги считает, что под другие операционные системы (кроме Windows) все мало чем будет отличаться, 
а держать кластера PostgreSQL под ОС Windows, по меньшей мере, неразумно.
Обозначения: 
\begin{itemize}
\item host1~--- мастер; 
\item host2~--- слейв;
\end{itemize}

\subsection{Установка}


