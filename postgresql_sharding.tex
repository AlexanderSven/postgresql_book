\chapter{Шардинг БД}
\section{Введение}
Шардинг~--- разделение данных на уровне ресурсов. Концепция шардинга заключается в логическом разделении данных по различным 
ресурсам исходя из требований к нагрузке.

Рассмотрим пример. Пусть у нас есть приложение с регистрацией пользователей, которое позволяет писать друг другу 
личные сообщения. Допустим оно очень популярно и много людей им пользуются ежедневно. Естественно, что таблица с личными 
сообщениями будет намного больше всех остальных таблиц в базе (скажем, будет занимать 90\% всех ресурсов). Зная это, 
мы можем подготовить для этой (только одной!) таблицы выделенный сервер помощнее, а остальные оставить на другом (послабее). 
Теперь мы можем идеально подстроить сервер для работы с одной специфической таблицей, постараться уместить ее в память, возможно, 
дополнительно партиционировать ее и т.д. Такое распределение называется вертикальным шардингом.

Что делать, если наша таблица с сообщениями стала настолько большой, что даже выделенный сервер под нее одну уже не спасает. 
Необходимо делать горизонтальный шардинг~--- т.е. разделение одной таблицы по разным ресурсам. Как это выглядит на практике? 
Все просто. На разных серверах у нас будет таблица с одинаковой структурой, но разными данными. Для нашего случая с сообщениями, 
мы можем хранить первые 10 миллионов сообщений на одном сервере, вторые 10 - на втором и т.д. Т.е. необходимо иметь критерий 
шардинга~--- какой-то параметр, который позволит определять, на каком именно сервере лежат те или иные данные.

Обычно, в качестве параметра шардинга выбирают ID пользователя (user\_id)~--- это позволяет делить данные по серверам равномерно 
и просто. Т.о. при получении личных сообщений пользователей алгоритм работы будет такой:
\begin{itemize}
\item Определить, на каком сервере БД лежат сообщения пользователя исходя из user\_id
\item Инициализировать соединение с этим сервером
\item Выбрать сообщения
\end{itemize}

Задачу определения конкретного сервера можно решать двумя путями:
\begin{itemize}
\item Хранить в одном месте хеш-таблицу с соответствиями <<пользователь=сервер>>. Тогда, при определении сервера, нужно будет 
выбрать сервер из этой таблицы. В этом случае узкое место~--- это большая таблица соответсвия, которую нужно хранить в одном месте. 
Для таких целей очень хорошо подходят базы данных <<ключ=значение>>
\item Определять имя сервера с помощью числового (буквенного) преобразования. Например, можно вычислять номер сервера, 
как остаток от деления на определенное число (количество серверов, между которыми Вы делите таблицу). В этом случае узкое место~--- 
это проблема добавления новых серверов~--- Вам придется делать перераспределение данных между новым количеством серверов.
\end{itemize}

Для шардинга не существует решения на уровне известных платформ, т.к. это весьма специфическая для отдельно взятого приложения задача.

Естественно, делая горизонтальный шардинг, Вы ограничиваете себя в возможности выборок, которые требуют 
пересмотра всей таблицы (например, последние посты в блогах людей будет достать невозможно, если таблица постов шардится). 
Такие задачи придется решать другими подходами. Например, для описанного примера, можно при появлении нового поста, заносить 
его ID в общий стек, размером в 100 элементом.

Горизонтальный шардинг имеет одно явное преимущество~--- он бесконечно масштабируем.