\section{Проблема чтения данных}

Проблема с чтением данных обычно начинается, когда СУБД не в состоянии обеспечить то количество выборок, которое требуется.
В основном такое происходит в блогах, новостных лентах и т.д. Хочу сразу отметить, что подобную проблему лучше решать
внедрением кеширования, а потом уже думать как масштабировать СУБД.

\subsection{Методы решения}

\begin{itemize}
\item \textbf{PgPool-II v.3 + PostgreSQL v.9 с Streaming Replication}~--- отличное решение для масштабирования на чтение,
более подробно можно ознакомится по \href{http://pgpool.projects.pgfoundry.org/contrib\_docs/simple\_sr\_setting/index.html}{ссылке}.
Основные преимущества:
\begin{itemize}
\item Низкая задержка репликации между мастером и слейвом
\item Производительность записи падает незначительно
\item Отказоустойчивость (failover)
\item Пулы соединений
\item Интеллектуальная балансировка нагрузки~-- проверка задержки репликации между мастером и слейвом (сам проверяет pg\_current\_xlog\_location и pg\_last\_xlog\_receive\_location).
\item Добавление слейвов СУБД без остановки pgpool-II
\item Простота в настройке и обслуживании
\end{itemize}

\item \textbf{PgPool-II v.3 + PostgreSQL с Slony}~--- аналогично предыдущему решению, но с использованием Slony.
Основные преимущества:
\begin{itemize}
\item Отказоустойчивость (failover)
\item Пулы соединений
\item Интеллектуальная балансировка нагрузки~-- проверка задержки репликации между мастером и слейвом.
\item Добавление слейв СУБД без остановки pgpool-II
\item Можно использовать Postgresql ниже 9 версии
\end{itemize}

\item \textbf{Postgres-XC}~--- подробнее можно прочитать в \Sref{sec:postgres-xc} главе.
\end{itemize}
