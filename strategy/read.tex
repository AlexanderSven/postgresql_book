\section{Проблема чтения данных}

Обычно начинается проблема с чтением данных, когда СУБД не в состоянии обеспечить то количество выборок, которое требуется. 
В основном такое происходит в блогах, новостных лентах и т.д. Хочу сразу отметить, что подобную проблему лучше решать 
внедрением кеширования, а потом уже думать как масштабировать СУБД.

\subsection{Методы решения}

\begin{itemize}
\item \textbf{PgPool-II v.3 + Postgresql v.9 с Streaming Replication}~--- отличное решение для масштабирования на чтение, 
более подробно можно ознакомится по ссылке\footnote{http://pgpool.projects.postgresql.org/contrib\_docs/simple\_sr\_setting/index.html}. 
Основные преимущества:
\begin{itemize}
\item Низкая задержка репликации между мастером и слейвом
\item Производительность записи падает незначительно
\item Отказоустойчивость (failover)
\item Пулы соединений
\item Интеллектуальная балансировка нагрузки~-- проверка задержки репликации между мастером и слейвом.
\item Добавление слейв СУБД без остановки pgpool-II
\item Простота в настройке и обслуживании
\end{itemize}
Также можете глянуть презентацию\footnote{http://www.slideshare.net/leopard\_me/postgresql-5845499}.

\item \textbf{PgPool-II v.3 + Postgresql с Slony}~--- аналогично предыдущему решение, но с использованием Slony.
Основные преимущества:
\begin{itemize}
\item Отказоустойчивость (failover)
\item Пулы соединений
\item Интеллектуальная балансировка нагрузки~-- проверка задержки репликации между мастером и слейвом.
\item Добавление слейв СУБД без остановки pgpool-II
\item Можно использовать Postgresql ниже 9 версии
\end{itemize}
\end{itemize}
