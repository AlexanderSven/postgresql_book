\chapter{Расширения}
\begin{epigraphs}
\qitem{Гибкость ума может заменить красоту.}{Стендаль}
\end{epigraphs}

\section{Введение}
Один из главных плюсов PostgreSQL это возможность расширения его функционала с помощью расширений. 
В данной статье я затрону только самые интересные и популярные из существующих расширений. 

\section{PostGIS}
\textbf{Лицензия}: Open Source

\textbf{Ссылка}: \href{http://www.postgis.org/}{www.postgis.org}

PostGIS добавляет поддержку для географических объектов в PostgreSQL. По сути PostGIS позволяет использовать PostgreSQL в качестве 
бэкэнда пространственной базы данных для геоинформационных систем (ГИС), так же, как ESRI SDE или пространственного расширения Oracle. 
PostGIS следует OpenGIS <<Простые особенности Спецификация для SQL>> и был сертифицирован.
\section{Smlar}
\textbf{Лицензия}: Open Source

\textbf{Ссылка}: \href{http://sigaev.ru/git/gitweb.cgi?p=smlar.git;a=blob;hb=HEAD;f=README}{sigaev.ru}

Поиск похожестей в больших базах данных является важным вопросом в настоящее время для таких систем как блоги (похожие статьи), интернет-магазины (похожие продукты), хостинг изображений (похожие изображения, поиск дубликатов изображений) и т.д. PostgreSQL позволяет сделать такой поиск более легким. Прежде всего, необходимо понять, как мы будем вычислять сходство двух объектов.

\subsection{Похожесть}

Любой объект может быть описан как список характеристик. Например, статья в блоге может быть описана тегами, продукт в интернет-магазине может быть описан размером, весом, цветом и т.д. Это означает, что для каждого объекта можно создать цифровую подпись~--- массив чисел, описывающих объект (отпечатки пальцев\footnote{http://en.wikipedia.org/wiki/Fingerprint}, n-grams\footnote{http://en.wikipedia.org/wiki/N-gram}). Тоесть нужно создать массив из цифр для описания каждого объекта. Что делать дальше?

\subsection{Расчет похожести}

Есть несколько методов вычисления похожести сигнатур обьектов. Прежде всего, легенда для расчетов:

$N_a$, $N_b$~-- количество уникальных элементов в массивах

$N_u$~-- количество уникальных элементов при объединении массивов

$N_i$~-- количество уникальных элементов при пересечение массивов

Один из простейших расчетов похожести двух объектов - количество уникальных элементов при пересечение массивов делить на количество уникальных элементов в двух массивах:

\begin{equation}
 \label{eq:smlar1}
 S(A,B) = \frac{N_{i}}{(N_{a}+N_{b})}
\end{equation}

или проще

\begin{equation}
 \label{eq:smlar2}
 S(A,B) = \frac{N_{i}}{N_{u}}
\end{equation}

Преимущества:

\begin{itemize}
\item Легко понять
\item Скорость расчета: $N * \log{N}$
\item Хорошо работает на похожих и больших $N_a$ и $N_b$
\end{itemize}

Также похожесть можно рассчитана по формуле косинусов\footnote{http://en.wikipedia.org/wiki/Law\_of\_cosines}:

\begin{equation}
 \label{eq:smlar3}
 S(A,B) = \frac{N_{i}}{\sqrt{N_{a}*N_{b}}}
\end{equation}

Преимущества:

\begin{itemize}
\item Скорость расчета: $N * \log{N}$
\item Отлично работает на больших $N$
\end{itemize}

Но у обоих этих методов есть общие проблемы:

\begin{itemize}
\item Если элементов мало то разброс похожестей не велик
\item Глобальная статистика: частые элементы ведут к тому, что вес ниже
\item Спамеры и недобросовестные пользователи. Один <<залетевший дятел>> разрушит цивилизацию - алгоритм перестанет работать на Вас.
\end{itemize}

Для избежания этих проблем можно воспользоватся TF/IDF\footnote{http://en.wikipedia.org/wiki/Tf*idf} метрикой:

\begin{equation}
 \label{eq:smlar4}
 S(A,B) = \frac{\sum_{i < N_{a}, j < N_{b}, A_{i} = B_{j}}TF_{i} * TF_{j}}{\sqrt{\sum_{i < N_{a}}TF_{i}^{2} * \sum_{j < N_{b}}TF_{j}^{2}}}
\end{equation}

где инвертированный вес элемента в коллекции:

\begin{equation}
 \label{eq:smlar5}
 IDF_{element} = \log{(\frac{N_{objects}}{N_{objects\ with\ element}} + 1)}
\end{equation}

и вес элемента в массиве:

\begin{equation}
 \label{eq:smlar6}
 TF_{element} = IDF_{element} * N_{occurrences}
\end{equation}

Не пугайтесь! Все эти алгоритмы встроены в smlar расширение, учить (или даже глубоко понимать) их не нужно. Главное понимать, что для TF/IDF метрики требуются вспомогательная таблица для хранения данных, по сравнению с другими простыми метриками.

\subsection{Smlar}

Перейдем к практике. Олег Бартунов и Теодор Сигаев разработали PostgreSQL расширение smlar, которое предоставляет несколько методов для расчета похожестей массивов (все встроенные типы данных поддерживаются) и оператор для расчета похожести с поддержкой индекса на базе GIST и GIN. Для начала установим это расширение (PostgreSQL уже должен быть установлен):

\begin{lstlisting}[label=lst:smlar1,caption=Установка smlar]
git clone git://sigaev.ru/smlar
cd smlar
USE_PGXS=1 make && make install
\end{lstlisting}

В PostgreSQL 9.2 и выше это расширение должно встать без проблем, для PostgreSQL 9.1 и ниже вам нужно сделать небольшое исправление в исходниках. В файле <<smlar\_guc.c>> на линии 214 сделайте изменение с:

\begin{lstlisting}[label=lst:smlar2,caption=Фикс для 9.1 и ниже]
set_config_option("smlar.threshold", buf, PGC_USERSET, PGC_S_SESSION ,GUC_ACTION_SET, true, 0);
\end{lstlisting}

на (нужно убрать последний аргумент):

\begin{lstlisting}[label=lst:smlar3,caption=Фикс для 9.1 и ниже]
set_config_option("smlar.threshold", buf, PGC_USERSET, PGC_S_SESSION ,GUC_ACTION_SET, true);
\end{lstlisting}

Теперь проверим расширение:

\begin{lstlisting}[label=lst:smlar3,caption=Проверка smlar]
$ psql
psql (9.2.1)
Type "help" for help.

test=# CREATE EXTENSION smlar;
CREATE EXTENSION

test=# SELECT smlar('{1,4,6}'::int[], '{5,4,6}'::int[]);
  smlar  
----------
 0.666667
(1 row)

test=# SELECT smlar('{1,4,6}'::int[], '{5,4,6}'::int[], 'N.i / sqrt(N.a * N.b)' );
  smlar  
----------
 0.666667
(1 row)
\end{lstlisting}

Расширение установленно успешно, если у Вас такой же вывод в консоли.
\section{PLV8}
\textbf{Лицензия}: Open Source

\textbf{Ссылка}: \href{http://code.google.com/p/plv8js/}{code.google.com/p/plv8js}

PLV8 является библиотекой, которая предоставляет PostgreSQL процедурный язык с движком V8 JavaScript. С помощью этого расширения можно писать в PostgreSQL JavaScript функции, которые можно вызывать из SQL.
\section{HStore}
\textbf{Лицензия}: Open Source

HStore~-- расширение, которое реализует тип данных для хранения ключ/значение в пределах одного значения в PostgreSQL (например в одном текстовом поле). Это может быть полезно в различных ситуациях, таких как строки с многими атрибутами, которые редко вибираются, или полу-структурированные данные. Ключи и значения являются простыми текстовыми строками.

\section{PostPic}
\textbf{Лицензия}: Open Source

\textbf{Ссылка}: http://github.com/drotiro/postpic

PostPic расширение для СУБД PostgreSQL, которое позволяет обрабатывать изображения в базе данных, как PostGIS делает это с пространственными данными.
Он добавляет новый типа поля <<image>>, а также несколько функций для обработки изображений (кроп, создание миниатюр, поворот и т.д.) и 
извлечений его атрибутов (размер, тип, разрешение).

\section{Tsearch2}
\textbf{Лицензия}: Open Source

Tsearch2~-- расширение для полнотекстового поиска. Встроен в PostgreSQL начиная с версии 8.3.

\section{OpenFTS}
\textbf{Лицензия}: Open Source

\textbf{Ссылка}: http://openfts.sourceforge.net/

OpenFTS (Open Source Full Text Search engine) является продвинутой PostgreSQL поисковой системой, которая обеспечивает 
онлайн индексирования данных и актуальность данных для поиска по базе. Тесная интеграция с базой данных позволяет использовать метаданные, 
чтобы ограничить результаты поиска.

\section{PL/Proxy}
\textbf{Лицензия}: Open Source

\textbf{Ссылка}: http://pgfoundry.org/projects/plproxy/

PL/Proxy представляет собой прокси-язык для удаленного вызова процедур и партицирования данных между разными базами. 
Подробнее можно почитать в \Sref{sec:plproxy} главе.

\section{Texcaller}
\textbf{Лицензия}: Open Source

\textbf{Ссылка}: http://www.profv.de/texcaller/

Texcaller~--- это удобный интерфейс для командной строки TeX, которая обрабатывает все виды ошибок. Он написан в простом C, довольно портативный, 
и не имеет внешних зависимостей, кроме TeX. Неверный TeX документы обрабатываются путем простого возвращения NULL, 
а не прерывать с ошибкой. В случае неудачи, а также в случае успеха, дополнительная обработка информации осуществляется через NOTICEs.

\section{Pgmemcache}
\textbf{Лицензия}: Open Source

\textbf{Ссылка}: http://pgfoundry.org/projects/pgmemcache/

Pgmemcache~--- это PostgreSQL API библиотека на основе libmemcached для взаимодействия с memcached. С помощью данной библиотеки 
PostgreSQL может записывать, считывать, искать и удалять данные из memcached. Подробнее можно почитать в \Sref{sec:pgmemcache} главе.

\section{Prefix}
\textbf{Лицензия}: Open Source

\textbf{Ссылка}: http://pgfoundry.org/projects/prefix

Prefix реализует поиск текста по префиксу (prefix @> text). 
Prefix используется в приложениях телефонии, где маршрутизация вызовов и расходы зависят от 
вызывающего/вызываемого префикса телефонного номера оператора.

\section{pgSphere}
\textbf{Лицензия}: Open Source

\textbf{Ссылка}: http://pgsphere.projects.postgresql.org/

pgSphere обеспечивает PostgreSQL сферическими типами данных, а также функциями и операторами для работы с ними. 
Используется для работы с географическими (может использоватся вместо PostGIS) или астронамическими типами данных.

\section{Заключение}
Расширения помогают улучшить работу PostgreSQL в решении специфичеких проблем. Расширяемость PostgreSQL позволяет создавать собственные расширения, 
или же наоборот, не нагружать СУБД лишним, не требуемым функционалом.