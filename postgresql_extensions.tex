\chapter{Расширения}
\begin{epigraphs}
\qitem{Гибкость ума может заменить красоту.}{Стендаль}
\end{epigraphs}

\section{Введение}
Один из главных плюсов PostgreSQL это возможность расширения его функционала с помощью расширений. 
В данной статье я затрону только самые интересные и популярные из существующих. 

\section{PostGIS}
\textbf{Лицензия}: Open Source

\textbf{Ссылка}: http://www.postgis.org/

PostGIS добавляет поддержку для географических объектов в PostgreSQL. По сути PostGIS позволяет использовать PostgreSQL в качестве 
бэкэнда пространственной базы данных для геоинформационных систем (ГИС), так же, как ESRI SDE или пространственного расширения Oracle. 
PostGIS следует OpenGIS "Простые особенности Спецификация для SQL" и был сертифицирован.

\section{PostPic}
\textbf{Лицензия}: Open Source

\textbf{Ссылка}: http://github.com/drotiro/postpic

PostPic расширение для СУБД PostgreSQL, которое позволяет обрабатывать изображения в базе данных, как PostGIS делает это с пространственными данными.
Он добавляет новый типа поля <<image>>, а также несколько функций для обработки изображений (кроп, создание миниатюр, поворот и т.д.) и 
извлечений его атрибутов (размер, тип, разрешение).

\section{OpenFTS}
\textbf{Лицензия}: Open Source

\textbf{Ссылка}: http://openfts.sourceforge.net/

OpenFTS (Open Source Full Text Search engine) является продвинутой PostgreSQL поисковой системой, которая обеспечивает 
онлайн индексирования данных и актуальность данных для поиска по базе. Тесная интеграция с базой данных позволяет использовать метаданные, 
чтобы ограничить результаты поиска.

\section{PL/Proxy}
\textbf{Лицензия}: Open Source

\textbf{Ссылка}: http://pgfoundry.org/projects/plproxy/

PL/Proxy представляет собой прокси-язык для удаленного вызова процедур и партицирования данных между разными базами. 
Подробнее можно почитать в \Sref{sec:plproxy} главе.

\section{Texcaller}
\textbf{Лицензия}: Open Source

\textbf{Ссылка}: http://www.profv.de/texcaller/

Texcaller~--- это удобный интерфейс для командной строки TeX, которая обрабатывает все виды ошибок. Он написан в простом C, довольно портативный, 
и не имеет внешних зависимостей, кроме TeX. Неверный TeX документы обрабатываются путем простого возвращения NULL, 
а не прерывать с ошибкой. В случае неудачи, а также в случае успеха, дополнительная обработка информации осуществляется через NOTICEs.

\section{Pgmemcache}
\textbf{Лицензия}: Open Source

\textbf{Ссылка}: http://pgfoundry.org/projects/pgmemcache/

Pgmemcache~--- это PostgreSQL API библиотека на основе libmemcached для взаимодействия с memcached. С помощью данной библиотеки 
PostgreSQL может записывать, считывать, искать и удалять данные из memcached. Подробнее можно почитать в \Sref{sec:pgmemcache} главе.

\section{Prefix}
\textbf{Лицензия}: Open Source

\textbf{Ссылка}: http://pgfoundry.org/projects/prefix

Prefix реализует поиск текста по префиксу (prefix @> text). 
Prefix используется в приложениях телефонии, где маршрутизация вызовов и расходы зависят от 
вызывающего/вызываемого префикса телефонного номера оператора.

\section{pgSphere}
\textbf{Лицензия}: Open Source

\textbf{Ссылка}: http://pgsphere.projects.postgresql.org/

pgSphere обеспечивает PostgreSQL сферическими типами данных, а также функциями и операторами для работы с ними. 
Используется для работы с географическими (может использоватся вместо PostGIS) или астронамическими типами данных.

\section{Заключение}
Расширения помогают улучшить работу PostgreSQL в решении специфичеких проблем. Расширяемость PostgreSQL позволяет создавать собственные расширения, 
или же наоборот, не нагружать СУБД лишним, не требуемым функционалом.