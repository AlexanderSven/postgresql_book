\renewcommand\bibname{Литература}

\begin{thebibliography}{9}

\bibitem{pg1}
  Алексей Борзов (Sad Spirit) borz\_off@cs.msu.su
  \emph{PostgreSQL: настройка производительности}
  http://www.phpclub.ru/detail/store/pdf/postgresql-performance.pdf

\bibitem{pg2}
  Eugene Kuzin eugene@kuzin.net
  \emph{Настройка репликации в PostgreSQL с помощью системы Slony-I}
  http://www.kuzin.net/work/sloniki-privet.html

\bibitem{pg3}
  Sergey Konoplev gray.ru@gmail.com
  \emph{Установка Londiste в подробностях}
  http://gray-hemp.blogspot.com/2010/04/londiste.html

\bibitem{pg4}
  Dmitry Stasyuk
  \emph{Учебное руководство по pgpool-II}
  http://undenied.ru/2009/03/04/uchebnoe-rukovodstvo-po-pgpool-ii/

\bibitem{pg5}
  Чиркин Дима dmitry.chirkin@gmail.com
  \emph{Горизонтальное масштабирование PostgreSQL с помощью PL/Proxy}
  http://habrahabr.ru/blogs/postgresql/45475/

\bibitem{pg6}
  Иван Блинков wordpress@insight-it.ru
  \emph{Hadoop}
  http://www.insight-it.ru/masshtabiruemost/hadoop/

\bibitem{pg7}
  Padraig O'Sullivan
  \emph{Up and Running with HadoopDB}
  http://posulliv.github.com/2010/05/10/hadoopdb-mysql.html

\bibitem{pg8}
  Иван Золотухин
  \emph{Масштабирование PostgreSQL: готовые решения от Skype}
  http://postgresmen.ru/articles/view/25

\bibitem{pg9}
  \emph{Streaming Replication}.
  http://wiki.postgresql.org/wiki/Streaming\_Replication

\bibitem{pg10}
  Den Golotyuk
  \emph{Шардинг, партиционирование, репликация - зачем и когда?}
  http://highload.com.ua/index.php/2009/05/06/шардинг-партиционирование-репликац/

\bibitem{pg11}
  \emph{Postgres-XC — A PostgreSQL Clustering Solution}
  http://www.linuxforu.com/2012/01/postgres-xc-database-clustering-solution/

\bibitem{pg12}
  \emph{Введение в PostgreSQL BDR}
  http://habrahabr.ru/post/227959/

\bibitem{pg13}
  \emph{Популярный обзор внутренностей базы данных. Часть пятая}
  http://zamotivator.livejournal.com/332814.html

\bibitem{pg14}
  \emph{BRIN-индексы в PostgreSQL}
  http://langtoday.com/?p=485

\end{thebibliography}