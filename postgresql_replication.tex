\chapter{Репликация}

\begin{epigraphs}
\qitem{Когда решаете проблему, ни о чем не беспокойтесь.
Вот когда вы её решите, тогда и наступит время беспокоиться.}{Ричард Филлипс Фейман}
\end{epigraphs}

\section{Введение}

Репликация (англ. replication)~--- механизм синхронизации содержимого нескольких копий объекта (например, содержимого базы данных). Репликация~--- это процесс, под которым понимается копирование данных из одного источника на множество других и наоборот. При репликации изменения, сделанные в одной копии объекта, могут быть распространены в другие копии. Репликация может быть синхронной или асинхронной.

В случае синхронной репликации, если данная реплика обновляется, все другие реплики того же фрагмента данных также должны быть обновлены в одной и той же транзакции. Логически это означает, что существует лишь одна версия данных. В большинстве продуктов синхронная репликация реализуется с помощью триггерных процедур (возможно, скрытых и управляемых системой). Но синхронная репликация имеет тот недостаток, что она создаёт дополнительную нагрузку при выполнении всех транзакций, в которых обновляются какие-либо реплики (кроме того, могут возникать проблемы, связанные с доступностью данных).

В случае асинхронной репликации обновление одной реплики распространяется на другие спустя некоторое время, а не в той же транзакции. Таким образом, при асинхронной репликации вводится задержка, или время ожидания, в течение которого отдельные реплики могут быть фактически неидентичными (то есть определение реплика оказывается не совсем подходящим, поскольку мы не имеем дело с точными и своевременно созданными копиями). В большинстве продуктов асинхронная репликация реализуется посредством чтения журнала транзакций или постоянной очереди тех обновлений, которые подлежат распространению. Преимущество асинхронной репликации состоит в том, что дополнительные издержки репликации не связаны с транзакциями обновлений, которые могут иметь важное значение для функционирования всего предприятия и предъявлять высокие требования к производительности. К недостаткам этой схемы относится то, что данные могут оказаться несовместимыми (то есть несовместимыми с точки зрения пользователя). Иными словами, избыточность может проявляться на логическом уровне, а это, строго говоря, означает, что термин контролируемая избыточность в таком случае не применим.

Рассмотрим кратко проблему согласованности (или, скорее, несогласованности). Дело в том, что реплики могут становиться несовместимыми в результате ситуаций, которые трудно (или даже невозможно) избежать и последствия которых трудно исправить. В частности, конфликты могут возникать по поводу того, в каком порядке должны применяться обновления. Например, предположим, что в результате выполнения транзакции А происходит вставка строки в реплику X, после чего транзакция B удаляет эту строку, а также допустим, что Y~--- реплика X. Если обновления распространяются на Y, но вводятся в реплику Y в обратном порядке (например, из-за разных задержек при передаче), то транзакция B не находит в Y строку, подлежащую удалению, и не выполняет своё действие, после чего транзакция А вставляет эту строку. Суммарный эффект состоит в том, что реплика Y содержит указанную строку, а реплика X~--- нет.

В целом задачи устранения конфликтных ситуаций и обеспечения согласованности реплик являются весьма сложными. Следует отметить, что, по крайней мере, в сообществе пользователей коммерческих баз данных термин репликация стал означать преимущественно (или даже исключительно) асинхронную репликацию.

Основное различие между репликацией и управлением копированием заключается в следующем: если используется репликация, то обновление одной реплики в конечном счёте распространяется на все остальные автоматически. В режиме управления копированием, напротив, не существует такого автоматического распространения обновлений. Копии данных создаются и управляются с помощью пакетного или фонового процесса, который отделён во времени от транзакций обновления. Управление копированием в общем более эффективно по сравнению с репликацией, поскольку за один раз могут копироваться большие объёмы данных. К недостаткам можно отнести то, что большую часть времени копии данных не идентичны базовым данным, поэтому пользователи должны учитывать, когда именно были синхронизированы эти данные. Обычно управление копированием упрощается благодаря тому требованию, чтобы обновления применялись в соответствии со схемой первичной копии того или иного вида.

Для репликации PostgreSQL существует несколько решений, как закрытых, так и свободных. Закрытые системы репликации не будут рассматриваться в этой книге. Вот список свободных решений:
\begin{itemize}
  \item \href{http://www.slony.info/}{Slony-I}~--- асинхронная Master-Slave репликация, поддерживает каскады(cascading) и отказоустойчивость(failover). Slony-I использует триггеры PostgreSQL для привязки к событиям INSERT/DELETE/UPDATE и хранимые процедуры для выполнения действий;

  \item \href{http://pgfoundry.org/projects/pgcluster/}{PGCluster}~--- синхронная Multi-Master репликация. Проект на мой взгляд мертв;

  \item \href{http://pgpool.projects.postgresql.org/}{Pgpool-I/II}~--- это замечательный инструмент для PostgreSQL (лучше сразу работать с II версией). Позволяет делать:
  \begin{itemize}
    \item репликацию (в том числе, с автоматическим переключением на резервный stand-by сервер);
    \item online-бэкап;
    \item pooling коннектов;
    \item очередь соединений;
    \item балансировку SELECT-запросов на несколько postgresql-серверов;
    \item разбиение запросов для параллельного выполнения над большими объемами данных;
  \end{itemize}

  \item \href{http://bucardo.org/}{Bucardo}~--- асинхронная репликация, которая поддерживает Multi-Master и Master-Slave режимы, а также несколько видов синхронизации и обработки конфликтов;

  \item \href{http://skytools.projects.postgresql.org/doc/londiste.ref.html}{Londiste}~--- асинхронная Master-Slave репликация. Входит в состав Skytools\footnote{http://pgfoundry.org/projects/skytools/}. Проще в использовании, чем Slony-I;

  \item \href{http://www.commandprompt.com/products/mammothreplicator/}{Mammoth Replicator}~--- асинхронная Multi-Master репликация;

  \item \href{http://www.postgres-r.org/}{Postgres-R}~--- асинхронная Multi-Master репликация. Проект на мой взгляд мертв;

  \item \href{http://www.rubyrep.org/}{RubyRep}~--- написанная на Ruby, асинхронная Multi-Master репликация, которая поддерживает PostgreSQL и MySQL;
\end{itemize}

Это, конечно, не весь список свободных систем для репликации, но я думаю даже из этого есть что выбрать для PostgreSQL.


\section{Streaming Replication (Потоковая репликация)}
\subsection{Введение}
Потоковая репликация (Streaming Replication, SR) дает возможность непрерывно отправлять и применять 
wall xlog записи на резервные сервера для создания точной копии текущего. Данная функциональность 
появилась у PostgreSQL начиная с 9 версии (репликация из коробки!). Этот тип репликации простой, надежный и, вероятней всего,  
будет использоваться в качестве стандартной репликации в большинстве высоконагруженых приложений, что используют PostgreSQL. 

Отличительными особенностями решения являются:
\begin{itemize}
\item репликация всего инстанса PostgreSQL
\item асинхронный механизм репликации
\item простота установки
\item мастер база данных может обслуживать огромное количество слейвов из-за минимальной нагрузки
\end{itemize}

К недостаткам можно отнести:
\begin{itemize}
\item невозможность реплицировать только определенную базу данных из всех на PostgreSQL инстансе
\item асинхронный механизм~--- слейв отстает от мастера (но в отличие от других методов репликации, 
это отставание очень короткое, и может составлять всего лишь одну транзакцию, в зависимости от скорости сети, 
нагружености БД и настроек <<Hot Standby>>)
\end{itemize}

\subsection{Установка}
Для начала нам потребуется PostgreSQL не ниже 9 версии. В момент написания этой главы была доступна 9.0.1 версия. 
Все работы, как пологается, будут проводится на ОС Linux. 

\subsection{Настройка}
Для начала обозначим мастер сервер как masterdb(192.168.0.10) и слейв как slavedb(192.168.0.20).

\subsubsection{Предварительная настройка}
Для начала позволим определенному пользователю без пароля ходить по ssh. Пусть это будет postgres юзер. 
Если же нет, то создаем набором команд:
\begin{lstlisting}[label=lst:streaming1,caption=Создаем пользователя userssh]
$sudo groupadd userssh
$sudo useradd -m -g userssh -d /home/userssh -s /bin/bash \
-c "user ssh allow" userssh
\end{lstlisting}

Дальше выполняем команды от имени пользователя (в данном случае postgres):
\begin{lstlisting}[label=lst:streaming2,caption=Логинимся под пользователем postgres]
su postgres
\end{lstlisting}

Генерим RSA-ключ для обеспечения аутентификации в условиях отсутствия возможности использовать пароль:
\begin{lstlisting}[label=lst:streaming3,caption=Генерим RSA-ключ]
postgres@localhost ~ $ ssh-keygen -t rsa -P ""
Generating public/private rsa key pair.
Enter file in which to save the key (/var/lib/postgresql/.ssh/id_rsa): 
Created directory '/var/lib/postgresql/.ssh'.
Your identification has been saved in /var/lib/postgresql/.ssh/id_rsa.
Your public key has been saved in /var/lib/postgresql/.ssh/id_rsa.pub.
The key fingerprint is:
16:08:27:97:21:39:b5:7b:86:e1:46:97:bf:12:3d:76 postgres@localhost
\end{lstlisting}

И добавляем его в список авторизованных ключей:
\begin{lstlisting}[label=lst:streaming4,caption=Добавляем его в список авторизованных ключей]
cat $HOME/.ssh/id_rsa.pub >> $HOME/.ssh/authorized_keys
\end{lstlisting}

Этого должно быть более чем достаточно, проверить работоспособность соединения можно просто написав:
\begin{lstlisting}[label=lst:streaming5,caption=Пробуем зайти на ssh без пароля]
ssh localhost
\end{lstlisting}

Не забываем предварительно инициализировать sshd:
\begin{lstlisting}[label=lst:streaming6,caption=Запуск sshd]
/etc/init.d/sshd start
\end{lstlisting}

После успешно проделаной операции скопируйте <<\$HOME/.ssh>> на slavedb. 
Теперь мы должны иметь возможность без пароля заходить с мастера на слейв и со слейва на мастер через ssh.

Также отредактируем pg\_hba.conf на мастере и слейве, разрешив им друг к другу доступ без пароля(trust) (тут добавляется роль replication):
\begin{lstlisting}[label=lst:streaming7,caption=Мастер pg\_hba.conf]
host  replication  all  192.168.0.20/32  trust
\end{lstlisting}
\begin{lstlisting}[label=lst:streaming8,caption=Слейв pg\_hba.conf]
host  replication  all  192.168.0.10/32  trust
\end{lstlisting}

Не забываем после этого перегрузить postgresql на обоих серверах.

\subsubsection{Настройка мастера}
Для начала настроим masterdb. Установим параметры в postgresql.conf для репликации:
\begin{lstlisting}[label=lst:streaming9,caption=Настройка мастера]
# To enable read-only queries on a standby server, wal_level must be set to
# "hot_standby". But you can choose "archive" if you never connect to the
# server in standby mode.
wal_level = hot_standby

# Set the maximum number of concurrent connections from the standby servers.
max_wal_senders = 5

# To prevent the primary server from removing the WAL segments required for
# the standby server before shipping them, set the minimum number of segments
# retained in the pg_xlog directory. At least wal_keep_segments should be
# larger than the number of segments generated between the beginning of
# online-backup and the startup of streaming replication. If you enable WAL
# archiving to an archive directory accessible from the standby, this may
# not be necessary.
wal_keep_segments = 32

# Enable WAL archiving on the primary to an archive directory accessible from
# the standby. If wal_keep_segments is a high enough number to retain the WAL
# segments required for the standby server, this may not be necessary.
archive_mode    = on
archive_command = 'cp %p /path_to/archive/%f'
\end{lstlisting}

Давайте по порядку:
\begin{itemize}
\item <<wal\_level = hot\_standby>>~--- сервер начнет писать в WAL логи так же как и при режиме <<archive>>, 
добавляя информацию, необходимую для востановления транзакции (можно также поставить <<archive>>, 
но тогда сервер не может быть слейвом при необходимости).
\item <<max\_wal\_senders = 5>>~--- максимальное количество слейвов.
\item <<wal\_keep\_segments = 32>>~--- минимальное количество файлов c WAL сегментами в pg\_xlog директории.
\item <<archive\_mode = on>>~--- позволяем сохранять WAL сегменты в указаное переменной <<archive\_command>> хранилище. 
В данном случае в директорию <</path\_to/archive/>>.
\end{itemize}

По-умолчанию репликация асинхронная. В версии 9.1 добавили параметр <<synchronous\_standby\_names>>, который включает синхронную репликацию. В данные параметр передается <<application\_name>>, который используется на слейвах в recovery.conf:

\begin{lstlisting}[label=lst:streaming91,caption=recovery.conf для синхронной репликации на слейве]
restore_command = 'cp /tmp/%f %p'               # e.g. 'cp /mnt/server/archivedir/%f %p'
standby_mode = on
primary_conninfo = 'host=localhost port=59121 user=replication password=replication application_name=newcluster'            # e.g. 'host=localhost port=5432'
trigger_file = '/tmp/trig_f_newcluster'
\end{lstlisting}


После изменения параметров перегружаем PostgreSQL сервер. Теперь перейдем к slavedb.

\subsubsection{Настройка слейва}
Для начала нам потребуется создать на slavedb точную копию masterdb. Перенесем данные с помощью <<Онлайн бекапа>>.

Для начала зайдем на masterdb сервер. Выполним в консоли:
\begin{lstlisting}[label=lst:streaming10,caption=Выполняем на мастере]
psql -c "SELECT pg_start_backup('label', true)" 
\end{lstlisting}

Теперь нам нужно перенести данные с мастера на слейв. Выполняем на мастере:
\begin{lstlisting}[label=lst:streaming11,caption=Выполняем на мастере]
rsync -C -a --delete -e ssh --exclude postgresql.conf --exclude postmaster.pid \
--exclude postmaster.opts --exclude pg_log --exclude pg_xlog \
--exclude recovery.conf master_db_datadir/ slavedb_host:slave_db_datadir/ 
\end{lstlisting}
где
\begin{itemize}
\item <<master\_db\_datadir>>~--- директория с postgresql данными на masterdb
\item <<slave\_db\_datadir>>~--- директория с postgresql данными на slavedb
\item <<slavedb\_host>>~--- хост slavedb(в нашем случае - 192.168.1.20)
\end{itemize}

После копирования данных с мастера на слейв, остановим онлайн бекап. Выполняем на мастере:
\begin{lstlisting}[label=lst:streaming12,caption=Выполняем на мастере]
psql -c "SELECT pg_stop_backup()"
\end{lstlisting}

Устанавливаем такие же данные в конфиге postgresql.conf, что и у мастера (чтобы при падении мастера слейв мог его заменить). 
Так же установим допольнительный параметр:
\begin{lstlisting}[label=lst:streaming13,caption=Конфиг слейва]
hot_standby = on
\end{lstlisting}
Внимание! Если на мастере поставили <<wal\_level = archive>>, тогда параметр оставляем по умолчанию (hot\_standby = off).

Далее на slavedb в директории с данными PostgreSQL создадим файл recovery.conf с таким содержимым:
\begin{lstlisting}[label=lst:streaming14,caption=Конфиг recovery.conf]
# Specifies whether to start the server as a standby. In streaming replication,
# this parameter must to be set to on.
standby_mode          = 'on'

# Specifies a connection string which is used for the standby server to connect
# with the primary.
primary_conninfo      = 'host=192.168.0.10 port=5432 user=postgres'

# Specifies a trigger file whose presence should cause streaming replication to
# end (i.e., failover).
trigger_file = '/path_to/trigger'

# Specifies a command to load archive segments from the WAL archive. If
# wal_keep_segments is a high enough number to retain the WAL segments
# required for the standby server, this may not be necessary. But
# a large workload can cause segments to be recycled before the standby
# is fully synchronized, requiring you to start again from a new base backup.
restore_command = 'scp masterdb_host:/path_to/archive/%f "%p"'
\end{lstlisting}

где
\begin{itemize}
\item <<standby\_mode='on'>>~--- указываем серверу работать в режиме слейв
\item <<primary\_conninfo>>~--- настройки соединения слейва с мастером
\item <<trigger\_file>>~--- указываем триггер-файл, при наличии которого будет остановлена репликация.
\item <<restore\_command>>~--- команда, которой будет востанавливатся WAL логи. В нашем случае через 
scp копируем с masterdb (masterdb\_host - хост masterdb).
\end{itemize}

Теперь мы можем запустить PostgreSQL на slavedb.

\subsubsection{Тестирование репликации}
Теперь мы можем посмотреть отставание слейвов от мастера с помощью таких команд:
\begin{lstlisting}[label=lst:streaming15,caption=Тестирование репликации]
$ psql -c "SELECT pg_current_xlog_location()" -h192.168.0.10 (masterdb)
 pg_current_xlog_location 
--------------------------
 0/2000000
(1 row)

$ psql -c "select pg_last_xlog_receive_location()" -h192.168.0.20 (slavedb)
 pg_last_xlog_receive_location 
-------------------------------
 0/2000000
(1 row)

$ psql -c "select pg_last_xlog_replay_location()" -h192.168.0.20 (slavedb)
 pg_last_xlog_replay_location 
------------------------------
 0/2000000
(1 row)
\end{lstlisting}

Начиная с версии 9.1 добавили дополнительные view для просмотра состояния репликации. Теперь master знает все состояния slaves:

\begin{lstlisting}[label=lst:streaming151,caption=Состояние слейвов]
# SELECT * from pg_stat_replication ;
  procpid | usesysid |   usename   | application_name | client_addr | client_hostname | client_port |        backend_start         |   state   | sent_location | write_location | flush_location | replay_location | sync_priority | sync_state 
 ---------+----------+-------------+------------------+-------------+-----------------+-------------+------------------------------+-----------+---------------+----------------+----------------+-----------------+---------------+------------
    17135 |    16671 | replication | newcluster       | 127.0.0.1   |                 |       43745 | 2011-05-22 18:13:04.19283+02 | streaming | 1/30008750    | 1/30008750     | 1/30008750     | 1/30008750      |             1 | sync
\end{lstlisting}

Также с версии 9.1 добавили view pg\_stat\_database\_conflicts, с помощью которой на слейв базах можно просмотреть сколько запросов было отменено и по каким причинам:

\begin{lstlisting}[label=lst:streaming152,caption=Состояние слейва]
# SELECT * from pg_stat_database_conflicts ;
  datid |  datname  | confl_tablespace | confl_lock | confl_snapshot | confl_bufferpin | confl_deadlock 
 -------+-----------+------------------+------------+----------------+-----------------+----------------
      1 | template1 |                0 |          0 |              0 |               0 |              0
  11979 | template0 |                0 |          0 |              0 |               0 |              0
  11987 | postgres  |                0 |          0 |              0 |               0 |              0
  16384 | marc      |                0 |          0 |              1 |               0 |              0
\end{lstlisting}

Еще проверить работу репликации можно с помощью утилиты ps:
\begin{lstlisting}[label=lst:streaming16,caption=Тестирование репликации]
[masterdb] $ ps -ef | grep sender
postgres  6879  6831  0 10:31 ?        00:00:00 postgres: wal sender process postgres 127.0.0.1(44663) streaming 0/2000000

[slavedb] $ ps -ef | grep receiver
postgres  6878  6872  1 10:31 ?        00:00:01 postgres: wal receiver process   streaming 0/2000000
\end{lstlisting}

Теперь проверим реприкацию. Выполним на мастере:
\begin{lstlisting}[language=SQL,label=lst:streaming17,caption=Выполняем на мастере]
$psql test_db
test_db=# create table test3(id int not null primary key,name varchar(20));
NOTICE:  CREATE TABLE / PRIMARY KEY will create implicit index "test3_pkey" for table "test3"
CREATE TABLE
test_db=# insert into test3(id, name) values('1', 'test1');
INSERT 0 1
test_db=#
\end{lstlisting}

Теперь проверим на слейве:
\begin{lstlisting}[language=SQL,label=lst:streaming18,caption=Выполняем на слейве]
$psql test_db
test_db=# select * from test3;
 id | name  
----+-------
  1 | test1
(1 row)
\end{lstlisting}

Как видим, таблица с данными успешно скопирована с мастера на слейв. 

\subsection{Общие задачи}
\subsubsection{Переключение на слейв при падении мастера}
Достаточно создать триггер-файл (trigger\_file) на слейве, который становится мастером.

\subsubsection{Остановка репликации на слейве}
Создать триггер-файл (trigger\_file) на слейве. Также с версии 9.1 добавили команды pg\_xlog\_replay\_pause() и pg\_xlog\_replay\_resume() для остановки и возобновления репликации.

\subsubsection{Перезапуск репликации после сбоя}
Повторяем операции из раздела <<Настройка слейва>>. Хочется заметить, что мастер при этом не нуждается в остановке при выполнении данной задачи.

\subsubsection{Перезапуск репликации после сбоя слейва}
Перезагрузить PostgreSQL на слейве после устранения сбоя.

\subsubsection{Повторно синхронизировать репликации на слейве}
Это может потребоватся, например, после длительного отключения от мастера. 
Для этого останавливаем PostgreSQL на слейве и повторяем операции из раздела <<Настройка слейва>>.
\section{Slony-I}
\subsection{Введение}
Slony это система репликации реального времени, позволяющая организовать синхронизацию нескольких серверов
PostgreSQL по сети. Slony использует триггеры Postgre для привязки к событиям INSERT/ DELETE/UPDATE и
хранимые процедуры для выполнения действий.

Система Slony с точки зрения администратора состоит из двух главных компонент: репликационного демона slony и
административной консоли slonik. Администрирование системы сводится к общению со slonik-ом, демон slon только
следит за собственно процессом репликации. А админ следит за тем, чтобы slon висел там, где ему положено.

\subsubsection{О slonik-e}
Все команды slonik принимает на свой stdin. До начала выполнения скрипт slonik-a проверяется на соответствие синтаксису,
если обнаруживаются ошибки, скрипт не выполняется, так что можно не волноваться если slonik сообщает о syntax error,
ничего страшного не произошло. И он ещё ничего не сделал. Скорее всего.

\subsection{Установка}
Установка на Ubuntu производится простой командой:
\begin{lstlisting}[label=lst:slony1,caption=Установка]
sudo aptitude install slony1-bin
\end{lstlisting}

\subsection{Настройка}
\label{sec:slonyI}
Рассмотрим теперь установку на гипотетическую базу данных customers
(названия узлов, кластеров и таблиц являются вымышленными).

Наши данные
\begin{itemize}
\item БД: customers
\item master\_host: customers\_master.com
\item slave\_host\_1: customers\_slave.com
\item cluster name (нужно придумать): customers\_rep
\end{itemize}

\subsubsection{Подготовка master-сервера}
Для начала нам нужно создать пользователя Postgres, под которым будет действовать Slony.
По умолчанию, и отдавая должное системе, этого пользователя обычно называют slony.
\begin{lstlisting}[label=lst:slony2,caption=Подготовка master-сервера]
pgsql@customers_master$ createuser -a -d slony
pgsql@customers_master$ psql -d template1 -c "alter \
user slony with password 'slony_user_password';"
\end{lstlisting}

Также на каждом из узлов лучше завести системного пользователя slony, чтобы запускать от его имени
репликационного демона slon. В дальнейшем подразумевается, что он (и пользователь и slon) есть на
каждом из узлов кластера.

\subsubsection{Подготовка одного slave-сервера}
Здесь я рассматриваю, что серверы кластера соединены посредством сети Internet (как в моём случае), необходимо
чтобы с каждого из ведомых серверов можно было установить соединение с PostgreSQL на мастер-хосте, и наоборот.
То есть, команда:
\begin{lstlisting}[label=lst:slony3,caption=Подготовка одного slave-сервера]
anyuser@customers_slave$ psql -d customers \
-h customers_master.com -U slony
\end{lstlisting}

должна подключать нас к мастер-серверу (после ввода пароля, желательно). Если что-то не так, возможно требуется
поковыряться в настройках firewall-a, или файле pg\_hba.conf, который лежит в \$PGDATA.

Теперь устанавливаем на slave-хост сервер PostgreSQL. Следующего обычно не требуется, сразу после установки Postgres
<<up and ready>>, но в случае каких-то ошибок можно начать <<с чистого листа>>, выполнив следующие команды
(предварительно сохранив конфигурационные файлы и остановив postmaster):
\begin{lstlisting}[label=lst:slony4,caption=Подготовка одного slave-сервера]
pgsql@customers_slave$ rm -rf $PGDATA
pgsql@customers_slave$ mkdir $PGDATA
pgsql@customers_slave$ initdb -E UTF8 -D $PGDATA
pgsql@customers_slave$ createuser -a -d slony
pgsql@customers_slave$ psql -d template1 -c "alter \
user slony with password 'slony_user_password';"
\end{lstlisting}

Запускаем postmaster.

Внимание! Обычно требуется определённый владелец для реплицируемой БД. В этом случае необходимо завести его тоже!
\begin{lstlisting}[label=lst:slony5,caption=Подготовка одного slave-сервера]
pgsql@customers_slave$ createuser -a -d customers_owner
pgsql@customers_slave$ psql -d template1 -c "alter \
user customers_owner with password 'customers_owner_password';"
\end{lstlisting}

Эти две команды можно запускать с customers\_master, к командной строке в этом случае нужно добавить
<<-h customers\_slave>>, чтобы все операции выполнялись на slave.

На slave, как и на master, также нужно установить Slony.

\subsubsection{Инициализация БД и plpgsql на slave}

Следующие команды выполняются от пользователя slony. Скорее всего для выполнения каждой из них потребуется
ввести пароль (slony\_user\_password). Итак:
\begin{lstlisting}[label=lst:slony6,caption=Инициализация БД и plpgsql на slave]
slony@customers_master$ createdb -O customers_owner \
-h customers_slave.com customers
slony@customers_master$ createlang -d customers \
-h customers_slave.com plpgsql
\end{lstlisting}

Внимание! Все таблицы, которые будут добавлены в replication set должны иметь primary key.
Если какая-то из таблиц не удовлетворяет этому условию, задержитесь на этом шаге и дайте каждой таблице primary key
командой ALTER TABLE ADD PRIMARY KEY.

Если столбца который мог бы стать primary key не находится, добавьте новый столбец типа serial (ALTER TABLE ADD COLUMN),
и заполните его значениями. Настоятельно НЕ рекомендую использовать <<table add key>> slonik-a.

Продолжаем.
Создаём таблицы и всё остальное на slave:
\begin{lstlisting}[label=lst:slony7,caption=Инициализация БД и plpgsql на slave]
slony@customers_master$ pg_dump -s customers | \
psql -U slony -h customers_slave.com customers
\end{lstlisting}

pg\_dump -s сдампит только структуру нашей БД.

pg\_dump -s customers должен пускать без пароля, а вот для psql -U slony -h customers\_slave.com
customers придётся набрать пароль (slony\_user\_pass). Важно: я подразумеваю что сейчас на мастер-хосте
ещё не установлен Slony (речь не про make install), то есть в БД нет таблиц sl\_*, триггеров и прочего.
Если есть, то возможно два варианта:
\begin{itemize}
\item добавляется узел в уже функционирующую систему репликации (читайте раздел 5)
\item это ошибка :-) Тогда до переноса структуры на slave выполните следующее:
\begin{lstlisting}[label=lst:slony8,caption=Инициализация БД и plpgsql на slave]
slonik <<EOF
cluster name = customers_slave;
node Y admin conninfo = 'dbname=customers host=customers_master.com
port=5432 user=slony password=slony_user_pass';
uninstall node (id = Y);
echo 'okay';
EOF
\end{lstlisting}
Y~--- число. Любое. Важно: если это действительно ошибка, cluster name может иметь какой-то другое значение, например T1
(default). Нужно его выяснить и сделать uninstall.

Если структура уже перенесена (и это действительно ошибка), сделайте uninstall с обоих узлов (с master и slave).
\end{itemize}

\subsubsection{Инициализация кластера}
Если Сейчас мы имеем два сервера PgSQL которые свободно <<видят>> друг друга по сети,
на одном из них находится мастер-база с данными, на другом~--- только структура.

На мастер-хосте запускаем такой скрипт:
\begin{lstlisting}[label=lst:slony9,caption=Инициализация кластера]
#!/bin/sh

CLUSTER=customers_rep

DBNAME1=customers
DBNAME2=customers

HOST1=customers_master.com
HOST2=customers_slave.com

PORT1=5432
PORT2=5432

SLONY_USER=slony

slonik <<EOF
cluster name = $CLUSTER;
node 1 admin conninfo = 'dbname=$DBNAME1 host=$HOST1 port=$PORT1
user=slony password=slony_user_password';
node 2 admin conninfo = 'dbname=$DBNAME2 host=$HOST2
port=$PORT2 user=slony password=slony_user_password';
init cluster ( id = 1, comment = 'Customers DB
replication cluster' );

echo 'Create set';

create set ( id = 1, origin = 1, comment = 'Customers
DB replication set' );

echo 'Adding tables to the subscription set';

echo ' Adding table public.customers_sales...';
set add table ( set id = 1, origin = 1, id = 4, full qualified
name = 'public.customers_sales', comment = 'Table public.customers_sales' );
echo ' done';

echo ' Adding table public.customers_something...';
set add table ( set id = 1, origin = 1, id = 5, full qualified
name = 'public.customers_something,
comment = 'Table public.customers_something );
echo ' done';

echo 'done adding';
store node ( id = 2, comment = 'Node 2, $HOST2' );
echo 'stored node';
store path ( server = 1, client = 2, conninfo = 'dbname=$DBNAME1 host=$HOST1
port=$PORT1 user=slony password=slony_user_password' );
echo 'stored path';
store path ( server = 2, client = 1, conninfo = 'dbname=$DBNAME2 host=$HOST2
port=$PORT2 user=slony password=slony_user_password' );

store listen ( origin = 1, provider = 1, receiver = 2 );
store listen ( origin = 2, provider = 2, receiver = 1 );
EOF
\end{lstlisting}

Здесь мы инициализируем кластер, создаём репликационный набор, включаем в него две таблицы.
Важно: нужно перечислить все таблицы, которые нужно реплицировать, id таблицы в наборе должен быть уникальным,
таблицы должны иметь primary key.

Важно: replication set запоминается раз и навсегда. Чтобы добавить узел в схему репликации не нужно заново инициализировать set.

Важно: если в набор добавляется или удаляется таблица нужно переподписать все узлы.
То есть сделать unsubscribe и subscribe заново.

\subsubsection{Подписываем slave-узел на replication set}
Скрипт:
\begin{lstlisting}[label=lst:slony10,caption=Подписываем slave-узел на replication set]
#!/bin/sh

CLUSTER=customers_rep

DBNAME1=customers
DBNAME2=customers

HOST1=customers_master.com
HOST2=customers_slave.com

PORT1=5432
PORT2=5432

SLONY_USER=slony

slonik <<EOF
cluster name = $CLUSTER;
node 1 admin conninfo = 'dbname=$DBNAME1 host=$HOST1
port=$PORT1 user=slony password=slony_user_password';
node 2 admin conninfo = 'dbname=$DBNAME2 host=$HOST2
port=$PORT2 user=slony password=slony_user_password';

echo'subscribing';
subscribe set ( id = 1, provider = 1, receiver = 2, forward = no);

EOF
\end{lstlisting}

\subsubsection{Старт репликации}
Теперь, на обоих узлах необходимо запустить демона репликации.
\begin{lstlisting}[label=lst:slony11,caption=Старт репликации]
slony@customers_master$ slon customers_rep \
"dbname=customers user=slony"
\end{lstlisting}

и
\begin{lstlisting}[label=lst:slony12,caption=Старт репликации]
slony@customers_slave$ slon customers_rep \
"dbname=customers user=slony"
\end{lstlisting}

Сейчас слоны обменяются сообщениями и начнут передачу данных. Начальное наполнение происходит с помощью COPY,
slave DB на это время полностью блокируется.

В среднем время актуализации данных на slave-системе составляет до 10-ти секунд.
slon успешно обходит проблемы со связью и подключением к БД, и вообще требует к
себе достаточно мало внимания.

\subsection{Общие задачи}
\subsubsection{Добавление ещё одного узла в работающую схему репликации}
Выполнить~\ref{sec:slonyI}.1 и выполнить~\ref{sec:slonyI}.2.

Новый узел имеет id = 3. Находится на хосте customers\_slave3.com, <<видит>> мастер-сервер по сети и
мастер может подключиться к его PgSQL.

после дублирования структуры (п~\ref{sec:slonyI}.2) делаем следующее:
\begin{lstlisting}[label=lst:slony13,caption=Общие задачи]
slonik <<EOF
cluster name = customers_slave;
node 3 admin conninfo = 'dbname=customers host=customers_slave3.com
port=5432 user=slony password=slony_user_pass';
uninstall node (id = 3);
echo 'okay';
EOF
\end{lstlisting}

Это нужно чтобы удалить схему, триггеры и процедуры, которые были сдублированы вместе с таблицами и структурой БД.

Инициализировать кластер не надо. Вместо этого записываем информацию о новом узле в сети:
\begin{lstlisting}[label=lst:slony14,caption=Общие задачи]
#!/bin/sh

CLUSTER=customers_rep

DBNAME1=customers
DBNAME3=customers

HOST1=customers_master.com
HOST3=customers_slave3.com

PORT1=5432
PORT2=5432

SLONY_USER=slony

slonik <<EOF
cluster name = $CLUSTER;
node 1 admin conninfo = 'dbname=$DBNAME1 host=$HOST1
port=$PORT1 user=slony password=slony_user_pass';
node 3 admin conninfo = 'dbname=$DBNAME3
host=$HOST3 port=$PORT2 user=slony password=slony_user_pass';

echo 'done adding';

store node ( id = 3, comment = 'Node 3, $HOST3' );
echo 'sored node';
store path ( server = 1, client = 3, conninfo = 'dbname=$DBNAME1
host=$HOST1 port=$PORT1 user=slony password=slony_user_pass' );
echo 'stored path';
store path ( server = 3, client = 1, conninfo = 'dbname=$DBNAME3
host=$HOST3 port=$PORT2 user=slony password=slony_user_pass' );

echo 'again';
store listen ( origin = 1, provider = 1, receiver = 3 );
store listen ( origin = 3, provider = 3, receiver = 1 );

EOF
\end{lstlisting}

Новый узел имеет id 3, потому что 2 уже есть и работает. Подписываем новый узел 3 на replication set:
\begin{lstlisting}[label=lst:slony15,caption=Общие задачи]
#!/bin/sh

CLUSTER=customers_rep

DBNAME1=customers
DBNAME3=customers

HOST1=customers_master.com
HOST3=customers_slave3.com

PORT1=5432
PORT2=5432

SLONY_USER=slony

slonik <<EOF
cluster name = $CLUSTER;
node 1 admin conninfo = 'dbname=$DBNAME1 host=$HOST1
port=$PORT1 user=slony password=slony_user_pass';
node 3 admin conninfo = 'dbname=$DBNAME3 host=$HOST3
port=$PORT2 user=slony password=slony_user_pass';

echo'subscribing';
subscribe set ( id = 1, provider = 1, receiver = 3, forward = no);

EOF
\end{lstlisting}

Теперь запускаем slon на новом узле, так же как и на остальных. Перезапускать slon на мастере не надо.
\begin{lstlisting}[label=lst:slony16,caption=Общие задачи]
slony@customers_slave3$ slon customers_rep \
"dbname=customers user=slony"
\end{lstlisting}

Репликация должна начаться как обычно.

\subsection{Устранение неисправностей}
\subsubsection{Ошибка при добавлении узла в систему репликации}
Периодически, при добавлении новой машины в кластер возникает следующая ошибка: на новой ноде всё начинает
жужжать и работать, имеющиеся же отваливаются с примерно следующей диагностикой:
\begin{lstlisting}[label=lst:slony17,caption=Устранение неисправностей]
%slon customers_rep "dbname=customers user=slony_user"
CONFIG main: slon version 1.0.5 starting up
CONFIG main: local node id = 3
CONFIG main: loading current cluster configuration
CONFIG storeNode: no_id=1 no_comment='CustomersDB
replication cluster'
CONFIG storeNode: no_id=2 no_comment='Node 2,
node2.example.com'
CONFIG storeNode: no_id=4 no_comment='Node 4,
node4.example.com'
CONFIG storePath: pa_server=1 pa_client=3
pa_conninfo="dbname=customers
host=mainhost.com port=5432 user=slony_user
password=slony_user_pass" pa_connretry=10
CONFIG storeListen: li_origin=1 li_receiver=3
li_provider=1
CONFIG storeSet: set_id=1 set_origin=1
set_comment='CustomersDB replication set'
WARN remoteWorker_wakeup: node 1 - no worker thread
CONFIG storeSubscribe: sub_set=1 sub_provider=1 sub_forward='f'
WARN remoteWorker_wakeup: node 1 - no worker thread
CONFIG enableSubscription: sub_set=1
WARN remoteWorker_wakeup: node 1 - no worker thread
CONFIG main: configuration complete - starting threads
CONFIG enableNode: no_id=1
CONFIG enableNode: no_id=2
CONFIG enableNode: no_id=4
ERROR remoteWorkerThread_1: "begin transaction; set
transaction isolation level
serializable; lock table "_customers_rep".sl_config_lock; select
"_customers_rep".enableSubscription(1, 1, 4);
notify "_customers_rep_Event"; notify "_customers_rep_Confirm";
insert into "_customers_rep".sl_event (ev_origin, ev_seqno,
ev_timestamp, ev_minxid, ev_maxxid, ev_xip,
ev_type , ev_data1, ev_data2, ev_data3, ev_data4 ) values
('1', '219440',
'2005-05-05 18:52:42.708351', '52501283', '52501292',
'''52501283''', 'ENABLE_SUBSCRIPTION',
'1', '1', '4', 'f'); insert into "_customers_rep".
sl_confirm (con_origin, con_received,
con_seqno, con_timestamp) values (1, 3, '219440',
CURRENT_TIMESTAMP); commit transaction;"
PGRES_FATAL_ERROR ERROR: insert or update on table
"sl_subscribe" violates foreign key
constraint "sl_subscribe-sl_path-ref"
DETAIL: Key (sub_provider,sub_receiver)=(1,4)
is not present in table "sl_path".
INFO remoteListenThread_1: disconnecting from
'dbname=customers host=mainhost.com
port=5432 user=slony_user password=slony_user_pass'
%
\end{lstlisting}


Это означает что в служебной таблице \_<имя кластера>.sl\_path;, например
\_customers\_rep.sl\_path на уже имеющихся узлах отсутствует информация о новом узле. В данном случае,
id нового узла 4, пара (1,4) в sl\_path отсутствует.

Видимо, это баг Slony. Как избежать этого и последующих ручных вмешательств пока не ясно.

Чтобы это устранить, нужно выполнить на каждом из имеющихся узлов приблизительно следующий запрос
(добавить путь, в данном случае (1,4)):
\begin{lstlisting}[label=lst:slony18,caption=Устранение неисправностей]
slony_user@masterhost$ psql -d customers -h _every_one_of_slaves -U slony
customers=# insert into _customers_rep.sl_path
values ('1','4','dbname=customers host=mainhost.com
port=5432 user=slony_user password=slony_user_password,'10');
\end{lstlisting}

Если возникают затруднения, да и вообще для расширения кругозора можно посмотреть на служебные таблицы
и их содержимое. Они не видны обычно и находятся в рамках пространства имён \_<имя кластера>,
например \_customers\_rep.

\subsubsection{Что делать если репликация со временем начинает тормозить}
В процессе эксплуатации наблюдаю как со временем растёт нагрузка на master-сервере, в списке активных бекендов~---
постоянные SELECT-ы со слейвов. В pg\_stat\_activity видим примерно такие запросы:
\begin{lstlisting}[label=lst:slony19,caption=Устранение неисправностей]
select ev_origin, ev_seqno, ev_timestamp, ev_minxid, ev_maxxid, ev_xip,
ev_type, ev_data1, ev_data2, ev_data3, ev_data4, ev_data5, ev_data6,
ev_data7, ev_data8 from "_customers_rep".sl_event e where
(e.ev_origin = '2' and e.ev_seqno > '336996') or
(e.ev_origin = '3' and e.ev_seqno > '1712871') or
(e.ev_origin = '4' and e.ev_seqno > '721285') or
(e.ev_origin = '5' and e.ev_seqno > '807715') or
(e.ev_origin = '1' and e.ev_seqno > '3544763') or
(e.ev_origin = '6' and e.ev_seqno > '2529445') or
(e.ev_origin = '7' and e.ev_seqno > '2512532') or
(e.ev_origin = '8' and e.ev_seqno > '2500418') or
(e.ev_origin = '10' and e.ev_seqno > '1692318')
order by e.ev_origin, e.ev_seqno;
\end{lstlisting}

Не забываем что \_customers\_rep~--- имя схемы из примера, у вас будет другое имя.

Таблица sl\_event почему-то разрастается со временем, замедляя выполнение этих
запросов до неприемлемого времени. Удаляем ненужные записи:
\begin{lstlisting}[label=lst:slony20,caption=Устранение неисправностей]
delete from _customers_rep.sl_event where
ev_timestamp<NOW()-'1 DAY'::interval;
\end{lstlisting}

Производительность должна вернуться к изначальным значениям.
Возможно имеет смысл почистить таблицы \_customers\_rep.sl\_log\_* где
вместо звёздочки подставляются натуральные числа, по-видимому по
количеству репликационных сетов, так что \_customers\_rep.sl\_log\_1
точно должна существовать.
\section{Londiste}
\label{sec:londiste}

\subsection{Введение}
Londiste представляет собой движок для организации репликации, написанный на языке python.
Основные принципы: надежность и простота использования. Из-за этого данное решение имеет меньше функциональности,
чем Slony-I. Londiste использует в качестве транспортного механизма очередь PgQ  (описание этого более чем интересного
проекта остается за рамками данной главы, поскольку он представляет интерес скорее для низкоуровневых программистов
баз данных, чем для конечных пользователей~--- администраторов СУБД PostgreSQL). Отличительными особенностями решения являются:

\begin{itemize}
  \item возможность потабличной репликации
  \item начальное копирование ничего не блокирует
  \item возможность двухстороннего сравнения таблиц
  \item простота установки
\end{itemize}

К недостаткам можно отнести:

\begin{itemize}
  \item триггерная репликация, что ухудшает производительность базы
\end{itemize}


\subsection{Установка}
На серверах, которые мы настраиваем рассматривается ОС Linux, а именно Ubuntu Server.
Автор данной книги считает, что под другие операционные системы (кроме Windows) все мало чем будет отличаться,
а держать кластера PostgreSQL под ОС Windows, по меньшей мере, неразумно.

Поскольку Londiste~--- это часть Skytools, то нам нужно ставить этот пакет. На таких системах, как Debian или Ubuntu skytools
можно найти в репозитории пакетов и поставить одной командой:
\begin{lstlisting}[label=lst:londiste1,caption=Установка]
% sudo aptitude install skytools
\end{lstlisting}

Но в системных пакетах может содержатся версия 2.x, которая не поддерживает каскадную репликацию, отказоустойчивость(failover) и переключение между серверами (switchover). По этой причине я не буду её рассматривать. Скачать самую последнюю версию пакета можно с \href{http://pgfoundry.org/projects/skytools}{официального сайта}.
На момент написания главы последняя версия была 3.2. Итак, начнем:

\begin{lstlisting}[label=lst:londiste2,caption=Установка]
$ wget http://pgfoundry.org/frs/download.php/3622/skytools-3.2.tar.gz
$ tar zxvf skytools-3.2.tar.gz
$ cd skytools-3.2/
# пакеты для сборки deb
$ sudo aptitude install build-essential autoconf \
automake autotools-dev dh-make \
debhelper devscripts fakeroot xutils lintian pbuilder \
python-all-dev python-support xmlto asciidoc \
libevent-dev libpq-dev libtool
# python-psycopg нужен для работы Londiste
$ sudo aptitude install python-psycopg2 postgresql-server-dev-all
# данной командой собираем deb пакет
$ make deb
$ cd ../
# ставим skytools
$ dpkg -i *.deb
\end{lstlisting}

Для других систем можно собрать Skytools командами:

\begin{lstlisting}[label=lst:londiste3,caption=Установка]
$ ./configure
$ make
$ make install
\end{lstlisting}

Дальше проверим, что все у нас правильно установилось
\begin{lstlisting}[label=lst:londiste4,caption=Установка]
$ londiste3 -V
londiste3, Skytools version 3.2
$ pgqd -V
bad switch: usage: pgq-ticker [switches] config.file
Switches:
  -v        Increase verbosity
  -q        No output to console
  -d        Daemonize
  -h        Show help
  -V        Show version
 --ini      Show sample config file
  -s        Stop - send SIGINT to running process
  -k        Kill - send SIGTERM to running process
  -r        Reload - send SIGHUP to running process
\end{lstlisting}

Если у Вас похожий вывод, значит все установлено правильно и можно приступать к настройке.


\subsection{Настройка}

Обозначения:
\begin{itemize}
  \item master-host~--- мастер база данных;
  \item slave1-host, slave2-host~--- слейв базы данных;
  \item l3simple - название реплицируемой базы данных;
\end{itemize}

\subsubsection{Создаём конфигурацию репликаторов}
Для начала создадим конфигурационный файл для master базы
(пусть конфиг будет у нас /etc/skytools/master-londiste.ini):
\begin{lstlisting}[label=lst:londiste-replica1,caption=Создаём конфигурацию репликатора]
[londiste3]
job_name = master_l3simple
db = dbname=l3simple
queue_name = replika
logfile = /var/log/skytools/master_l3simple.log
pidfile = /var/pid/skytools/master_l3simple.pid

# Задержка между проверками наличия активности
# (новых пакетов данных) в секундах
loop_delay = 0.5
\end{lstlisting}

Инициализируем Londiste для master базы:

\begin{lstlisting}[label=lst:londiste-replica2,caption=Инициализируем Londiste]
$ londiste3 /etc/skytools/master-londiste.ini create-root master-node "dbname=l3simple host=master-host"
INFO plpgsql is installed
INFO Installing pgq
INFO   Reading from /usr/share/skytools3/pgq.sql
INFO pgq.get_batch_cursor is installed
INFO Installing pgq_ext
INFO   Reading from /usr/share/skytools3/pgq_ext.sql
INFO Installing pgq_node
INFO   Reading from /usr/share/skytools3/pgq_node.sql
INFO Installing londiste
INFO   Reading from /usr/share/skytools3/londiste.sql
INFO londiste.global_add_table is installed
INFO Initializing node
INFO Location registered
INFO Node "master-node" initialized for queue "replika" with type "root"
INFO Done
\end{lstlisting}

master-server~--- это имя провайдера (мастера базы).

Теперь можем запустить демон:

\begin{lstlisting}[label=lst:londiste-replica3,caption=Запускаем демон для master базы]
$ londiste3 -d /etc/skytools/master-londiste.ini worker
$ tail -f /var/log/skytools/master_l3simple.log
INFO {standby: 1}
INFO {standby: 1}
\end{lstlisting}

Если нужно перегрузить демон (например, изменили конфиг), то можно воспользоватся параметром <<-r>>:

\begin{lstlisting}[label=lst:londiste-replica4,caption=Перегрузка демона]
$ londiste3 /etc/skytools/master-londiste.ini -r
\end{lstlisting}

Для остановки демона есть параметр <<-s>>:
\begin{lstlisting}[label=lst:londiste-replica5,caption=Остановка демона]
$ londiste3 /etc/skytools/master-londiste.ini -s
\end{lstlisting}

или если потребуется <<убить>> (kill -9) демон:
\begin{lstlisting}[label=lst:londiste-replica6,caption=Остановка демона]
$ londiste3 /etc/skytools/master-londiste.ini -k
\end{lstlisting}

Для автоматизации этого процесса skytools3 имеет встроенный демон, который подымает все воркеры из директории /etc/skytools/. Сама конфигурация демона находится в /etc/skytools.ini. Что бы запустить все демоны londiste достаточно выполнить:

\begin{lstlisting}[label=lst:londiste-replica7,caption=Демон для ticker]
$ /etc/init.d/skytools3 start
INFO Starting master_l3simple
\end{lstlisting}

Перейдем к slave базе.

Для начала мы должны создать базу данных:

\begin{lstlisting}[label=lst:londiste-replica8,caption=Копирования структуры базы]
$ psql -h slave1-host -U postgres
# CREATE DATABASE l3simple;
\end{lstlisting}

Подключение должно быть <<trust>> (без паролей) между master и slave базами данных.

Далее создадим конфиг для slave базы (/etc/skytools/slave1-londiste.ini):
\begin{lstlisting}[label=lst:londiste-replica9,caption=Создаём конфигурацию для slave]
[londiste3]
job_name = slave1_l3simple
db = dbname=l3simple
queue_name = replika
logfile = /var/log/skytools/slave1_l3simple.log
pidfile = /var/pid/skytools/slave1_l3simple.pid

# Задержка между проверками наличия активности
# (новых пакетов данных) в секундах
loop_delay = 0.5
\end{lstlisting}

Инициализируем Londiste для slave базы:

\begin{lstlisting}[label=lst:londiste-replica10,caption=Инициализируем Londiste для slave]
$ londiste3 /etc/skytools/slave1-londiste.ini create-leaf slave1-node "dbname=l3simple host=slave1-host" --provider="dbname=l3simple host=master-host"
\end{lstlisting}

Теперь можем запустить демон:

\begin{lstlisting}[label=lst:londiste-replica11,caption=Запускаем демон для slave базы]
$ londiste3 -d /etc/skytools/slave1-londiste.ini worker
\end{lstlisting}

Или же через главный демон:

\begin{lstlisting}[label=lst:londiste-replica12,caption=Запускаем демон для slave базы]
$ /etc/init.d/skytools3 start
INFO Starting master_l3simple
INFO Starting slave1_l3simple
\end{lstlisting}


\subsubsection{Создаём конфигурацию для PgQ ticker}

Londiste требуется PgQ ticker для работы с мастер базой данных, который может быть запущен и на другой машине. Но, конечно, лучше его запускать на той же, где и master база данных. Для этого мы настраиваем специальный конфиг для ticker демона (пусть конфиг будет у нас /etc/skytools/pgqd.ini):

\begin{lstlisting}[label=lst:londiste-replica13,caption=PgQ ticker конфиг]
[pgqd]
logfile = /var/log/skytools/pgqd.log
pidfile = /var/pid/skytools/pgqd.pid
\end{lstlisting}

Запускаем демон:

\begin{lstlisting}[label=lst:londiste-replica11,caption=Запускаем PgQ ticker]
$ pgqd -d /etc/skytools/pgqd.ini
$ tail -f /var/log/skytools/pgqd.log
LOG Starting pgqd 3.2
LOG auto-detecting dbs ...
LOG l3simple: pgq version ok: 3.2
\end{lstlisting}

Или же через глобальный демон:

\begin{lstlisting}[label=lst:londiste-replica12,caption=Запускаем PgQ ticker]
$ /etc/init.d/skytools3 restart
INFO Starting master_l3simple
INFO Starting slave1_l3simple
INFO Starting pgqd
LOG Starting pgqd 3.2
\end{lstlisting}

Теперь можно увидеть статус кластера:

\begin{lstlisting}[label=lst:londiste-replica13,caption=Статус кластера]
$ londiste3 /etc/skytools/master-londiste.ini status
Queue: replika   Local node: slave1-node

master-node (root)
  |                           Tables: 0/0/0
  |                           Lag: 44s, Tick: 5
  +--: slave1-node (leaf)
                              Tables: 0/0/0
                              Lag: 44s, Tick: 5

$ londiste3 /etc/skytools/master-londiste.ini members
Member info on master-node@replika:
node_name        dead             node_location
---------------  ---------------  --------------------------------
master-node      False            dbname=l3simple host=master-host
slave1-node      False            dbname=l3simple host=slave1-host
\end{lstlisting}

Но репликация еще не запущенна: требуется добавить таблици в очередь, которые мы хотим реплицировать. Для этого используем команду <<add-table>>:

\begin{lstlisting}[label=lst:londiste-replica13,caption=Добавляем таблицы]
$ londiste3 /etc/skytools/master-londiste.ini add-table --all
$ londiste3 /etc/skytools/slave1-londiste.ini add-table --all --create-full
\end{lstlisting}

В данном примере используется параметр <<--all>>, который означает все таблицы,
но вместо него вы можете перечислить список конкретных таблиц, если не хотите
реплицировать все. Если имена таблиц отличаются на master и slave, то можно использовать <<--dest-table>> параметр при добавлении таблиц на slave базе.
Также, если вы не перенесли струкруру таблиц заранее с master на slave базы, то это можно сделать автоматически через <<--create>> параметр (или <<--create-full>>, если нужно перенести полностью всю схему таблицы).

Подобным образом добавляем последовательности (sequences) для репликации:

\begin{lstlisting}[label=lst:londiste-replica14,caption=Добавляем последовательности]
$ londiste3 /etc/skytools/master-londiste.ini add-seq --all
$ londiste3 /etc/skytools/slave1-londiste.ini add-seq --all
\end{lstlisting}

Но последовательности должны на slave базе созданы заранее (тут не поможет <<--create-full>> для таблиц). Поэтому иногда проще перенести точную копию структуры master базы на slave:

\begin{lstlisting}[label=lst:londiste-replica-dump1,caption=Клонирование структуры базы]
$ pg_dump -s -npublic l3simple | psql -hslave1-host l3simple
\end{lstlisting}

Далее проверяем состояние репликации:

\begin{lstlisting}[label=lst:londiste-replica15,caption=Статус кластера]
$ londiste3 /etc/skytools/master-londiste.ini status
Queue: replika   Local node: master-node

master-node (root)
  |                           Tables: 4/0/0
  |                           Lag: 18s, Tick: 12
  +--: slave1-node (leaf)
                              Tables: 0/4/0
                              Lag: 18s, Tick: 12
\end{lstlisting}

Как можно заметить, возле <<Table>> содержится три цифры (x/y/z). Каждая обозначает:

\begin{itemize}
  \item x - количество таблиц в состоянии <<ok>> (replicated). На master базе указывает, что она в норме, а на slave базах - таблица синхронизирована с master базой;
  \item y - количество таблиц в состоянии half (initial copy, not finnished), у master должно быть 0, а у slave базах это указывает количество таблиц в процессе копирования;
  \item z - количество таблиц в состоянии ignored (table not replicated locally), у master должно быть 0, а у slave базах это количество таблиц, которые не добавлены для репликации с мастера (т.е. master отдает на репликацию эту таблицу, но slave их просто не забирает);
\end{itemize}

Через небольшой интервал времени все таблици синхронизировались:

\begin{lstlisting}[label=lst:londiste-replica16,caption=Статус кластера]
$ londiste3 /etc/skytools/master-londiste.ini status
Queue: replika   Local node: master-node

master-node (root)
  |                           Tables: 4/0/0
  |                           Lag: 31s, Tick: 20
  +--: slave1-node (leaf)
                              Tables: 4/0/0
                              Lag: 31s, Tick: 20
\end{lstlisting}

Дополнительно Londiste позволяет просмотреть состояние таблиц и последовательностей на master и slave базах:

\begin{lstlisting}[label=lst:londiste-replica17,caption=Статус таблиц и последовательностей]
$ londiste3 /etc/skytools/master-londiste.ini tables
Tables on node
table_name               merge_state      table_attrs
-----------------------  ---------------  ---------------
public.pgbench_accounts  ok
public.pgbench_branches  ok
public.pgbench_history   ok
public.pgbench_tellers   ok

$ londiste3 /etc/skytools/master-londiste.ini seqs
Sequences on node
seq_name                        local            last_value
------------------------------  ---------------  ---------------
public.pgbench_history_hid_seq  True             33345
\end{lstlisting}



\subsubsection{Проверка}

Для этого буду использовать pgbench утилиту. Запуская добавление данных в таблицу и мониторим логи одновлеменно:

\begin{lstlisting}[label=lst:londiste-check1,caption=Проверка]
$ pgbench -T 10 -c 5 l3simple
$ tail -f /var/log/skytools/slave1_l3simple.log
INFO {count: 1508, duration: 0.307, idle: 0.0026}
INFO {count: 1572, duration: 0.3085, idle: 0.002}
INFO {count: 1600, duration: 0.3086, idle: 0.0026}
INFO {count: 36, duration: 0.0157, idle: 2.0191}
\end{lstlisting}

Как видно по логам slave база успешно реплицируется с master базой.


\subsection{Общие задачи}

\subsubsection{Проверка состояния слейвов}
Этот запрос на мастере дает некоторую информацию о каждой очереди и слейве.
\begin{lstlisting}[language=SQL,label=lst:londiste21,caption=Проверка состояния слейвов]
# SELECT queue_name, consumer_name, lag, last_seen FROM pgq.get_consumer_info();
 queue_name |     consumer_name      |       lag       |    last_seen
------------+------------------------+-----------------+-----------------
 replika    | .global_watermark      | 00:03:37.108259 | 00:02:33.013915
 replika    | slave1_l3simple        | 00:00:32.631509 | 00:00:32.533911
 replika    | .slave1-node.watermark | 00:03:37.108259 | 00:03:05.01431
\end{lstlisting}

<<lag>> столбец показывает отставание от мастера в синхронизации,
<<last\_seen>>~--- время последней запроса от слейва. Значение этого столбца не должно быть больше, чем 60 секунд для конфигурации по умолчанию.

\subsubsection{Удаление очереди всех событий из мастера}
При работе с Londiste может потребоваться удалить все ваши настройки для того, чтобы начать все заново.
Для PGQ, чтобы остановить накопление данных, используйте следующие API:

\begin{lstlisting}[label=lst:londiste22,caption=Удаление очереди всех событий из мастера]
SELECT pgq.unregister_consumer('queue_name', 'consumer_name');
\end{lstlisting}

\subsubsection{Добавление столбца в таблицу}
Добавляем в следующей последовательности:
\begin{enumerate}
 \item добавить поле на все слейвы
 \item BEGIN; -- на мастере
 \item добавить поле на мастере
 \item COMMIT;
\end{enumerate}

\subsubsection{Удаление столбца из таблицы}
\begin{enumerate}
 \item BEGIN; -- на мастере
 \item удалить поле на мастере
 \item COMMIT;
 \item Проверить <<lag>>, когда londiste пройдет момент удаления поля
 \item удалить поле на всех слейвах
\end{enumerate}

Хитрость тут в том, чтобы удалить поле на слейвах только тогда, когда больше нет событий в очереди на это поле.


\subsection{Устранение неисправностей}

\subsubsection{Londiste пожирает процессор и lag растет}
Это происходит, например, если во время сбоя админ забыл перезапустить ticker. Или когда вы сделали большой
UPDATE или DELETE в одной транзакции, но теперь что бы реализовать каждое событие в этом запросе создаются
транзакции на слейвах \dots

Следующий запрос позволяет подсчитать, сколько событий пришло в pgq.subscription в колонках sub\_last\_tick и sub\_next\_tick.
\begin{lstlisting}[language=SQL,label=lst:londiste24,caption=Устранение неисправностей]
SELECT count(*)
  FROM pgq.event_1,
    (SELECT tick_snapshot
      FROM pgq.tick
      WHERE tick_id BETWEEN 5715138 AND 5715139
    ) as t(snapshots)
WHERE txid_visible_in_snapshot(ev_txid, snapshots);
\end{lstlisting}

В нашем случае, это было более чем 5 миллионов и 400 тысяч событий. Чем больше событий
с базы данных требуется обработать Londiste, тем больше ему требуется памяти для этого. Мы можем сообщить
Londiste не загружать все события сразу. Достаточно добавить в INI конфиг PgQ ticker следующую настройку:
\begin{lstlisting}[label=lst:londiste25,caption=Устранение неисправностей]
pgq_lazy_fetch = 500
\end{lstlisting}

Теперь Londiste будет брать максимум 500 событий в один пакет запросов. Остальные попадут в следующие пакеты запросов.
\section{Bucardo}

\subsection{Введение}

Bucardo~--- асинхронная master-master или master-slave репликация PostgreSQL, которая написана на Perl. Система очень гибкая, поддерживает несколько видов синхронизации и обработки конфликтов.

\subsection{Установка}

Установка будет проводиться на Ubuntu Server. Сначала нужно установить \lstinline!DBIx::Safe Perl! модуль.

\begin{lstlisting}[language=Bash,label=lst:bucardo1,caption=Установка]
$ apt-get install libdbix-safe-perl
\end{lstlisting}

Для других систем можно поставить из \href{http://search.cpan.org/CPAN/authors/id/T/TU/TURNSTEP/}{исходников}:

\begin{lstlisting}[language=Bash,label=lst:bucardo2,caption=Установка]
$ tar xvfz dbix_safe.tar.gz
$ cd DBIx-Safe-1.2.5
$ perl Makefile.PL
$ make
$ make test
$ sudo make install
\end{lstlisting}

Теперь ставим сам Bucardo. \href{http://bucardo.org/wiki/Bucardo#Obtaining_Bucardo}{Скачиваем} его и инсталлируем:

\begin{lstlisting}[language=Bash,label=lst:bucardo3,caption=Установка]
$ wget http://bucardo.org/downloads/Bucardo-5.0.0.tar.gz
$ tar xvfz Bucardo-5.4.1.tar.gz
$ cd Bucardo-5.4.1
$ perl Makefile.PL
$ make
$ sudo make install
\end{lstlisting}

Для работы Bucardo потребуется установить поддержку pl/perl языка в PostgreSQL.

\begin{lstlisting}[language=Bash,label=lst:bucardo4,caption=Установка]
$ sudo aptitude install postgresql-plperl-9.5
\end{lstlisting}

и дополнительные пакеты для Perl (DBI, DBD::Pg, Test::Simple, boolean):

\begin{lstlisting}[language=Bash,label=lst:bucardo-packet1,caption=Установка]
$ sudo aptitude install libdbd-pg-perl libboolean-perl
\end{lstlisting}

Теперь можем приступать к настройке репликации.

\subsection{Настройка}

\subsubsection{Инициализация Bucardo}

Запускаем установку Bucardo:

\begin{lstlisting}[language=Bash,label=lst:bucardo5,caption=Инициализация Bucardo]
$ bucardo install
\end{lstlisting}

Во время установки будут показаны настройки подключения к PostgreSQL, которые можно будет изменить:

\begin{lstlisting}[language=Bash,label=lst:bucardo6,caption=Инициализация Bucardo]
This will install the bucardo database into an existing Postgres cluster.
Postgres must have been compiled with Perl support,
and you must connect as a superuser

We will create a new superuser named 'bucardo',
and make it the owner of a new database named 'bucardo'

Current connection settings:
1. Host:          <none>
2. Port:          5432
3. User:          postgres
4. Database:      postgres
5. PID directory: /var/run/bucardo
\end{lstlisting}

После подтверждения настроек, Bucardo создаст пользователя \lstinline!bucardo! и базу данных \lstinline!bucardo!.
Данный пользователь должен иметь право логиниться через Unix socket, поэтому лучше заранее дать ему такие права в \lstinline!pg_hda.conf!.

После успешной установки можно проверить конфигурацию через команду \lstinline!bucardo show all!:

\begin{lstlisting}[language=Bash,label=lst:bucardo-status1,caption=Инициализация Bucardo]
$ bucardo show all
autosync_ddl              = newcol
bucardo_initial_version   = 5.0.0
bucardo_vac               = 1
bucardo_version           = 5.0.0
ctl_checkonkids_time      = 10
ctl_createkid_time        = 0.5
ctl_sleep                 = 0.2
default_conflict_strategy = bucardo_latest
default_email_from        = nobody@example.com
default_email_host        = localhost
default_email_to          = nobody@example.com
...
\end{lstlisting}

\subsubsection{Настройка баз данных}

Теперь нужно настроить базы данных, с которыми будет работать Bucardo. Обозначим базы как \lstinline!master_db! и \lstinline!slave_db!. Реплицировать будем \lstinline!simple_database! базу. Сначала настроим мастер базу:

\begin{lstlisting}[language=Bash,label=lst:bucardo7,caption=Настройка баз данных]
$ bucardo add db master_db dbname=simple_database host=master_host
Added database "master_db"
\end{lstlisting}

Данной командой указали базу данных и дали ей имя \lstinline!master_db! (для того, что в реальной жизни \lstinline!master_db! и \lstinline!slave_db! имеют одинаковое название базы \lstinline!simple_database! и их нужно отличать в Bucardo).

Дальше добавляем \lstinline!slave_db!:

\begin{lstlisting}[language=Bash,label=lst:bucardo8,caption=Настройка баз данных]
$ bucardo add db slave_db dbname=simple_database port=5432 host=slave_host
\end{lstlisting}

\subsubsection{Настройка репликации}

Теперь требуется настроить синхронизацию между этими базами данных. Делается это командой \lstinline!sync!:

\begin{lstlisting}[language=Bash,label=lst:bucardo9,caption=Настройка репликации]
$ bucardo add sync delta dbs=master_db:source,slave_db:target conflict_strategy=bucardo_latest tables=all
Added sync "delta"
Created a new relgroup named "delta"
Created a new dbgroup named "delta"
  Added table "public.pgbench_accounts"
  Added table "public.pgbench_branches"
  Added table "public.pgbench_history"
  Added table "public.pgbench_tellers"
\end{lstlisting}

Данная команда устанавливает Bucardo триггеры в PostgreSQL для master-slave репликации. Значения параметров:

\begin{itemize}
  \item \lstinline!dbs!~--- список баз данных, которые следует синхронизировать. Значение \lstinline!source! или \lstinline!target! указывает, что это master или slave база данных соответственно (их может быть больше одной);

  \item \lstinline!conflict_strategy!~--- для работы в режиме master-master нужно указать как Bucardo должен решать конфликты синхронизации. Существуют следующие стратегии:

  \begin{itemize}
    \item \lstinline!bucardo_source!~--- при конфликте мы копируем данные с source;
    \item \lstinline!bucardo_target!~--- при конфликте мы копируем данные с target;
    \item \lstinline!bucardo_skip!~--- конфликт мы просто не реплицируем. Не рекомендуется для продакшен систем;
    \item \lstinline!bucardo_random!~--- каждая БД имеет одинаковый шанс, что её изменение будет взято для решение конфликта;
    \item \lstinline!bucardo_latest!~--- запись, которая была последней изменена решает конфликт;
    \item \lstinline!bucardo_abort!~--- синхронизация прерывается;
  \end{itemize}

  \item \lstinline!tables!~--- таблици, которые требуется синхронизировать. Через <<all>> указываем все;
\end{itemize}

Для master-master репликации требуется выполнить:

\begin{lstlisting}[language=Bash,label=lst:bucardo10,caption=Настройка репликации]
$ bucardo add sync delta dbs=master_db:source,slave_db:source conflict_strategy=bucardo_latest tables=all
\end{lstlisting}

Пример для создания master-master и master-slave репликации:

\begin{lstlisting}[language=Bash,label=lst:bucardo-master-slave1,caption=Настройка репликации]
$ bucardo add sync delta dbs=master_db1:source,master_db2:source,slave_db1:target,slave_db2:target conflict_strategy=bucardo_latest tables=all
\end{lstlisting}

Для проверки состояния репликации:

\begin{lstlisting}[language=Bash,label=lst:bucardo-master-slave2,caption=Проверка состояния репликации]
$ bucardo status
PID of Bucardo MCP: 12122
 Name    State    Last good    Time     Last I/D    Last bad    Time
=======+========+============+========+===========+===========+=======
 delta | Good   | 13:28:53   | 13m 6s | 3685/7384 | none      |
\end{lstlisting}

\subsubsection{Запуск/Остановка репликации}

Запуск репликации:

\begin{lstlisting}[language=Bash,label=lst:bucardo11,caption=Запуск репликации]
$ bucardo start
\end{lstlisting}

Остановка репликации:

\begin{lstlisting}[language=Bash,label=lst:bucardo12,caption=Остановка репликации]
$ bucardo stop
\end{lstlisting}

\subsection{Общие задачи}

\subsubsection{Просмотр значений конфигурации}

\begin{lstlisting}[language=Bash,label=lst:bucardo13,caption=Просмотр значений конфигурации]
$ bucardo show all
\end{lstlisting}

\subsubsection{Изменения значений конфигурации}

\begin{lstlisting}[language=Bash,label=lst:bucardo14,caption=Изменения значений конфигурациии]
$ bucardo set name=value
\end{lstlisting}

Например:

\begin{lstlisting}[language=Bash,label=lst:bucardo15,caption=Изменения значений конфигурации]
$ bucardo_ctl set syslog_facility=LOG_LOCAL3
\end{lstlisting}

\subsubsection{Перегрузка конфигурации}

\begin{lstlisting}[language=Bash,label=lst:bucardo16,caption=Перегрузка конфигурации]
$ bucardo reload_config
\end{lstlisting}

Более подробную информацию можно найти на \href{http://bucardo.org/}{официальном сайте}.


\subsection{Репликация в другие типы баз данных}

Начиная с версии 5.0 Bucardo поддерживает репликацию в другие источники данных: drizzle, mongo, mysql, oracle, redis и sqlite (тип базы задается при использовании команды \lstinline!bucardo add db! через ключ <<type>>, который по умолчанию postgres). Давайте рассмотрим пример с redis базой. Для начала потребуется установить redis адаптер для Perl (для других баз устанавливаются соответствующие):

\begin{lstlisting}[language=Bash,label=lst:bucardo-redis1,caption=Установка redis]
$ aptitude install libredis-perl
\end{lstlisting}

Далее зарегистрируем redis базу в Bucardo:

\begin{lstlisting}[language=Bash,label=lst:bucardo-redis2,caption=Добавление redis базы]
$ bucardo add db R dbname=simple_database type=redis
Added database "R"
\end{lstlisting}

Создадим группу баз данных под названием \lstinline!pg_to_redis!:

\begin{lstlisting}[language=Bash,label=lst:bucardo-redis3,caption=Группа баз данных]
$ bucardo add dbgroup pg_to_redis master_db:source slave_db:source R:target
Created dbgroup "pg_to_redis"
Added database "master_db" to dbgroup "pg_to_redis" as source
Added database "slave_db" to dbgroup "pg_to_redis" as source
Added database "R" to dbgroup "pg_to_redis" as target
\end{lstlisting}

И создадим репликацию:

\begin{lstlisting}[language=Bash,label=lst:bucardo-redis4,caption=Установка sync]
$ bucardo add sync pg_to_redis_sync tables=all dbs=pg_to_redis status=active
Added sync "pg_to_redis_sync"
  Added table "public.pgbench_accounts"
  Added table "public.pgbench_branches"
  Added table "public.pgbench_history"
  Added table "public.pgbench_tellers"
\end{lstlisting}

После перезапуска Bucardo данные с PostgreSQL таблиц начнуть реплицироватся в Redis:

\begin{lstlisting}[language=Bash,label=lst:bucardo-redis5,caption=Репликация в redis]
$ pgbench -T 10 -c 5 simple_database
$ redis-cli monitor
"HMSET" "pgbench_history:6" "bid" "2" "aid" "36291" "delta" "3716" "mtime" "2014-07-11 14:59:38.454824" "hid" "4331"
"HMSET" "pgbench_history:2" "bid" "1" "aid" "65179" "delta" "2436" "mtime" "2014-07-11 14:59:38.500896" "hid" "4332"
"HMSET" "pgbench_history:14" "bid" "2" "aid" "153001" "delta" "-264" "mtime" "2014-07-11 14:59:38.472706" "hid" "4333"
"HMSET" "pgbench_history:15" "bid" "1" "aid" "195747" "delta" "-1671" "mtime" "2014-07-11 14:59:38.509839" "hid" "4334"
"HMSET" "pgbench_history:3" "bid" "2" "aid" "147650" "delta" "3237" "mtime" "2014-07-11 14:59:38.489878" "hid" "4335"
"HMSET" "pgbench_history:15" "bid" "1" "aid" "39521" "delta" "-2125" "mtime" "2014-07-11 14:59:38.526317" "hid" "4336"
"HMSET" "pgbench_history:14" "bid" "2" "aid" "60105" "delta" "2555" "mtime" "2014-07-11 14:59:38.616935" "hid" "4337"
"HMSET" "pgbench_history:15" "bid" "2" "aid" "186655" "delta" "930" "mtime" "2014-07-11 14:59:38.541296" "hid" "4338"
"HMSET" "pgbench_history:15" "bid" "1" "aid" "101406" "delta" "668" "mtime" "2014-07-11 14:59:38.560971" "hid" "4339"
"HMSET" "pgbench_history:15" "bid" "2" "aid" "126329" "delta" "-4236" "mtime" "2014-07-11 14:59:38.5907" "hid" "4340"
"DEL" "pgbench_tellers:20"
\end{lstlisting}

Данные в Redis хранятся в виде хешей:

\begin{lstlisting}[language=Bash,label=lst:bucardo-redis6,caption=Данные в redis]
$ redis-cli "HGETALL" "pgbench_history:15"
 1) "bid"
 2) "2"
 3) "aid"
 4) "126329"
 5) "delta"
 6) "-4236"
 7) "mtime"
 8) "2014-07-11 14:59:38.5907"
 9) "hid"
10) "4340"
\end{lstlisting}

Также можно проверить состояние репликации:

\begin{lstlisting}[language=Bash,label=lst:bucardo-redis6,caption=Установка redis]
$ bucardo status
PID of Bucardo MCP: 4655
 Name               State    Last good    Time     Last I/D    Last bad    Time
==================+========+============+========+===========+===========+========
 delta            | Good   | 14:59:39   | 8m 15s | 0/0       | none      |
 pg_to_redis_sync | Good   | 14:59:40   | 8m 14s | 646/2546  | 14:59:39  | 8m 15s
\end{lstlisting}


\section{RubyRep}
\subsection{Введение}
RubyRep представляет собой движок для организации асинхронной репликации, написанный на языке ruby.
Основные принципы: простота использования и не зависит от БД.
Поддерживает как master-master, так и master-slave репликацию, может работать с PostgreSQL и MySQL.
Отличительными особенностями решения являются:
\begin{itemize}
\item возможность двухстороннего сравнения и синхронизации баз данных
\item простота установки
\end{itemize}
К недостаткам можно отнести:
\begin{itemize}
\item работа только с двумя базами данных для MySQL
\item медленная работа синхронизации
\item при больших объемах данных <<ест>> процессор и память
\end{itemize}


\subsection{Установка}
RubyRep поддерживает два типа установки: через стандартный Ruby или JRuby.
Рекомендую ставить JRuby вариант~--- производительность будет выше.

\textbf{Установка JRuby версии}

Предварительно должна быть установлена Java (версия 1.6).
\begin{enumerate}
 \item Загрузите последнюю версию JRuby rubyrep c Rubyforge.
 \item Распакуйте
 \item Готово
\end{enumerate}

\textbf{Установка стандартной Ruby версии}
\begin{enumerate}
\item Установить Ruby, Rubygems.
\item Установить драйвера базы данных.

Для Mysql:
\begin{lstlisting}[label=lst:rubyrep1,caption=Установка]
$ sudo gem install mysql
\end{lstlisting}

Для PostgreSQL:
\begin{lstlisting}[label=lst:rubyrep2,caption=Установка]
$ sudo gem install postgres
\end{lstlisting}

\item Устанавливаем rubyrep:
\begin{lstlisting}[label=lst:rubyrep3,caption=Установка]
$ sudo gem install rubyrep
\end{lstlisting}
\end{enumerate}


\subsection{Настройка}
\subsubsection{Создание файла конфигурации}
Выполним команду:
\begin{lstlisting}[label=lst:rubyrep4,caption=Настройка]
rubyrep generate myrubyrep.conf
\end{lstlisting}

Команда generate создала пример конфигурации в файл myrubyrep.conf:
\begin{lstlisting}[label=lst:rubyrep5,caption=Настройка]
RR::Initializer::run do |config|
config.left = {
:adapter  => 'postgresql', # or 'mysql'
:database => 'SCOTT',
:username => 'scott',
:password => 'tiger',
:host     => '172.16.1.1'
}

config.right = {
:adapter  => 'postgresql',
:database => 'SCOTT',
:username => 'scott',
:password => 'tiger',
:host     => '172.16.1.2'
}

config.include_tables 'dept'
config.include_tables /^e/ # regexp matches all tables starting with e
# config.include_tables /./ # regexp matches all tables
end
\end{lstlisting}

В настройках просто разобраться. Базы данных делятся на <<left>> и <<right>>.
Через config.include\_tables мы указываем какие таблицы включать в репликацию (поддерживает RegEx).

\subsubsection{Сканирование баз данных}
Сканирование баз данных для поиска различий:
\begin{lstlisting}[label=lst:rubyrep6,caption=Сканирование баз данных]
rubyrep scan -c myrubyrep.conf
\end{lstlisting}

Пример вывода:
\begin{lstlisting}[label=lst:rubyrep7,caption=Сканирование баз данных]
dept 100% .........................   0
emp 100% .........................   1
\end{lstlisting}

Таблица dept полностью синхронизирована, а emp~--- имеет одну не синхронизированую запись.

\subsubsection{Синхронизация баз данных}
Выполним команду:
\begin{lstlisting}[label=lst:rubyrep8,caption=Синхронизация баз данных]
rubyrep sync -c myrubyrep.conf
\end{lstlisting}

Также можно указать только какие таблицы в базах данных синхронизировать:
\begin{lstlisting}[label=lst:rubyrep9,caption=Синхронизация баз данных]
rubyrep sync -c myrubyrep.conf dept /^e/
\end{lstlisting}

Настройки политики синхронизации позволяют указывать как решать конфликты синхронизации.
Более подробно можно почитать в \href{http://www.rubyrep.org/configuration.html}{документации}.

\subsubsection{Репликация}
Для запуска репликации достаточно выполнить:
\begin{lstlisting}[label=lst:rubyrep10,caption=Репликация]
rubyrep replicate -c myrubyrep.conf
\end{lstlisting}

Данная команда установить репликацию (если она не была установлена) на базы данных и запустит её.
Чтобы остановить репликацию, достаточно просто убить процесс. Даже если репликация остановлена,
все изменения будут обработаны триггерами rubyrep. После перезагрузки, все изменения
будут автоматически восстановлены.

Для удаления репликации достаточно выполнить:
\begin{lstlisting}[label=lst:rubyrep11,caption=Репликация]
rubyrep uninstall -c myrubyrep.conf
\end{lstlisting}

\subsection{Устранение неисправностей}
\subsubsection{Ошибка при запуске репликации}
При запуске rubyrep через Ruby может возникнуть подобная ошибка:
\begin{lstlisting}[label=lst:rubyrep12,caption=Устранение неисправностей]
$rubyrep replicate -c myrubyrep.conf
Verifying RubyRep tables
Checking for and removing rubyrep triggers from unconfigured tables
Verifying rubyrep triggers of configured tables
Starting replication
Exception caught: Thread#join: deadlock 0xb76ee1ac - mutual join(0xb758cfac)
\end{lstlisting}

Это проблема с запусками потоков в Ruby. Решается двумя способами:
\begin{enumerate}
\item Запускать rubyrep через JRuby (тут с потоками не будет проблем)
\item Пофиксить rubyrep патчем:
\begin{lstlisting}[label=lst:rubyrep13,caption=Устранение неисправностей]
--- /Library/Ruby/Gems/1.8/gems/rubyrep-1.1.2/lib/rubyrep/
replication_runner.rb 2010-07-16 15:17:16.000000000 -0400
+++ ./replication_runner.rb 2010-07-16 17:38:03.000000000 -0400
@@ -2,6 +2,12 @@

 require 'optparse'
 require 'thread'
+require 'monitor'
+
+class Monitor
+ alias lock mon_enter
+ alias unlock mon_exit
+end

 module RR
   # This class implements the functionality of the 'replicate' command.
@@ -94,7 +100,7 @@
     # Initializes the waiter thread used for replication pauses
     # and processing
     # the process TERM signal.
     def init_waiter
- @termination_mutex = Mutex.new
+ @termination_mutex = Monitor.new
       @termination_mutex.lock
       @waiter_thread ||= Thread.new {@termination_mutex.lock;
         self.termination_requested = true}
       %w(TERM INT).each do |signal|
\end{lstlisting}
\end{enumerate}


\section{Заключение}

Репликация~--- одна из важнейших частей крупных приложений, которые работают на PostgreSQL. Она помогает распределять нагрузку на базу данных, делать фоновый бэкап одной из копий без нагрузки на центральный сервер, создавать отдельный сервер для логирования и м.д.

В главе было рассмотрено несколько видов репликации PostgreSQL. Нельзя четко сказать какая лучше всех. Потоковая репликация~--- один из самых лучших вариантов для поддержки идентичных кластеров баз данных, но доступна только с 9.0 версии PostgreSQL. Slony-I~--- громоздкая и сложная в настройке система, но имеющая в своем арсенале множество функций, таких как поддержка каскадной репликации, отказоустойчивости (failover) и переключение между серверами (switchover). В тоже время Londiste имея в своем арсенале подобный функционал, может похвастатся еще компактностью и простой в установке. Bucardo~--- система которая может быть или master-master, или master-slave репликацией, но не может обработать огромные обьекты. RubyRep, как для master-master репликации, очень просто в установке и настройке, но за это ему приходится расплачиваться скоростью работы~--- самый
медленный из всех (синхронизация больших объемов данных между таблицами).