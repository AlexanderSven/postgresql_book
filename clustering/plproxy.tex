\section{PL/Proxy}
\label{sec:plproxy}

\href{https://plproxy.github.io/}{PL/Proxy} представляет собой прокси-язык для удаленного вызова процедур и партицирования данных между разными базами. Основная идея его использования заключается в том, что появляется возможность вызывать функции, расположенные в удаленных базах, а также свободно работать с кластером баз данных (например, вызвать функцию на всех узлах кластера, или на случайном узле, или на каком-то одном определенном).

Чем PL/Proxy может быть полезен? Он существенно упрощает горизонтальное масштабирование системы. Становится удобным разделять таблицу с пользователями, например, по первой латинской букве имени~--- на 26 узлов. При этом приложение, которое работает непосредственно с прокси-базой, ничего не будет замечать: запрос на авторизацию, например, сам будет направлен прокси-сервером на нужный узел. То есть администратор баз данных может проводить масштабирование системы практически независимо от разработчиков приложения.

PL/Proxy позволяет полностью решить проблемы масштабирования OLTP систем. В систему легко вводится резервирование с failover-ом не только по узлам, но и по самим прокси-серверам, каждый из которых работает со всеми узлами.

Недостатки и ограничения:

\begin{itemize}
  \item все запросы и вызовы функций вызываются в autocommit-режиме на удаленных серверах;
  \item в теле функции разрешен только один \lstinline!SELECT!. При необходимости нужно писать отдельную процедуру;
  \item при каждом вызове прокси-сервер стартует новое соединение к бакенд-серверу. В высоконагруженных системах целесообразно использовать менеджер для кеширования соединений к бакенд-серверам (для этой цели идеально подходит PgBouncer);
  \item изменение конфигурации кластера (например количества партиций) требует перезапуска прокси-сервера;
\end{itemize}


\subsection{Установка}


\begin{enumerate}
  \item Скачать \href{https://github.com/plproxy/plproxy}{PL/Proxy} и распаковать;
  \item Собрать PL/Proxy командами \lstinline!make! и \lstinline!make install!;
\end{enumerate}

Так же можно установить PL/Proxy из репозитория пакетов. Например в Ubuntu Server достаточно выполнить команду для PostgreSQL 9.6:

\begin{lstlisting}[language=Bash,label=lst:plproxy1,caption=Установка]
$ sudo aptitude install postgresql-9.6-plproxy
\end{lstlisting}


\subsection{Настройка}


Для примера настройки используется 3 сервера PostgreSQL. 2 сервера пусть будут \lstinline!node1! и \lstinline!node2!, а главный, что будет проксировать запросы на два других~--- \lstinline!proxy!. Для корректной работы pl/proxy рекомендуется использовать количество нод равное степеням двойки. База данных будет называтся \lstinline!plproxytest!, а таблица в ней~--- \lstinline!users!.

Для начала настроим \lstinline!node1! и \lstinline!node2!. Команды написанные ниже нужно выполнять на каждой ноде. Сначала создадим базу данных plproxytest (если её ещё нет):

\begin{lstlisting}[language=SQL,label=lst:plproxy2,caption=Настройка]
CREATE DATABASE plproxytest
     WITH OWNER = postgres
       ENCODING = 'UTF8';
\end{lstlisting}

Добавляем табличку \lstinline!users!:

\begin{lstlisting}[language=SQL,label=lst:plproxy3,caption=Настройка]
CREATE TABLE public.users
  (
   username character varying(255),
   email character varying(255)
  )
  WITH (OIDS=FALSE);
ALTER TABLE public.users OWNER TO postgres;
\end{lstlisting}

Теперь создадим функцию для добавления данных в таблицу \lstinline!users!:

\begin{lstlisting}[language=SQL,label=lst:plproxy4,caption=Настройка]
CREATE OR REPLACE FUNCTION public.insert_user(i_username text,
i_emailaddress   text)
RETURNS integer AS
$BODY$
INSERT INTO public.users (username, email) VALUES ($1,$2);
    SELECT 1;
$BODY$
  LANGUAGE 'sql' VOLATILE;
ALTER FUNCTION public.insert_user(text, text) OWNER TO postgres;
\end{lstlisting}

С настройкой нод закончено. Приступим к серверу proxy. Как и на всех нодах, на главном сервере (\lstinline!proxy!) должна присутствовать база данных:

\begin{lstlisting}[language=SQL,label=lst:plproxy5,caption=Настройка]
CREATE DATABASE plproxytest
     WITH OWNER = postgres
       ENCODING = 'UTF8';
\end{lstlisting}

Теперь надо указать серверу что эта база данных управляется с помощью pl/proxy:

\begin{lstlisting}[language=SQL,label=lst:plproxy6,caption=Настройка]
CREATE OR REPLACE FUNCTION public.plproxy_call_handler()
  RETURNS language_handler AS
'$libdir/plproxy', 'plproxy_call_handler'
  LANGUAGE 'c' VOLATILE
COST 1;
ALTER FUNCTION public.plproxy_call_handler()
OWNER TO postgres;
-- language
CREATE LANGUAGE plproxy HANDLER plproxy_call_handler;
CREATE LANGUAGE plpgsql;
\end{lstlisting}

Также, для того что бы сервер знал где и какие ноды у него есть, надо создать 3 сервисные функции, которые pl/proxy будет использовать в своей работе. Первая функция~--- конфиг для кластера баз данных. Тут указываются параметры через key-value:

\begin{lstlisting}[language=SQL,label=lst:plproxy7,caption=Настройка]
CREATE OR REPLACE FUNCTION public.get_cluster_config
(IN cluster_name text,   OUT "key" text, OUT val text)
  RETURNS SETOF record AS
$BODY$
BEGIN
  -- lets use same config for all clusters
  key := 'connection_lifetime';
  val := 30*60; -- 30m
  RETURN NEXT;
  RETURN;
END;
$BODY$
  LANGUAGE 'plpgsql' VOLATILE
  COST 100
  ROWS 1000;
ALTER FUNCTION public.get_cluster_config(text)
OWNER TO postgres;
\end{lstlisting}

Вторая важная функция, код которой надо будет подправить. В ней надо будет указать DSN нод:

\begin{lstlisting}[language=SQL,label=lst:plproxy8,caption=Настройка]
CREATE OR REPLACE FUNCTION
public.get_cluster_partitions(cluster_name text)
  RETURNS SETOF text AS
$BODY$
BEGIN
  IF cluster_name = 'usercluster' THEN
    RETURN NEXT 'dbname=plproxytest host=node1 user=postgres';
    RETURN NEXT 'dbname=plproxytest host=node2 user=postgres';
    RETURN;
  END IF;
  RAISE EXCEPTION 'Unknown cluster';
END;
$BODY$
  LANGUAGE 'plpgsql' VOLATILE
  COST 100
  ROWS 1000;
ALTER FUNCTION public.get_cluster_partitions(text)
OWNER TO postgres;
\end{lstlisting}

И последняя:

\begin{lstlisting}[language=SQL,label=lst:plproxy9,caption=Настройка]
CREATE OR REPLACE FUNCTION
public.get_cluster_version(cluster_name text)
  RETURNS integer AS
$BODY$
BEGIN
  IF cluster_name = 'usercluster' THEN
    RETURN 1;
  END IF;
  RAISE EXCEPTION 'Unknown cluster';
END;
$BODY$
  LANGUAGE 'plpgsql' VOLATILE
  COST 100;
ALTER FUNCTION public.get_cluster_version(text)
OWNER TO postgres;
\end{lstlisting}

Ну и собственно самая главная функция, которая будет вызываться уже непосредственно в приложении:

\begin{lstlisting}[language=SQL,label=lst:plproxy10,caption=Настройка]
CREATE OR REPLACE FUNCTION
public.insert_user(i_username text, i_emailaddress text)
  RETURNS integer AS
$BODY$
  CLUSTER 'usercluster';
  RUN ON hashtext(i_username);
$BODY$
  LANGUAGE 'plproxy' VOLATILE
  COST 100;
ALTER FUNCTION public.insert_user(text, text)
OWNER TO postgres;
\end{lstlisting}

Все готово. Подключаемся к серверу proxy и заносим данные в базу:

\begin{lstlisting}[language=SQL,label=lst:plproxy11,caption=Настройка]
SELECT insert_user('Sven','sven@somewhere.com');
SELECT insert_user('Marko', 'marko@somewhere.com');
SELECT insert_user('Steve','steve@somewhere.com');
\end{lstlisting}

Пробуем извлечь данные. Для этого напишем новую серверную функцию:

\begin{lstlisting}[language=SQL,label=lst:plproxy12,caption=Настройка]
CREATE OR REPLACE FUNCTION
public.get_user_email(i_username text)
 RETURNS SETOF text AS
$BODY$
 CLUSTER 'usercluster';
 RUN ON hashtext(i_username) ;
 SELECT email FROM public.users
 WHERE username = i_username;
$BODY$
 LANGUAGE 'plproxy' VOLATILE
 COST 100
 ROWS 1000;
ALTER FUNCTION public.get_user_email(text)
OWNER TO postgres;
\end{lstlisting}

И попробуем её вызвать:

\begin{lstlisting}[language=SQL,label=lst:plproxy13,caption=Настройка]
SELECT plproxy.get_user_email('Steve');
\end{lstlisting}

Если потом подключиться к каждой ноде отдельно, то можно четко увидеть, что данные \lstinline!users! разбросаны по таблицам каждой ноды.


\subsection{Все ли так просто?}


Как видно на тестовом примере ничего сложного в работе с pl/proxy нет. Но в реальной жизни все не так просто. Представьте что у вас 16 нод. Это же надо как-то синхронизировать код функций. А что если ошибка закрадётся~--- как её оперативно исправлять?

Этот вопрос был задан и на конференции Highload++ 2008, на что Аско Ойя ответил что соответствующие средства уже реализованы внутри самого Skype, но ещё не достаточно готовы для того что бы отдавать их на суд сообществу opensource.

Второй проблема, которая не дай бог коснётся вас при разработке такого рода системы, это проблема перераспределения данных в тот момент когда нам захочется добавить ещё нод в кластер. Планировать эту масштабную операцию придётся очень тщательно, подготовив все сервера заранее, занеся данные и потом в один момент подменив код функции \lstinline!get_cluster_partitions!.
