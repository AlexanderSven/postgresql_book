\section{Postgres-XC}
\label{sec:postgres-xc}

Postgres-XC~-- система для создания мульти-мастер кластеров, работающих в синхронном режиме~-- все узлы всегда содержат актуальные данные. Postgres-XC поддерживает опции для увеличения масштабирования кластера как при преобладании операций записи, так и при основной нагрузке на чтение данных: поддерживается выполнение транзакций с распараллеливанием на несколько узлов, за целостностью транзакций в пределах всего кластера отвечает специальный узел GTM (Global Transaction Manager).

Измерение производительности показало, что КПД кластера Postgres-XC составляет примерно 64\%, т.е. кластер из 10 серверов позволяет добиться увеличения производильности системы в целом в 6.4 раза, относительно производительности одного сервера (цифры приблизительные). 

Система не использует в своей работе триггеры и представляет собой набор дополнений и патчей к PostgreSQL, дающих возможность в прозрачном режиме обеспечить работу в кластере стандартных приложений, без их дополнительной модификации и адаптации (полная совместимость с PostgreSQL API). Кластер состоит из одного управляющего узла (GTM), предоставляющего информацию о состоянии транзакций, и произвольного набора рабочих узлов, каждый из которых в свою очередь состоит из координатора и обработчика данных (обычно эти элементы реализуются на одном сервере, но могут быть и разделены).

Хоть Postgres-XC и выглядит похожим на MultiMaster, но он им не является. Все сервера кластера должны быть соединены сетью с минимальными задержками, никакое географически-распределенное решение с разумной производительностью построить на нем не возможно (это важный момент).

