\section{RubyRep}
\subsection{Введение}
RubyRep представляет собой движок для организации асинхронной репликации, написанный на языке ruby. 
Основные принципы: простота использования и не зависить от БД. 
Поддерживает как master-master, так и master-slave репликацию, может работать с PostgreSQL и MySQL.
Отличительными особенностями решения являются:
\begin{itemize}
\item возможность двухстороннего сравнения и синхронизации баз данных
\item простота установки
\end{itemize}
К недостаткам можно отнести:
\begin{itemize}
\item работа только с двумя базами данных для MySQL
\item медленная работа синхронизации 
\item при больших обьемах данных <<ест>> процессор и память
\end{itemize}


\subsection{Установка}
RubyRep поддерживает два типа установки: через стандартный Ruby или JRuby. 
Рекомендую ставить JRuby вариант~--- производительность будет выше.

\textbf{Установка JRuby версии}

Предварительно должна быть установлена Java (версия 1.6).
\begin{enumerate}
 \item Загрузите последнюю версию JRuby rubyrep c Rubyforge\footnote{http://rubyforge.org/frs/?group\_id=7932, выберите ZIP}.
 \item Распакуйте
 \item Готово
\end{enumerate}

\textbf{Установка стандартной Ruby версии}
\begin{enumerate}
\item Установить Ruby, Rubygems.
\item Установить драйвера базы данных.

Для Mysql: 
\begin{lstlisting}[label=lst:rubyrep1,caption=Установка]
sudo gem install mysql
\end{lstlisting}

Для PostgreSQL:
\begin{lstlisting}[label=lst:rubyrep2,caption=Установка]
sudo gem install postgres
\end{lstlisting}

\item Устанавливаем rubyrep:
\begin{lstlisting}[label=lst:rubyrep3,caption=Установка]
sudo gem install rubyrep
\end{lstlisting}
\end{enumerate}


\subsection{Настройка}
\subsubsection{Создание файла конфигурации}
Выполним команду:
\begin{lstlisting}[label=lst:rubyrep4,caption=Настройка]
rubyrep generate myrubyrep.conf
\end{lstlisting}

Команда generate создала пример конфигурации в файл myrubyrep.conf:
\begin{lstlisting}[label=lst:rubyrep5,caption=Настройка]
RR::Initializer::run do |config|
config.left = {
:adapter  => 'postgresql', # or 'mysql'
:database => 'SCOTT',
:username => 'scott',
:password => 'tiger',
:host     => '172.16.1.1'
}
 
config.right = {
:adapter  => 'postgresql',
:database => 'SCOTT',
:username => 'scott',
:password => 'tiger',
:host     => '172.16.1.2'
}
 
config.include_tables 'dept'
config.include_tables /^e/ # regexp matches all tables starting with e
# config.include_tables /./ # regexp matches all tables
end
\end{lstlisting}

В настройках просто разобраться. Базы данных делятся на <<left>> и <<right>>.
Через config.include\_tables мы указываем какие таблицы включать в репликацию (поддерживает RegEx).

\subsubsection{Сканирование баз данных}
Сканирование баз данных для поиска различий:
\begin{lstlisting}[label=lst:rubyrep6,caption=Сканирование баз данных]
rubyrep scan -c myrubyrep.conf
\end{lstlisting}

Пример вывода:
\begin{lstlisting}[label=lst:rubyrep7,caption=Сканирование баз данных]
dept 100% .........................   0
emp 100% .........................   1
\end{lstlisting}

Таблица dept полностью синхронизирована, а emp~--- имеет одну не синхронизированую запись.

\subsubsection{Синхронизация баз данных}
Выполним команду:
\begin{lstlisting}[label=lst:rubyrep8,caption=Синхронизация баз данных]
rubyrep sync -c myrubyrep.conf
\end{lstlisting}

Также можно указать только какие таблицы в базах данных синхронизировать:
\begin{lstlisting}[label=lst:rubyrep9,caption=Синхронизация баз данных]
rubyrep sync -c myrubyrep.conf dept /^e/
\end{lstlisting}

Настройки политики синхронизации позволяют указывать как решать конфликты синхронизации. 
Более подробно можно почитать в документации http://www.rubyrep.org/configuration.html. 

\subsubsection{Репликация}
Для запуска репликации достаточно выполнить:
\begin{lstlisting}[label=lst:rubyrep10,caption=Репликация]
rubyrep replicate -c myrubyrep.conf
\end{lstlisting}

Данная команда установить репликацию (если она не была установлена) на базы данных и запустит её.
Чтобы остановить репликацию, достаточно просто убить процесс. Даже если репликация остановлена, 
все изменения будут обработаны триггерами rubyrep. После перезагрузки, все изменения 
будут автоматически востановлены.

Для удаления репликации достаточно выполнить:
\begin{lstlisting}[label=lst:rubyrep11,caption=Репликация]
rubyrep uninstall -c myrubyrep.conf
\end{lstlisting}

\subsection{Устранение неисправностей}
\subsubsection{Ошибка при запуске репликации}
При зепуске rubyrep через Ruby может возникнуть подобная ошибка:
\begin{lstlisting}[label=lst:rubyrep12,caption=Устранение неисправностей]
$rubyrep replicate -c myrubyrep.conf 
Verifying RubyRep tables
Checking for and removing rubyrep triggers from unconfigured tables
Verifying rubyrep triggers of configured tables
Starting replication
Exception caught: Thread#join: deadlock 0xb76ee1ac - mutual join(0xb758cfac)
\end{lstlisting}

Это проблема с запусками потоков в Ruby. Решается двумя способами:
\begin{enumerate}
\item Запускать rubyrep через JRuby (тут с потоками не будет проблем)
\item Пофиксить rubyrep патчем:
\begin{lstlisting}[label=lst:rubyrep13,caption=Устранение неисправностей]
--- /Library/Ruby/Gems/1.8/gems/rubyrep-1.1.2/lib/rubyrep/
replication_runner.rb 2010-07-16 15:17:16.000000000 -0400
+++ ./replication_runner.rb 2010-07-16 17:38:03.000000000 -0400
@@ -2,6 +2,12 @@
 
 require 'optparse'
 require 'thread'
+require 'monitor'
+
+class Monitor
+ alias lock mon_enter
+ alias unlock mon_exit
+end
 
 module RR
   # This class implements the functionality of the 'replicate' command.
@@ -94,7 +100,7 @@
     # Initializes the waiter thread used for replication pauses 
     # and processing
     # the process TERM signal.
     def init_waiter
- @termination_mutex = Mutex.new
+ @termination_mutex = Monitor.new
       @termination_mutex.lock
       @waiter_thread ||= Thread.new {@termination_mutex.lock; 
         self.termination_requested = true}
       %w(TERM INT).each do |signal|
\end{lstlisting}
\end{enumerate}