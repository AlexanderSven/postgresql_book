\section{Заключение}
Репликация~--- одна из важнейших частей крупных приложений, которые работают на PostgreSQL. Она помогает 
распределять нагрузку на базу данных, делать фоновый бэкап одной из копий без нагрузки на центральный сервер, 
создавать отдельный сервер для логирования и м.д.

В главе было рассмотрено несколько видов репликации PostgreSQL. Нельзя четко сказать какая лучше всех. 
Slony-I~--- громоздкая и сложная в настройке система, 
но имеющая в своем арсенале множество функций, таких как поддержка каскадной репликации, отказоустойчивости(failover) 
и переключение между серверами (switchover). В тоже время Londiste не обладает подобным функционалом, 
но компактный и прост в установке. Bucardo~--- система которая может быть или master-master, или 
master-slave репликацией, но не может обработать огромные обьекты, нет отказоустойчивости(failover) 
и переключение между серверами (switchover). RubyRep, как для master-master репликации, 
очень просто в установке и настройке, но за это ему приходится расплачиватся скоростью работы~--- самый 
медленный из всех(синхронизация больших обьемов данных между таблицами).