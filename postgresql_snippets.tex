\chapter{Советы по разным вопросам (Performance Snippets)}
\begin{epigraphs}
\qitem{Быстро найти правильный ответ на трудный вопрос~--- ни с чем не сравнимое удовольствие.}{Макс Фрай. Обжора-Хохотун}
\qitem{-- Вопрос риторический.

-- Нет, но он таким кажется, потому что у тебя нет ответа.}{Доктор Хаус (House M.D.), сезон 1 серия 1}
\end{epigraphs}

\section{Введение}
Иногда возникают очень интересные проблемы по работе с PostgreSQL, которые при нахождении ответа поражают своей лаконичностью, 
красотой и простым исполнением (а может и не простым). В данной главе я решил собрать интересные методы решения разных проблем, с 
которыми сталкиваются люди при работе с PostgreSQL. Я не являюсь огромным специалистом по данной теме, поэтому многие решения 
мне помогали находить люди из PostgreSQL комьюнити, а иногда хватало и поиска по Интернету. 

Если вы вледеете интересными методами решения разных проблем, то добавляйте их на данной странице 
\footnote{https://github.com/le0pard/postgresql\_book/issues}.

\section{Советы}

\subsection{Размер объектов в базе данных}

\begin{framed}
Данный запрос показывает размер объектов в базе данных (например таблиц и индексов). PostgreSQL версии >= 8.1.
\end{framed}

\begin{lstlisting}[label=lst:snippets1,caption=Поиск самых больших объектов в БД. SQL запрос]
SELECT nspname || '.' || relname AS "relation",
    pg_size_pretty(pg_relation_size(C.oid)) AS "size"
  FROM pg_class C
  LEFT JOIN pg_namespace N ON (N.oid = C.relnamespace)
  WHERE nspname NOT IN ('pg_catalog', 'information_schema')
  ORDER BY pg_relation_size(C.oid) DESC
  LIMIT 20;
\end{lstlisting}

Код для копирования: https://gist.github.com/910674

Пример вывода:
\begin{lstlisting}[label=lst:snippets2,caption=Поиск самых больших объектов в БД. Пример вывод]
        relation        |    size    
------------------------+------------
 public.accounts        | 326 MB
 public.accounts_pkey   | 44 MB
 public.history         | 592 kB
 public.tellers_pkey    | 16 kB
 public.branches_pkey   | 16 kB
 public.tellers         | 16 kB
 public.branches        | 8192 bytes
\end{lstlisting}

\subsection{Размер самых больших таблиц}

\begin{framed}
Данный запрос показывает размер самых больших таблиц в базе данных. PostgreSQL версии >= 8.1.
\end{framed}

\begin{lstlisting}[label=lst:snippets3,caption=Размер самых больших таблиц. SQL запрос]
SELECT nspname || '.' || relname AS "relation",
    pg_size_pretty(pg_total_relation_size(C.oid)) AS "total_size"
  FROM pg_class C
  LEFT JOIN pg_namespace N ON (N.oid = C.relnamespace)
  WHERE nspname NOT IN ('pg_catalog', 'information_schema')
    AND C.relkind <> 'i'
    AND nspname !~ '^pg_toast'
  ORDER BY pg_total_relation_size(C.oid) DESC
  LIMIT 20;
\end{lstlisting}

Код для копирования: https://gist.github.com/910696

Пример вывода:
\begin{lstlisting}[label=lst:snippets4,caption=Размер самых больших таблиц. Пример вывод]
            relation            | total_size 
--------------------------------+------------
 public.actions                 | 4249 MB
 public.product_history_records | 197 MB
 public.product_updates         | 52 MB
 public.import_products         | 34 MB
 public.products                | 29 MB
 public.visits                  | 25 MB
\end{lstlisting}

\subsection{<<Средний>> count}

\begin{framed}
Данный метод позволяет узнать приблизительное количество записей в таблице. 
Для огромных таблиц этот метод работает быстрее, чем обыкновенный count.
\end{framed}

\begin{lstlisting}[label=lst:snippets5,caption=<<Средний>> count. SQL запрос]
CREATE LANGUAGE plpgsql;
CREATE FUNCTION count_estimate(query text) RETURNS integer AS $$
DECLARE
    rec   record;
    rows  integer;
BEGIN
    FOR rec IN EXECUTE 'EXPLAIN ' || query LOOP
        rows := substring(rec."QUERY PLAN" FROM ' rows=([[:digit:]]+)');
        EXIT WHEN rows IS NOT NULL;
    END LOOP;
 
    RETURN rows;
END;
$$ LANGUAGE plpgsql VOLATILE STRICT;


Testing:


CREATE TABLE foo (r double precision);
INSERT INTO foo SELECT random() FROM generate_series(1, 1000);
ANALYZE foo;

# SELECT count(*) FROM foo WHERE r < 0.1;
 count 
-------
    92
(1 row)

# SELECT count_estimate('SELECT * FROM foo WHERE r < 0.1');
 count_estimate 
----------------
             94
(1 row)
\end{lstlisting}

Код для копирования: https://gist.github.com/910728


