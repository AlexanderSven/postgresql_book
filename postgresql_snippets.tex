\chapter{Полезные мелочи}

\begin{epigraphs}
\qitem{Быстро найти правильный ответ на трудный вопрос~--- ни с чем не сравнимое удовольствие.}{Макс Фрай. Обжора-Хохотун}
\end{epigraphs}

\section{Введение}

Иногда возникают очень интересные проблемы по работе с PostgreSQL, которые при нахождении ответа поражают своей лаконичностью, красотой и простым исполнением (а может и не простым). В данной главе я решил собрать интересные методы решения разных проблем, с которыми сталкиваются люди при работе с PostgreSQL. Я не являюсь огромным специалистом по данной теме, поэтому многие решения мне помогали находить люди из PostgreSQL комьюнити, а иногда хватало и поиска в интернете.



\section{Мелочи}


\subsection{Размер объектов в базе данных}

Данный запрос показывает размер объектов в базе данных (например, таблиц и индексов).

\lstinputlisting[language=SQL,label=lst:snippets1,title=snippets/biggest\_relations.sql]{example_code/snippets/biggest_relations.sql}

Пример вывода:

\begin{lstlisting}[language=SQL,label=lst:snippets2,caption=Поиск самых больших объектов в БД. Пример вывода]
        relation        |    size
------------------------+------------
 public.accounts        | 326 MB
 public.accounts_pkey   | 44 MB
 public.history         | 592 kB
 public.tellers_pkey    | 16 kB
 public.branches_pkey   | 16 kB
 public.tellers         | 16 kB
 public.branches        | 8192 bytes
\end{lstlisting}


\subsection{Размер самых больших таблиц}

Данный запрос показывает размер самых больших таблиц в базе данных.

\lstinputlisting[language=SQL,label=lst:snippets3,title=snippets/biggest\_tables.sql]{example_code/snippets/biggest_tables.sql}

Пример вывода:
\begin{lstlisting}[language=SQL,label=lst:snippets4,caption=Размер самых больших таблиц. Пример вывода]
            relation            | total_size
--------------------------------+------------
 public.actions                 | 4249 MB
 public.product_history_records | 197 MB
 public.product_updates         | 52 MB
 public.import_products         | 34 MB
 public.products                | 29 MB
 public.visits                  | 25 MB
\end{lstlisting}


\subsection{<<Средний>> count}

Данный метод позволяет узнать приблизительное количество записей в таблице. Для огромных таблиц этот метод работает быстрее, чем обыкновенный count.

\lstinputlisting[language=SQL,label=lst:snippets5,title=snippets/count\_estimate.sql]{example_code/snippets/count_estimate.sql}

Пример:

\begin{lstlisting}[language=SQL,label=lst:snippets51,caption=<<Средний>> count. Пример]
CREATE TABLE foo (r double precision);
INSERT INTO foo SELECT random() FROM generate_series(1, 1000);
ANALYZE foo;

# SELECT count(*) FROM foo WHERE r < 0.1;
 count
-------
    92
(1 row)

# SELECT count_estimate('SELECT * FROM foo WHERE r < 0.1');
 count_estimate
----------------
             94
(1 row)
\end{lstlisting}



\subsection{Случайное число из диапазона}

Данный метод позволяет взять случайное число из указаного диапазона (целое или с плавающей запятой).

\lstinputlisting[language=SQL,label=lst:snippets8,title=snippets/random\_from\_range.sql]{example_code/snippets/random_from_range.sql}

Пример:

\begin{lstlisting}[language=SQL,label=lst:snippets9,caption=Случайное число из диапазона. Пример]
SELECT random(1,10)::int, random(1,10);
 random |      random
--------+------------------
      6 | 5.11675184825435
(1 row)

SELECT random(1,10)::int, random(1,10);
 random |      random
--------+------------------
      7 | 1.37060070643201
(1 row)
\end{lstlisting}



\subsection{Алгоритм Луна}

\href{http://en.wikipedia.org/wiki/Luhn\_algorithm}{Алгоритм Луна или формула Луна}~--- алгоритм вычисления контрольной цифры, получивший широкую популярность. Он используется, в частности, при первичной проверке номеров банковских пластиковых карт, номеров социального страхования в США и Канаде. Алгоритм был разработан сотрудником компании <<IBM>> Хансом Петером Луном и запатентован в 1960 году.

Контрольные цифры вообще и алгоритм Луна в частности предназначены для защиты от случайных ошибок, а не преднамеренных искажений данных.

Алгоритм Луна реализован на чистом SQL. Обратите внимание, что эта реализация является чисто арифметической.

\lstinputlisting[language=SQL,label=lst:snippets10,title=snippets/luhn\_algorithm.sql]{example_code/snippets/luhn_algorithm.sql}

Пример:

\begin{lstlisting}[language=SQL,label=lst:snippets11,caption=Алгоритм Луна. Пример]
Select luhn_verify(49927398716);
 luhn_verify
-------------
 t
(1 row)

Select luhn_verify(49927398714);
 luhn_verify
-------------
 f
(1 row)

\end{lstlisting}



\subsection{Выборка и сортировка по данному набору данных}

Выбор данных по определенному набору данных можно сделать с помощью обыкновенного \lstinline!IN!. Но как сделать подобную выборку и отсортировать данные в том же порядке, в котором передан набор данных? Например:

Дан набор: (2,6,4,10,25,7,9). Нужно получить найденные данные в таком же порядке т.е. 2 2 2 6 6 4 4

\lstinputlisting[language=SQL,label=lst:snippets12,title=snippets/order\_like\_in.sql]{example_code/snippets/order_like_in.sql}

где

\lstinline!VALUES(3),(2),(6),(1),(4)!~--- наш набор данных

\lstinline!foo!~-- таблица, из которой идет выборка

\lstinline!foo.catalog_id!~--- поле, по которому ищем набор данных (замена \lstinline!foo.catalog_id IN(3,2,6,1,4)!)



\subsection{Quine~--- запрос который выводит сам себя}

Куайн, квайн (англ. quine)~--- компьютерная программа (частный случай метапрограммирования), которая выдаёт на выходе точную копию своего исходного текста.

\lstinputlisting[language=SQL,label=lst:snippets13,title=snippets/quine.sql]{example_code/snippets/quine.sql}



\subsection{Ускоряем LIKE}

Автокомплит~--- очень популярная фишка в web системах. Реализуется это простым \lstinline!LIKE 'some\%'!, где <<some>>~--- то, что пользователь успел ввести. Проблема в том, что и в огромной таблице (например таблица тегов) такой запрос будет работать очень медленно.

Для ускорения запроса типа \lstinline!LIKE 'bla\%'! можно использовать \lstinline!text_pattern_ops! (или \lstinline!varchar_pattern_ops! если у поле varchar).

\lstinputlisting[language=SQL,label=lst:snippets14,title=snippets/speed\_like.sql]{example_code/snippets/speed_like.sql}



\subsection{Поиск дубликатов индексов}

Запрос находит индексы, созданные на одинаковый набор столбцов (такие индексы эквивалентны, а значит бесполезны).

\lstinputlisting[language=SQL,label=lst:snippets15,title=snippets/duplicate\_indexes.sql]{example_code/snippets/duplicate_indexes.sql}



\subsection{Размер и статистика использования индексов}

\lstinputlisting[language=SQL,label=lst:snippets16,title=snippets/indexes\_statustic.sql]{example_code/snippets/indexes_statustic.sql}



\subsection{Размер распухания (bloat) таблиц и индексов в базе данных}
\label{sec:snippets-bloating}

Запрос, который показывает <<приблизительный>> bloat (раздутие) таблиц и индексов в базе:

\lstinputlisting[language=SQL,label=lst:snippets17,title=snippets/bloating.sql]{example_code/snippets/bloating.sql}
