\chapter{Советы по разным вопросам (Performance Snippets)}
\begin{epigraphs}
\qitem{Быстро найти правильный ответ на трудный вопрос~--- ни с чем не сравнимое удовольствие.}{Макс Фрай. Обжора-Хохотун}
\qitem{-- Вопрос риторический.

-- Нет, но он таким кажется, потому что у тебя нет ответа.}{Доктор Хаус (House M.D.), сезон 1 серия 1}
\end{epigraphs}

\section{Введение}
Иногда возникают очень интересные проблемы по работе с PostgreSQL, которые при нахождении ответа поражают своей лаконичностью, 
красотой и простым исполнением (а может и не простым). В данной главе я решил собрать интересные методы решения разных проблем, с 
которыми сталкиваются люди при работе с PostgreSQL. Я не являюсь огромным специалистом по данной теме, поэтому многие решения 
мне помогали находить люди из PostgreSQL комьюнити, а иногда хватало и поиска по Интернету. 

\section{Советы}

\subsection{Размер объектов в базе данных}
Данный запрос показывает размер объектов в базе данных (например таблиц и индексов).

\begin{lstlisting}[label=lst:snippets1,title=snippets/biggest\_relations.sql]
SELECT nspname || '.' || relname AS "relation",
    pg_size_pretty(pg_relation_size(C.oid)) AS "size"
  FROM pg_class C
  LEFT JOIN pg_namespace N ON (N.oid = C.relnamespace)
  WHERE nspname NOT IN ('pg_catalog', 'information_schema')
  ORDER BY pg_relation_size(C.oid) DESC
  LIMIT 20;
\end{lstlisting}

Пример вывода:
\begin{lstlisting}[label=lst:snippets2,caption=Поиск самых больших объектов в БД. Пример вывода]
        relation        |    size    
------------------------+------------
 public.accounts        | 326 MB
 public.accounts_pkey   | 44 MB
 public.history         | 592 kB
 public.tellers_pkey    | 16 kB
 public.branches_pkey   | 16 kB
 public.tellers         | 16 kB
 public.branches        | 8192 bytes
\end{lstlisting}

\subsection{Размер самых больших таблиц}
Данный запрос показывает размер самых больших таблиц в базе данных.

\begin{lstlisting}[label=lst:snippets3,title=snippets/biggest\_tables.sql]
SELECT nspname || '.' || relname AS "relation",
    pg_size_pretty(pg_total_relation_size(C.oid)) AS "total_size"
  FROM pg_class C
  LEFT JOIN pg_namespace N ON (N.oid = C.relnamespace)
  WHERE nspname NOT IN ('pg_catalog', 'information_schema')
    AND C.relkind <> 'i'
    AND nspname !~ '^pg_toast'
  ORDER BY pg_total_relation_size(C.oid) DESC
  LIMIT 20;
\end{lstlisting}

Пример вывода:
\begin{lstlisting}[label=lst:snippets4,caption=Размер самых больших таблиц. Пример вывода]
            relation            | total_size 
--------------------------------+------------
 public.actions                 | 4249 MB
 public.product_history_records | 197 MB
 public.product_updates         | 52 MB
 public.import_products         | 34 MB
 public.products                | 29 MB
 public.visits                  | 25 MB
\end{lstlisting}

\subsection{<<Средний>> count}
Данный метод позволяет узнать приблизительное количество записей в таблице. 
Для огромных таблиц этот метод работает быстрее, чем обыкновенный count.

\begin{lstlisting}[label=lst:snippets5,title=snippets/count\_estimate.sql]
CREATE LANGUAGE plpgsql;
CREATE FUNCTION count_estimate(query text) RETURNS integer AS $$
DECLARE
    rec   record;
    rows  integer;
BEGIN
    FOR rec IN EXECUTE 'EXPLAIN ' || query LOOP
        rows := substring(rec."QUERY PLAN" FROM ' rows=([[:digit:]]+)');
        EXIT WHEN rows IS NOT NULL;
    END LOOP;
 
    RETURN rows;
END;
$$ LANGUAGE plpgsql VOLATILE STRICT;
\end{lstlisting}

Пример:
\begin{lstlisting}[label=lst:snippets51,caption=<<Средний>> count. Пример]
CREATE TABLE foo (r double precision);
INSERT INTO foo SELECT random() FROM generate_series(1, 1000);
ANALYZE foo;

# SELECT count(*) FROM foo WHERE r < 0.1;
 count 
-------
    92
(1 row)

# SELECT count_estimate('SELECT * FROM foo WHERE r < 0.1');
 count_estimate 
----------------
             94
(1 row)
\end{lstlisting}

\subsection{Узнать значение по умолчанию у поля в таблице}
Данный метод позволяет узнать значение по умолчанию у поля в таблице (заданное через DEFAULT).

\begin{lstlisting}[label=lst:snippets6,title=snippets/default\_value.sql]
CREATE OR REPLACE FUNCTION ret_def(text,text,text) RETURNS text AS $$
SELECT 
  COLUMNS.column_default::text
FROM 
  information_schema.COLUMNS
  WHERE table_name = $2
  AND table_schema = $1
  AND column_name = $3
$$ LANGUAGE sql IMMUTABLE;
\end{lstlisting}

Пример:
\begin{lstlisting}[label=lst:snippets7,caption=Узнать значение по умолчанию у поля в таблице. Пример]
# SELECT ret_def('schema','table','column');

SELECT ret_def('public','image_files','id');
                 ret_def                 
-----------------------------------------
 nextval('image_files_id_seq'::regclass)
(1 row)

SELECT ret_def('public','schema_migrations','version');
 ret_def 
---------
 
(1 row)
\end{lstlisting}

\subsection{Случайное число из диапазона}
Данный метод позволяет взять случайное(random) число из указаного диапазона (целое или с плавающей запятой).

\begin{lstlisting}[label=lst:snippets8,title=snippets/random\_from\_range.sql]
CREATE OR REPLACE FUNCTION random(numeric, numeric)
RETURNS numeric AS
$$
   SELECT ($1 + ($2 - $1) * random())::numeric;
$$ LANGUAGE 'sql' VOLATILE;
\end{lstlisting}

Пример:
\begin{lstlisting}[label=lst:snippets9,caption=Случайное число из диапазона. Пример]
SELECT random(1,10)::int, random(1,10);
 random |      random      
--------+------------------
      6 | 5.11675184825435
(1 row)

SELECT random(1,10)::int, random(1,10);
 random |      random      
--------+------------------
      7 | 1.37060070643201
(1 row)
\end{lstlisting}

\subsection{Алгоритм Луна}
Алгоритм Луна или формула Луна\footnote{http://en.wikipedia.org/wiki/Luhn\_algorithm}~--- алгоритм вычисления контрольной цифры, получивший широкую популярность. 
Он используется, в частности, при первичной проверке номеров банковских пластиковых карт, номеров социального 
страхования в США и Канаде. Алгоритм был разработан сотрудником компании <<IBM>> Хансом Петером Луном и 
запатентован в 1960 году.

Контрольные цифры вообще и алгоритм Луна в частности предназначены для защиты от случайных ошибок, 
а не преднамеренных искажений данных.

Алгоритм Луна реализован на чистом SQL. Обратите внимание, что эта реализация является чисто арифметической.

\begin{lstlisting}[label=lst:snippets10,title=snippets/luhn\_algorithm.sql]
CREATE OR REPLACE FUNCTION luhn_verify(int8) RETURNS BOOLEAN AS $$
-- Take the sum of the
-- doubled digits and the even-numbered undoubled digits, and see if
-- the sum is evenly divisible by zero.
SELECT
         -- Doubled digits might in turn be two digits. In that case,
         -- we must add each digit individually rather than adding the
         -- doubled digit value to the sum. Ie if the original digit was
         -- `6' the doubled result was `12' and we must add `1+2' to the
         -- sum rather than `12'.
         MOD(SUM(doubled_digit / INT8 '10' + doubled_digit % INT8 '10'), 10) = 0
FROM
-- Double odd-numbered digits (counting left with
-- least significant as zero). If the doubled digits end up
-- having values
-- > 10 (ie they're two digits), add their digits together.
(SELECT
         -- Extract digit `n' counting left from least significant
         -- as zero
         MOD( ( $1::int8 / (10^n)::int8 ), 10::int8)
         -- Double odd-numbered digits
         * (MOD(n,2) + 1)
         AS doubled_digit
         FROM generate_series(0, CEIL(LOG( $1 ))::INTEGER - 1) AS n
) AS doubled_digits;
 
$$ LANGUAGE 'SQL'
IMMUTABLE
STRICT;
 
COMMENT ON FUNCTION luhn_verify(int8) IS 'Return true iff the last digit of the
input is a correct check digit for the rest of the input according to Luhn''s
algorithm.';
CREATE OR REPLACE FUNCTION luhn_generate_checkdigit(int8) RETURNS int8 AS $$
SELECT
     -- Add the digits, doubling even-numbered digits (counting left
     -- with least-significant as zero). Subtract the remainder of
     -- dividing the sum by 10 from 10, and take the remainder
     -- of dividing that by 10 in turn.
     ((INT8 '10' - SUM(doubled_digit / INT8 '10' + doubled_digit % INT8 '10') %
                       INT8 '10') % INT8 '10')::INT8
FROM (SELECT
         -- Extract digit `n' counting left from least significant\
         -- as zero
         MOD( ($1::int8 / (10^n)::int8), 10::int8 )
         -- double even-numbered digits
         * (2 - MOD(n,2))
         AS doubled_digit
         FROM generate_series(0, CEIL(LOG($1))::INTEGER - 1) AS n
) AS doubled_digits;
 
$$ LANGUAGE 'SQL'
IMMUTABLE
STRICT;
 
COMMENT ON FUNCTION luhn_generate_checkdigit(int8) IS 'For the input
value, generate a check digit according to Luhn''s algorithm';
CREATE OR REPLACE FUNCTION luhn_generate(int8) RETURNS int8 AS $$
SELECT 10 * $1 + luhn_generate_checkdigit($1);
$$ LANGUAGE 'SQL'
IMMUTABLE
STRICT;
 
COMMENT ON FUNCTION luhn_generate(int8) IS 'Append a check digit generated
according to Luhn''s algorithm to the input value. The input value must be no
greater than (maxbigint/10).';
CREATE OR REPLACE FUNCTION luhn_strip(int8) RETURNS int8 AS $$
SELECT $1 / 10;
$$ LANGUAGE 'SQL'
IMMUTABLE
STRICT;
 
COMMENT ON FUNCTION luhn_strip(int8) IS 'Strip the least significant digit from
the input value. Intended for use when stripping the check digit from a number
including a Luhn''s algorithm check digit.';
\end{lstlisting}

Пример:
\begin{lstlisting}[label=lst:snippets11,caption=Алгоритм Луна. Пример]
Select luhn_verify(49927398716);
 luhn_verify 
-------------
 t
(1 row)

Select luhn_verify(49927398714);
 luhn_verify 
-------------
 f
(1 row)

\end{lstlisting}

\subsection{Выборка и сортировка по данному набору данных}
Выбор данных по определенному набору данных можно сделать с помощью обыкновенного IN. Но как сделать подобную выборку и отсортировать 
данные в том же порядке, в котором передан набор данных. Например:

Дан набор: (2,6,4,10,25,7,9)
Нужно получить найденные данные в таком же порядке т.е. 2 2 2 6 6 4 4

\begin{lstlisting}[label=lst:snippets12,title=snippets/order\_like\_in.sql]
SELECT foo.* FROM foo                                           
JOIN (SELECT id.val, row_number() over() FROM (VALUES(3),(2),(6),(1),(4)) AS
id(val)) AS id
ON (foo.catalog_id = id.val) ORDER BY row_number;
\end{lstlisting}

где

VALUES(3),(2),(6),(1),(4)~--- наш набор данных

foo~-- таблица, из которой идет выборка

foo.catalog\_id~--- поле по которому ищем набор данных (замена foo.catalog\_id IN(3,2,6,1,4))

\subsection{Куайн, Запрос который выводит сам себя}
Куайн, квайн (англ. quine)~--- компьютерная программа (частный случай метапрограммирования), 
которая выдаёт на выходе точную копию своего исходного текста. 

\begin{lstlisting}[label=lst:snippets13,title=snippets/quine.sql]
select a || ' from (select ' || quote_literal(a) || b || ', ' || quote_literal(b) || '::text as b) as quine' from 
(select 'select a || '' from (select '' || quote_literal(a) || b || '', '' || quote_literal(b) || ''::text as b) as 
quine'''::text as a, '::text as a'::text as b) as quine;
\end{lstlisting}

\subsection{Ускоряем LIKE}
Автокомплит~--- очень популярная фишка в web системах. Реализуется это простым LIKE <<some\%>>, 
где <<some>>~--- то, что пользователь успел ввести. Проблема в том, что и огромной таблице 
(например таблица тегов) такой запрос будет очень медленный.

Для ускорения запроса типа <<LIKE 'bla\%'>> можно использовать text\_pattern\_ops 
(или varchar\_pattern\_ops если у поле varchar).

\begin{lstlisting}[label=lst:snippets14,title=snippets/speed\_like.sql]
prefix_test=# create table tags (
prefix_test(#  tag    text primary key,
prefix_test(#  name      text not null,
prefix_test(#  shortname text,
prefix_test(#  status    char default 'S',
prefix_test(# 
prefix_test(#  check( status in ('S', 'R') )
prefix_test(# );
NOTICE:  CREATE TABLE / PRIMARY KEY will create implicit index "tags_pkey" for table "tags"
CREATE TABLE
prefix_test=# CREATE INDEX i_tag ON tags USING btree(lower(tag)  text_pattern_ops);
CREATE INDEX

prefix_test=# create table invalid_tags (
prefix_test(#  tag    text primary key,
prefix_test(#  name      text not null,
prefix_test(#  shortname text,
prefix_test(#  status    char default 'S',
prefix_test(# 
prefix_test(#  check( status in ('S', 'R') )
prefix_test(# );
NOTICE:  CREATE TABLE / PRIMARY KEY will create implicit index "invalid_tags_pkey" for table "invalid_tags"
CREATE TABLE



prefix_test=# select count(*) from tags;
 count 
-------
 11966
(1 row)

prefix_test=# select count(*) from invalid_tags;
 count 
-------
 11966
(1 row)

# EXPLAIN ANALYZE select * from invalid_tags where lower(tag) LIKE lower('0146%');
                                                 QUERY PLAN                                                 
------------------------------------------------------------------------------------------------------------
 Seq Scan on invalid_tags  (cost=0.00..265.49 rows=60 width=26) (actual time=0.359..20.695 rows=1 loops=1)
   Filter: (lower(tag) ~~ '0146%'::text)
 Total runtime: 20.803 ms
(3 rows)

# EXPLAIN ANALYZE select * from invalid_tags where lower(tag) LIKE lower('0146%');
                                                 QUERY PLAN                                                 
------------------------------------------------------------------------------------------------------------
 Seq Scan on invalid_tags  (cost=0.00..265.49 rows=60 width=26) (actual time=0.549..19.503 rows=1 loops=1)
   Filter: (lower(tag) ~~ '0146%'::text)
 Total runtime: 19.550 ms
(3 rows)

# EXPLAIN ANALYZE select * from tags where lower(tag) LIKE lower('0146%');
                                                      QUERY PLAN                                                       
-----------------------------------------------------------------------------------------------------------------------
 Bitmap Heap Scan on tags  (cost=5.49..97.75 rows=121 width=26) (actual time=0.054..0.057 rows=1 loops=1)
   Filter: (lower(tag) ~~ '0146%'::text)
   ->  Bitmap Index Scan on i_tag (cost=0.00..5.46 rows=120 width=0) (actual time=0.032..0.032 rows=1 loops=1)
         Index Cond: ((lower(tag) ~>=~ '0146'::text) AND (lower(tag) ~<~ '0147'::text))
 Total runtime: 0.119 ms
(5 rows)

# EXPLAIN ANALYZE select * from tags where lower(tag) LIKE lower('0146%');
                                                      QUERY PLAN                                                       
-----------------------------------------------------------------------------------------------------------------------
 Bitmap Heap Scan on tags  (cost=5.49..97.75 rows=121 width=26) (actual time=0.025..0.025 rows=1 loops=1)
   Filter: (lower(tag) ~~ '0146%'::text)
   ->  Bitmap Index Scan on i_tag (cost=0.00..5.46 rows=120 width=0) (actual time=0.016..0.016 rows=1 loops=1)
         Index Cond: ((lower(tag) ~>=~ '0146'::text) AND (lower(tag) ~<~ '0147'::text))
 Total runtime: 0.050 ms
(5 rows)


\end{lstlisting}

\subsection{Поиск дубликатов индексов}
Запрос находит индексы, созданные на одинаковый набор столбцов (индексы эквивалентны, а значит бесполезны).

\begin{lstlisting}[label=lst:snippets15,title=snippets/duplicate\_indexes.sql]
SELECT pg_size_pretty(sum(pg_relation_size(idx))::bigint) AS size,
       (array_agg(idx))[1] AS idx1, (array_agg(idx))[2] AS idx2,
       (array_agg(idx))[3] AS idx3, (array_agg(idx))[4] AS idx4
FROM (
    SELECT indexrelid::regclass AS idx, (indrelid::text ||E'\n'|| indclass::text ||E'\n'|| indkey::text ||E'\n'||
                                         coalesce(indexprs::text,'')||E'\n' || coalesce(indpred::text,'')) AS KEY
    FROM pg_index) sub
GROUP BY KEY HAVING count(*)>1
ORDER BY sum(pg_relation_size(idx)) DESC;
\end{lstlisting}

\subsection{Размер и статистика использования индексов}

\begin{lstlisting}[label=lst:snippets16,title=snippets/indexes\_statustic.sql]
SELECT
    t.tablename,
    indexname,
    c.reltuples AS num_rows,
    pg_size_pretty(pg_relation_size(quote_ident(t.tablename)::text)) AS table_size,
    pg_size_pretty(pg_relation_size(quote_ident(indexrelname)::text)) AS index_size,
    CASE WHEN x.is_unique = 1  THEN 'Y'
       ELSE 'N'
    END AS UNIQUE,
    idx_scan AS number_of_scans,
    idx_tup_read AS tuples_read,
    idx_tup_fetch AS tuples_fetched
FROM pg_tables t
LEFT OUTER JOIN pg_class c ON t.tablename=c.relname
LEFT OUTER JOIN
       (SELECT indrelid,
           max(CAST(indisunique AS integer)) AS is_unique
       FROM pg_index
       GROUP BY indrelid) x
       ON c.oid = x.indrelid
LEFT OUTER JOIN
    ( SELECT c.relname AS ctablename, ipg.relname AS indexname, x.indnatts AS number_of_columns, idx_scan, idx_tup_read, idx_tup_fetch,indexrelname FROM pg_index x
           JOIN pg_class c ON c.oid = x.indrelid
           JOIN pg_class ipg ON ipg.oid = x.indexrelid
           JOIN pg_stat_all_indexes psai ON x.indexrelid = psai.indexrelid )
    AS foo
    ON t.tablename = foo.ctablename
WHERE t.schemaname='public'
ORDER BY 1,2;
\end{lstlisting}

